\sectionthree{RSA security}
\begin{python0}
from solutions import *; clear()
\end{python0}

First of all Bob has to be able to select two primes. Note that
the primes must be large. 

Why? 

First of all note that Eve has $(N,e)$ since it is the public key.
What does Eve want?
She wants the decryption key $(N, d)$ in order to decrypt messages, i.e.,
she wants $d$.
Both $e$ and $d$ are integers in mod $\phi(N)$, i.e., $0 < d < \phi(N)$. 
Recall that
\[
 \phi(N) = N \prod_{p \mid N } \biggl( 1 - \frac{1}{p}\biggr)
\]
For the case $N = pq$, 
\[
\phi(N) = (p-1)(q-1)
\]
Therefore if Eve has $p$ and $q$, she can compute
$\phi(N)$.
After that she can compute the multiplicative inverse of 
$e$ mod $\phi(N)$ using the Extended Euclidean algorithm.
Therefore Eve might try to factor $N$ to get $p$ and $q$ in order to get $\phi(N)$.

The obvious brute force method is to try to divide $N$ by 2, 3, $\ldots$.
So in summary here's what Eve might want to try:

\textsc{Eve's Dream 1}.
\begin{tightlist}
\item[1.] Factor $N$ to obtain $p$ and $q$
\item[2.] Compute $\phi(N) = (p-1)(q-1)$
\item[3.] Compute the multiplicative inverse $d$ of $e$ in mod $\phi(N)$ using
 the Euclidean algorithm.
\end{tightlist}

Note that the only reason why she needs $p$ and $q$ is so that she can compute 
$\phi(N)$.
What if somehow Eve got a hold of $\phi(N)$?
Then she can proceed onto step 3.
She doesn't really care about the primes $p$ and $q$ in this case.
The important is to compute $d$.
And to compute $d$ she needs to know $\phi(N)$.

\textsc{Eve's Dream 2}.
\begin{tightlist}
\item[1.] Compute $\phi(N)$ from $N$
\item[2.] Compute the multiplicative inverse $d$ of $e$ in mod $\phi(N)$ using
 the Euclidean algorithm.
\end{tightlist}

So here's an important question:
\textit{Is there a way to compute $\phi(N)$ without factorizing $N$?}


There's no known fast way to computing $\phi(N)$ from $N$
other than by definition
or through the factorization $N$.
Without the above, Eve is back to square 1, i.e., 
she has to factorize $N$.
And since the naive approach is to try to divide $N$ by $2, 3, 4, \ldots$,
the goal of Bob (who's setting up the RSA parameters) is to make sure that $p$
and $q$ are huge primes.


\begin{ex} 
  \label{ex:rsa-10}
  \tinysidebar{\debug{exercises/{nt-61/question.tex}}}
  In your \verb!Zmod.py!, complete the following methods:
  \begin{enumerate}[nosep]
    \li multiplicative inverse mod $N$ (i.e., \texttt{inv})
    \li invertibility mod $N$ (i.e., \texttt{is\_invertible})
    \li division (i.e., \texttt{\_\_div\_\_})
  \end{enumerate}

  \solutionlink{sol:rsa-10}
  \qed
\end{ex} 
\begin{python0}
from solutions import *
add(label="ex:rsa-10",
    srcfilename='exercises/rsa-10/answer.tex') 
\end{python0}

 
However ... the interesting thing is this:
If Eve has $N = pq$ and $\phi(N)$,
she can very quickly compute the primes $p$ and $q$.
Why?


\begin{ex} 
  \label{ex:rsa-10}
  \tinysidebar{\debug{exercises/{nt-61/question.tex}}}
  In your \verb!Zmod.py!, complete the following methods:
  \begin{enumerate}[nosep]
    \li multiplicative inverse mod $N$ (i.e., \texttt{inv})
    \li invertibility mod $N$ (i.e., \texttt{is\_invertible})
    \li division (i.e., \texttt{\_\_div\_\_})
  \end{enumerate}

  \solutionlink{sol:rsa-10}
  \qed
\end{ex} 
\begin{python0}
from solutions import *
add(label="ex:rsa-10",
    srcfilename='exercises/rsa-10/answer.tex') 
\end{python0}


\begin{comment}
Note that for $N = pq$
\[
\phi(N) = (p-1)(q-1) = pq - (p+q) + 1
\]
You see the sum and product of $p$ and $q$.
The sum and product of two quantities appear in a
\textit{very} familiar 
context.
Hmmm ... where have we seen this before?
Look at this:
\[
(x - a)(x - b) = x^2 - (a+b)x + (ab)
\] 
The sum and product of roots of a quadratic equation appears in the 
coefficients
of the polynomial (if the leading coefficient is 1).
For our case we have
\[
(x-p)(x-q) = x^2 - (p+q)x + (pq)
\]
But $pq = N$ and 
from
\[
\phi(N) = pq - (p+q) + 1
\]
we have
\[
p+q = pq - \phi(N) + 1 = n - \phi(N) + 1
\]
Putting this into our polynomial we get
\begin{align*}
(x-p)(x-q) 
&= x^2 - (p+q)x + (pq) \\
&= x^2 - (N - \phi(N) + 1)x + N
\end{align*}
The point: the polynomial $(x-p)(x-q)$ has coefficients that depends only on 
$N$ and $\phi(N)$.

So what does this mean for Eve?
If she has $N$ and $\phi(N)$, she can write down the polynomial
\begin{align*}
x^2 - (N - \phi(N) + 1) + N
\end{align*}
She can also solve for the roots of this polynomial (using the quadratic 
equation formula)
and therefore obtain the primes of $N$.
\end{comment}


Hence Bob must be able to generate large random primes in order to not let Eve
factorize $N$.
This is the main attack on RSA: Factorization of $N$.

Well ... \lq\lq large'' is kind of vague.
Just how big do we need the primes to be?

Suppose you have two primes of roughly 100 digits each.
Multiplying them together you get a number $n$ that roughly 200 digits long, 
i.e. $10^{200}$.
Suppose Eve can divide such a number by a trial divisor really fast, say she 
can perform such divisions
at a rate of $1,000,000,000$ per second, i.e. $10^9$ per second.
She would need to try the numbers up to roughly the square root of $10^{200}$,
 i.e. $10^{100}$.
In terms of number of years the amount of time needed is
\[
10^{100} / 10^9 / (60 \cdot 60 \cdot 24 \cdot 365)
\]
which is
\[
{\scriptsize 317097919837645865043125317097919837645865043125317097919837645865043125317097919837}
\]
i.e., roughly $10^{85}$ years (give or take a century or two!)
She won't be around that long.
In general, this methods requires
$O(n^{1/2})$ divisions.

There are many intricate factorization algorithms
other than brute force testing 2, 3, 5, ...: quadratic
sieve, elliptic curve factorization, number field sieve,
generalized number field sieve, etc. 
The fastest method known is the generalized number field sieve (GNFS)
with a expected runtime of
\[
  e^{(c + o(1)) (\ln n)^{1/3} (\ln \ln n)^{2/3}}
\]
where $c = (64/9)^{1/3}$.
It is superpolynomial (more than polynomial)
and subexponential (less than exponential).
(Example:
$n^2$ is polynomial,
$2^n$ is exponential, 
but $2^{\sqrt{n}}$ is greater than polynomial
but less than exponential.)
The $o(1)$ means a number that becomes 0 when $n$ grows.
Formally $f(n) = o(1)$ means
for any constant $c \neq 0$,
$f(n)/c \rightarrow 0$ as $n \rightarrow \infty$.
For instance $1/n = o(1)$.

In general public key cryptosystems
(and some other security protocols)
involve the solution of some complex
mathematical problem. In the case of RSA, the hard problem is
integer factorization. In order to gain credence, such companies
usually publish challenges on their web site. For instance if you
go to
\url{http://www.rsasecurity.com/rsalabs/node.asp?id=2093}
you'd find a list of number for you to factorize. The latest
challenge to be factored was the RSA-576 number, a 576-bit or
174-digit number. It was broken in December 2003. The prize for
factoring RSA-576 was \$10,000. The next challenge is RSA-640:
\begin{align*}
 & 31074 18240 49004 37213 50750 03588 85679 30037 34602 28427 \\
 & 27545 72016 19488 23206 44051 80815 04556 34682 96717 23286 \\
 & 78243 79162 72838 03341 54710 73108 50191 95485 29007 33772 \\
 & 48227 83525 74238 64540 14691 73660 24776 5234 6609
\end{align*}
This challenge is worth \$30,000. If this is too small for you you
can try the largest RSA challenge, RSA-2048:
\begin{align*}
& 25195908475657893494027183240048398571429282126204 \\
& 03202777713783604366202070759555626401852588078440 \\
& 69182906412495150821892985591491761845028084891200 \\
& 72844992687392807287776735971418347270261896375014 \\
& 97182469116507761337985909570009733045974880842840 \\
& 17974291006424586918171951187461215151726546322822 \\
& 16869987549182422433637259085141865462043576798423 \\
& 38718477444792073993423658482382428119816381501067 \\
& 48104516603773060562016196762561338441436038339044 \\
& 14952634432190114657544454178424020924616515723350 \\
& 77870774981712577246796292638635637328991215483143 \\
& 81678998850404453640235273819513786365643912120103 \\
& 97122822120720357
\end{align*}
This baby is worth \$200,000.


In order to defeat certain factorization techniques, researchers
have suggested the following guidelines for key generation:
\begin{enumerate}
 \item $pq$ should have at least $1024$ bits.
 \item $p$ and $q$ should have roughly the same size
 \item Both $p-1$ and $q-1$ should have a large prime factor
 \item $\gcd(p-1,q-1)$ should be small
 \item $d > n^{0.292}$
\end{enumerate}

Integer factorization and RSA security is obviously a very hot
area of research since it is used so widely. You can find many
theoretical results on attacking RSA online. For instance if $pq$
has $n$ bits and Eve knows either the least significant or most
significant $n/4$ bits, then she can factorize $pq$. If you want
to learn more about number theory and RSA talk to me.

