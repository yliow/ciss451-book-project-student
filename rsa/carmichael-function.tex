\sectionthree{Carmichael function}
\begin{python0}
from solutions import *; clear()
\end{python0}

\begin{defn}
  The \defone{multiplicative order} of $a$ mod $n$, if it exists,
  is the smallest positive integer $k$
  such that $a^k \equiv 1 \pmod{n}$.
\end{defn}

Some values might not have multiplicative order.
For instance $0^k$ is not $\equiv 1 \pmod{n}$ for all $k$.
Recall that 
$a$ has a multiplicative inverse mod $n$
iff $a^k \equiv 1 \pmod{n}$ for some $k > 0$.

Recall Euler's theorem: If $\gcd(a, n) = 1$, then
\[
a^{\phi(n)} \equiv 1 \pmod{n}
\]
Is $\phi(n)$ the best possible in the sense that $\phi(n)$ gives you
the smallest for the above to be true?

For instance when $n = 2$, $\phi(2) = 1$ which is the smallest
possible positive integer to satisfy
\[
a^{\phi(n)} \equiv 1 \pmod{n}
\]
For $n = 3$, $\phi(3) = 2$ and
\begin{align*}
  1^1 \equiv 1, \,\,\, 2^1 \equiv 2 \pmod{3} \\
  1^2 \equiv 1, \,\,\, 2^2 \equiv 1 \pmod{3}
\end{align*}
So $\phi(3) = 2$ is the smallest for
\[
a^{\phi(n)} \equiv 1 \pmod{n}
\]
to be true for both $a = 1$ and $a = 2$.
Etc.
But when you reach $n = 8$ when $\phi(8) = 4$,
if $\gcd(a, 8) = 1$, then $a = 1, 3, 5, 7$ and
\begin{align*}
  1^2 &\equiv 1 \pmod{8} \\
  3^2 &\equiv 1 \pmod{8} \\
  5^2 &\equiv 1 \pmod{8} \\
  7^2 &\equiv 1 \pmod{8}
\end{align*}
and $2 < 4$ (in fact $2 \mid 4$).
In other words $2$ satisfies
\[
a^2 \equiv 1 \pmod{8}
\]
for all $a$ such that $\gcd(a, 8) = 1$.
Of course we know from Euler's theorem that
\[
a^{\phi{(8)}} \equiv 1 \pmod{8}
\]
and $\phi(8) = 4$.
Clearly 
\[
a^2 \equiv 1 \pmod{8}
\implies
a^4 \equiv 1 \pmod{8}
\]
Let's write $\lambda(8) = 4$.
In general:

\begin{defn}
  Define the
  \defone{Carmichael function}
  \[
  \lambda(n)
  \]
  to be the $\operatorname{LCM}$ (lowest common multiple)
  of the multiplicative order of $a$ mod $n$
  for all $a \in \{1, 2, ..., n\}$
  satisfying $\gcd(a, n)$.
  The multiplicative order of $a$ is the smallest positive integer $k$
  such that $a^k \equiv 1 \pmod{n}$.
\end{defn}

In other words, $\lambda(n)$ is \lq\lq almost" $\phi(n)$.

\begin{thm}
  \mbox{}
  \begin{enumerate}[nosep]
  \item[\textnormal{(a)}]
    $\lambda(mn) = \lambda(m) \lambda(n)$ if $\gcd(m,n) = 1$.
  \item[\textnormal{(b)}]
    Let $p$ be a prime and $k \geq 0$. Then
    \[
    \lambda(p^k)
    =
    \begin{cases}
      \phi(p^k) & \text{ if $p > 2$ } \\
      \phi(p^k) & \text{ if $p = 2, k = 0, 1, 2$ } \\
      \frac{1}{2} \phi(p^k) & \text{ if $p = 2, k \geq 3$ } \\      
    \end{cases}
    \]
  \end{enumerate}
\end{thm}

\begin{thm}
  \mbox{}
  \begin{enumerate}[nosep]
  \item[\textnormal{(a)}] Let $\gcd(a, n) = 1$, if $a^k \equiv 1 \pmod{n}$, then
    $k \mid \lambda(n)$.
  \item[\textnormal{(b)}] If
    \[
    a^k \equiv 1 \pmod{n} \,\,\, \text{ for all $\gcd(a, n) = 1$}
    \]
    then
    \[
    \lambda(n) \mid k
    \]
  \item[\textnormal{(c)}] $\lambda(n) \mid \phi(n)$.
  \end{enumerate}
\end{thm}
