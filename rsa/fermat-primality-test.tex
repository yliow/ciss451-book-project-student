\sectionthree{Fermat primality test}
\begin{python0}
from solutions import *; clear()
\end{python0}

\newcommand\abcde{$\texttt{a}^\texttt{n - 1} \not\equiv \texttt{1} \ \texttt{(mod n)}$}

Recall that Fermat's Little Theorem says
that 
\[
  \text{$p$ prime}, \,\,\,
  p \nmid a \implies a^{p-1} \equiv 1 \pmod{p}
\]
I can also say that if $1 \leq a \leq p - 1$,
\[
\text{$p$ prime}
\implies
a^{p-1} \equiv 1 \pmod{p}
\]
And of course we have:
if
$1 \leq a \leq p - 1$,
\[
\text{$p$ not prime}
\impliedby
a^{p-1} \not\equiv 1 \pmod{p}
\]
Replacing \lq\lq $p$" by \lq\lq $n$",
we have the following:

\begin{prop} \textnormal{(\defone{Fermat compositeness test})}
If there is some $a$ such that $1 \leq a \leq n - 1$ and 
\[
a^{n-1} \not\equiv 1 \pmod{n}
\]
then $n$ is composite.
\end{prop}

Although the proposition above is true in general,
when used in an algorithm, there are several cases that we want to remove.
\begin{enumerate}[nosep]
\item
  Obviously $a = 1$ does not satisfy the hypothesis of the proposition.
  Therefore the interval for $a$ should be $[2, n-1]$ instead of $[1, n-1]$.
\item
  Also, the hypothesis does not hold if $a = n-1$ and $n$ is odd.
  For the case when $n$ is even, either $n = 2$ or $n = 2k$ where $k > 1$.
  If the goal is to show $n$ is composite, the case of $n = 2$ and
  $n = 2k$ where $k > 1$ can be checked quickly.
  Therefore the only useful scenario to use the proposition if now
  when $n$ is odd and $a \in [2, n-2]$.
\item
  For the condition \lq\lq $2 \leq a \leq n - 2$" to have any $a$ at
  at all, we need $2 \leq n - 2$, i.e., $n \geq 4$.
\end{enumerate}
Altogether the proposition is useful in an algorithm
when $n > 3$ is odd and when $a \in [2, n - 2]$.

\begin{Verbatim}[frame=single,fontsize=\footnotesize,commandchars=\\\{\}]
ALGORITHM: Fermat-compositeness-test
INPUTS: n -- number to be tested for compositeness
        t -- number of tries
OUTPUT: "n is composite" if n is composite

if n <= 2: return "n is not composite"
if n % 2 is 0 and n / 2 > 1: return "n is composite"

for i = 1, 2, 3, ..., t:
    pick random a in [2, n - 2]
    if \abcde return "n is composite"

return "no conclusion"
\end{Verbatim}
Note that if you return \verb!"no conclusion"!, it means either $n$
is prime or $n$ is composite but unfortunately your random
$a$ did not pick an $a$ satisfying
$a^{n-1} \not\equiv 1 \pmod{n}$.

\begin{defn}
  Suppose $n$ is composite and let $a \in [1, n-1]$.
  If $a$ satisfies
  \[
  a^{n - 1} \not\equiv 1 \pmod{n}
  \]
  then $a$ is called a \defone{Fermat witness} for the compositeness of $n$.
  Otherwise, if
  \[
  a^{n - 1} \equiv 1 \pmod{n}
  \]
  then $a$ is called a \defone{Fermat liar}.
  The curious reason for calling such an $a$ a liar will be clarified later.
\end{defn}

Note that the above test tells you that $n$ is composite
\textit{without} factoring $n$.
Let's compare the above Fermat compositeness test to other compositeness test.
\begin{enumerate}[nosep]
  \li Division compositeness test:
  Randomly pick $a$ such that $2 \leq a \leq \floor{\sqrt{n}}$.
  If $a$ divides $n$, then $n$ is a composite.
  If the chance of finding $a$ is low, then one would have to
  run $a$ through $[2, \floor{\sqrt{n}}]$.
  In general, iteration rather than randomization is used.
  \li GCD compositeness test:
  Randomly pick $a$ such that $2 \leq a \leq \floor{\sqrt{n}}$,
  if $\gcd(a, n) > 1$, then
  $n$ is composite.
  If the chance of finding $a$ is low, then one would have to
  run $a$ through $[2, \floor{\sqrt{n}}]$.
\end{enumerate}
Note that in both of these tests, a divisor of $n$ is found.
The speed of the above three compositeness tests depends on how fast we can
find some $a$ satisfying some condition in the test.

To better understand the condition
\[
a^{n-1} \equiv 1 \pmod{n}
\]
where $a \in [2, n - 2]$, here's a table:
\begin{Verbatim}[frame=single,fontsize=\footnotesize]
   3 1  None 1 ...
   4 0   0.0 0 ..0.
   5 1   1.0 1 ..11.
   6 0   0.0 0 ..000.
   7 1   1.0 1 ..1111.
   8 0   0.0 0 ..00000.
   9 0   0.0 0 ..000000.
  10 0   0.0 0 ..0000000.
  11 1   1.0 1 ..11111111.
  12 0   0.0 0 ..000000000.
  13 1   1.0 1 ..1111111111.
  14 0   0.0 0 ..00000000000.
  15 0  0.17 0 ..001000000100.
  16 0   0.0 0 ..0000000000000.
  17 1   1.0 1 ..11111111111111.
  18 0   0.0 0 ..000000000000000.
  19 1   1.0 1 ..1111111111111111.
  20 0   0.0 0 ..00000000000000000.
  21 0  0.11 0 ..000000100001000000.
  22 0   0.0 0 ..0000000000000000000.
  23 1   1.0 1 ..11111111111111111111.
  24 0   0.0 0 ..000000000000000000000.
  25 0  0.09 0 ..0000010000000000100000.
  26 0   0.0 0 ..00000000000000000000000.
  27 0   0.0 0 ..000000000000000000000000.
  28 0  0.08 0 ..0000000100000000000000010.
  29 1   1.0 1 ..11111111111111111111111111.
  30 0   0.0 0 ..000000000000000000000000000.
  31 1   1.0 1 ..1111111111111111111111111111.
  32 0   0.0 0 ..00000000000000000000000000000.
  33 0  0.07 0 ..000000001000000000000100000000.
  34 0   0.0 0 ..0000000000000000000000000000000.
  35 0  0.06 0 ..00001000000000000000000000010000.
  36 0   0.0 0 ..000000000000000000000000000000000.
  37 1   1.0 1 ..1111111111111111111111111111111111.
  38 0   0.0 0 ..00000000000000000000000000000000000.
  39 0  0.06 0 ..000000000000100000000001000000000000.
  40 0   0.0 0 ..0000000000000000000000000000000000000.
  41 1   1.0 1 ..11111111111111111111111111111111111111.
  42 0   0.0 0 ..000000000000000000000000000000000000000.
  43 1   1.0 1 ..1111111111111111111111111111111111111111.
  44 0   0.0 0 ..00000000000000000000000000000000000000000.
  45 0  0.14 0 ..000000100000000101000000101000000001000000.
  46 0   0.0 0 ..0000000000000000000000000000000000000000000.
  47 1   1.0 1 ..11111111111111111111111111111111111111111111.
  48 0   0.0 0 ..000000000000000000000000000000000000000000000.
  49 0  0.09 0 ..0000000000000000110000000000110000000000000000.
  50 0   0.0 0 ..00000000000000000000000000000000000000000000000.
\end{Verbatim}
Here are the descriptions of the columns:
\begin{enumerate}[nosep]
  \li First column: $n$
  \li Third column: 1--prime, 0--composite
  \li Fifth column: sequence of boolean values for
  $a^{n-1} \equiv 1 \pmod{n}$
  where $a = 0, 1, 2, ..., n - 1$, except that if
  $a$ is not in $[2, n-2]$,
  the value is \lq\lq ." to indicate \lq\lq not applicable".
  The \lq\lq not applicable" values are $a = 0, 1, n-1$.
  \li Second column: The percentage of \lq\lq 1"s in the fourth column
  (not counting the \lq\lq not applicable" cases).
  \li Third column: 1--percentage of witnesses is 0\%, 0--otherwise.
  \li Fourth column: 1--if values in fifth column are all 1.
  (Ignore this for now.)
\end{enumerate}
Of course when $n$ is prime, the sequence of boolean values are all
\lq\lq $1$"s.
Now let us focus on the composite $n$'s.

For $n = 15$:
\begin{Verbatim}[frame=single,fontsize=\footnotesize]
  15 0  0.17 0 ..001000000100.
\end{Verbatim}
there are 10 Fermat witnesses for the compositeness of $15$
(i.e., 2, 3, 5, 6, 7, 8, 9, 10, 12, 13)
and 2 Fermat liars (i.e., 4, 11).
For instance $2$ is a Fermat witness since
\[
2^{15 - 1} = 2^{14} = (2^4)^3 \cdot 2^2 = (16)^3 \cdot 4 \equiv 1^3 \cdot 4 = 4 \not\equiv 1 \pmod{16}
\]
and $4$ is a Fermat liar since
\[
4^{15 - 1} = 4^{14} = (4^2)^7 = (16)^7 \equiv 1^7 = 1 \pmod{16}
\]
The percentage of Fermat liars (among integers $a \in [2, 13]$) is $17\%$.
Therefore there are more Fermat witnesses than Fermat liars.
In fact, when $n$ is composite,
the fact that there are more Fermat witnesses than not seems to be very common.
The first time the percentage of Fermat liars is $> 50\%$ is
when $n = 561$ (at $57\%$).

This means that when $n$ is composite
and we randomly pick an $a$ such that $a \in [2, n-2]$,
the chance that $a$ is a Fermat witness to $n$'s compositeness
seems be to very high.
The percentage of Fermat witnesses will be quantified later.


\begin{ex} 
  \label{ex:hill-10}
  \tinysidebar{\debug{exercises/{rsa-43/question.tex}}}
  What if you test for the condition $n = x^3 - y^3$?
  Is there are version of
  factorization using difference of cubes?
  %We have
  %\[
  %  x^3 - y^3 = (x - y)(x^2 + xy + y^2)
  %\]
  %Suppose $n = abc$.
  %\[
  %  \frac{}{}
  %\]

  \solutionlink{sol:hill-10}
  \qed
\end{ex} 
\begin{python0}
from solutions import *
add(label="ex:hill-10",
    srcfilename='exercises/hill-10/answer.tex') 
\end{python0}


Compared to the brute force division compositeness test,
suppose $n = pq$ with $p < q$,
then in $[2, \floor{\sqrt{n}}]$, there's only one divisor.
For instance if $n = 11 \cdot 13 = 143$, $[2, \floor{\sqrt{n}}] = [2, 11]$.
and the only divisor of $n$ in this interval is $11$.
The percentage of finding a divisor (with one try) in this case is $10\%$.
The chance of finding some $a$ such that $\gcd(a, n) > 1$ is the same.
If $n = pq$ where $q$ is much larger than $p$, then the GCD compositeness
test has a better change of finding a good $a$, because
$a = p, 2p, 3p, ...kp$ would work where $kp$ is the largest integer
$\leq \sqrt{pq}$.
For instance is $n = 2 \cdot 13$, then $a = 2, 4 \leq \floor{\sqrt{2 \cdot 13}} = 5$ works
so the number of good $a$s is $2/4 = 50\%$.
However if you look at the table above, you will see that for $n = 26$,
the percentage of Fermat witness is $100\%$.

For Fermat compositeness test, from the above table,
the entry for $n = 143$ is
\begin{Verbatim}[frame=single,fontsize=\footnotesize]
143 0  0.01 0 ..0000000000100000000000000000000000000000000000000000000000
00000000000000000000000000000000000000000000000000000000000000000000000100
00000000.
\end{Verbatim}
There's a $99\%$ chance of finding a Fermat witness (with one try).


\begin{ex} 
  \label{ex:hill-10}
  \tinysidebar{\debug{exercises/{rsa-43/question.tex}}}
  What if you test for the condition $n = x^3 - y^3$?
  Is there are version of
  factorization using difference of cubes?
  %We have
  %\[
  %  x^3 - y^3 = (x - y)(x^2 + xy + y^2)
  %\]
  %Suppose $n = abc$.
  %\[
  %  \frac{}{}
  %\]

  \solutionlink{sol:hill-10}
  \qed
\end{ex} 
\begin{python0}
from solutions import *
add(label="ex:hill-10",
    srcfilename='exercises/hill-10/answer.tex') 
\end{python0}


\begin{ex} 
  \label{ex:hill-10}
  \tinysidebar{\debug{exercises/{rsa-43/question.tex}}}
  What if you test for the condition $n = x^3 - y^3$?
  Is there are version of
  factorization using difference of cubes?
  %We have
  %\[
  %  x^3 - y^3 = (x - y)(x^2 + xy + y^2)
  %\]
  %Suppose $n = abc$.
  %\[
  %  \frac{}{}
  %\]

  \solutionlink{sol:hill-10}
  \qed
\end{ex} 
\begin{python0}
from solutions import *
add(label="ex:hill-10",
    srcfilename='exercises/hill-10/answer.tex') 
\end{python0}


The above Fermat compositeness test tells you when $n$ is a composite.
It does not tell you if $n$ is a prime.
One can ask if the following is true: 
\[
a^{n-1} \equiv 1 \pmod{n} \implies n \text{ is prime}
\]
if $n$ is odd and $a \in [2, n-2]$.
This is clearly not true.
For instance in the above data for $n = 143$, you see that
there are two Fermat liars, the first being $a = 12$:
\[
12^{143 - 1} = 12^{142} = (12^2)^{71} = 144^{71} \equiv 1^{71} = 1 \pmod{143}
\]
The first composite $n$ to have a Fermat liar is $n = 15$, where the liar is $a = 4$:
\begin{Verbatim}[frame=single,fontsize=\footnotesize]
15 0  0.17 0 ..001000000100.
\end{Verbatim}
\[
4^{15 - 1} = 4^{14} = 16^7 \equiv 1^7 \equiv 1 \pmod{15} 
\]  
In this case,
we say that $15$ is a
\defterm{Fermat pseudoprime}\index{Fermat pseudoprime}
and
$4$ is a
\defterm{Fermat liar}\index{Fermat liar}\tinysidebar{Fermat liar \\ Fermat pseudoprime \\ base}
for $15$.
(Sometimes $4$ is called a
\defterm{base}\index{base} for $15$.)
Why is $15$ called a pseudoprime and $4$ is a liar?
The fact
\[
4^{15 - 1} \equiv 1 \pmod{15} 
\]
is very similar to Fermat Little Theorem where the modulus is a prime $p$ and $p \nmid a$:
\[
a^{p - 1} \equiv 1 \pmod{p} 
\]
$4$ and $15$ are trying to trick you into thinking that $15$ is like a prime. 

Note that if $a$ is a Fermat liar, then it is invertible:

\begin{prop}
  \mbox{}
  \begin{enumerate}[nosep,label=\textnormal{(\alph*)}]
    \item
      If $a$ is a Fermat liar for composite $n$,
      then $a$ is invertible mod $n$, i.e., $\gcd(a, n) = 1$.
    \item
      Therefore if $\gcd(a, n) > 1$, then $a$ cannot be a Fermat liar.
      In other words, if $\gcd(a, n) > 1$, then $a$ is a Fermat witness.
  \end{enumerate}
\end{prop}
\proof
(a)
TODO

(b) This follows from (a).
\qed

Therefore for $a \in [0, n-1]$ (although for compositeness testing
we are only interested in the case when $n > 3$ is odd and $a \in [2, n-2]$):
\begin{enumerate}
  \item $a$ is a Fermat witness and $\gcd(a, n) = 1$
  \item $a$ is a Fermat witness and $\gcd(a, n) > 1$
  \item $a$ is a Fermat liar (and in this case $\gcd(a, n) = 1$)
\end{enumerate}
So in the above table, to be \lq\lq fair" to Fermat liar,
if one is interested in comparing the number of
Fermat liars against Fermat witnesses for $n$,
sometimes one can temporarily ignore the $a$ such that
$\gcd(a, n) > 1$.
If we do this, then
there are composite numbers $n$ where \textit{every} $a$ in $[2, n-2]$
such that $\gcd(a, n) = 1$ is a Fermat liar for $n$.
Such an $n$ is called a \defone{Carmichael number}.

Let me collect all these definitions below.

\begin{defn}
  Let $n$ be a composite.
  Then
  \begin{enumerate}[nosep]
  \item[(a)] $n$ is a
    \defterm{Fermat pseudoprime}\index{Fermat pseudoprime}\tinysidebar{Fermat pseudoprime}
    if $n$ is not a prime and 
    if there is some $a \in [2, n-2]$ such that 
    \[
    a^{n-1} \equiv 1 \pmod{n}
    \]
    In this case, we say that $a$ is a
    \defone{Fermat liar} for $n$.
    ($a$ is also called a \defone{base} for $n$.)
  \item[(b)] $n$ is a \defone{Carmichael number}
    if every $a \in [2, n-2]$ such that $\gcd(a, n) = 1$
    is a Fermat liar for $n$, i.e., for all such $a$,
    \[
    a^{n-1} \equiv 1 \pmod{n}
    \]
  \end{enumerate}
\end{defn}

The following is similar to the earlier table:
\begin{Verbatim}[frame=single,fontsize=\footnotesize]
   3 1  None 1 ...
   4 0  None 1 ....
   5 1   1.0 1 ..11.
   6 0  None 1 ......
   7 1   1.0 1 ..1111.
   8 0   0.0 0 ...0.0..
   9 0   0.0 0 ..0.00.0.
  10 0   0.0 0 ...0...0..
  11 1   1.0 1 ..11111111.
  12 0   0.0 0 .....0.0....
  13 1   1.0 1 ..1111111111.
  14 0   0.0 0 ...0.0...0.0..
  15 0  0.33 0 ..0.1..00..1.0.
  16 0   0.0 0 ...0.0.0.0.0.0..
  17 1   1.0 1 ..11111111111111.
  18 0   0.0 0 .....0.0...0.0....
  19 1   1.0 1 ..1111111111111111.
  20 0   0.0 0 ...0...0.0.0.0...0..
  21 0   0.2 0 ..0.00..1.00.1..00.0.
  22 0   0.0 0 ...0.0.0.0...0.0.0.0..
  23 1   1.0 1 ..11111111111111111111.
  24 0   0.0 0 .....0.0...0.0...0.0....
  25 0  0.11 0 ..000.0100.0000.0010.000.
  26 0   0.0 0 ...0.0.0.0.0...0.0.0.0.0..
  27 0   0.0 0 ..0.00.00.00.00.00.00.00.0.
  28 0   0.2 0 ...0.0...1.0.0.0.0.0...0.1..
  29 1   1.0 1 ..11111111111111111111111111.
  30 0   0.0 0 .......0...0.0...0.0...0......
  31 1   1.0 1 ..1111111111111111111111111111.
  32 0   0.0 0 ...0.0.0.0.0.0.0.0.0.0.0.0.0.0..
  33 0  0.11 0 ..0.00.00.1..00.00.00..1.00.00.0.
  34 0   0.0 0 ...0.0.0.0.0.0.0...0.0.0.0.0.0.0..
  35 0  0.09 0 ..000.1.00.000..0000..000.00.1.000.
  36 0   0.0 0 .....0.0...0.0...0.0...0.0...0.0....
  37 1   1.0 1 ..1111111111111111111111111111111111.
  38 0   0.0 0 ...0.0.0.0.0.0.0.0...0.0.0.0.0.0.0.0..
  39 0  0.09 0 ..0.00.00.00..1.00.00.00.1..00.00.00.0.
  40 0   0.0 0 ...0...0.0.0.0...0.0.0.0...0.0.0.0...0..
  41 1   1.0 1 ..11111111111111111111111111111111111111.
  42 0   0.0 0 .....0.....0.0...0.0...0.0...0.0.....0....
  43 1   1.0 1 ..1111111111111111111111111111111111111111.
  44 0   0.0 0 ...0.0.0.0...0.0.0.0.0.0.0.0.0.0...0.0.0.0..
  45 0  0.27 0 ..0.0..01..0.00.01.1..00..1.10.00.0..10..0.0.
  46 0   0.0 0 ...0.0.0.0.0.0.0.0.0.0...0.0.0.0.0.0.0.0.0.0..
  47 1   1.0 1 ..11111111111111111111111111111111111111111111.
  48 0   0.0 0 .....0.0...0.0...0.0...0.0...0.0...0.0...0.0....
  49 0   0.1 0 ..00000.000000.000110.000000.011000.000000.00000.
  50 0   0.0 0 ...0...0.0.0.0...0.0.0.0...0.0.0.0...0.0.0.0...0..
\end{Verbatim}
except that for the fifth column,
an entry of \lq\lq ." is used to indicate \lq\lq not applicable",
where \lq\lq not applicable" is when
$a \in \{0, 1, n-1\}$ or when $\gcd(a, n) > 1$.
The fourth column is $1$ if all values of $a \in [2, n - 1]$
satisfying $\gcd(a, n) = 1$ are Fermat liars.
In this case $n$ is a Carmichael number
and the value for the third column (percentage of Fermat liars)
is $1.0 = 100\%$.


\begin{ex} 
  \label{ex:hill-10}
  \tinysidebar{\debug{exercises/{rsa-43/question.tex}}}
  What if you test for the condition $n = x^3 - y^3$?
  Is there are version of
  factorization using difference of cubes?
  %We have
  %\[
  %  x^3 - y^3 = (x - y)(x^2 + xy + y^2)
  %\]
  %Suppose $n = abc$.
  %\[
  %  \frac{}{}
  %\]

  \solutionlink{sol:hill-10}
  \qed
\end{ex} 
\begin{python0}
from solutions import *
add(label="ex:hill-10",
    srcfilename='exercises/hill-10/answer.tex') 
\end{python0}


The first Carmichael number is $561$. Here's the row for $561$
from the table:
\begin{Verbatim}[frame=single,fontsize=\footnotesize]
561 0   1.0 1 ..1.11.11.1..11.1..11..1.11.11.11..1.11.11.1..11.11.11..1.11
.11.11.1..11.11.1..11.11..1..1.11.11.11.11.11.11.1..11.11.1...1.11.11.11.1
1..1.11.1..11.11.11..1.11.11.11.11.1..11.1..11.11.11..1.11.11.11.11.11.11.
1..11.11.11....11.11.11.11.11..1.1..11.11.11..1.11.11.11.11.11.1..1..11.11
.11..1..1.11.11.11.11.11.1..11.11.11..1.1..11.11.11.11.11....11.11.11..1.1
1.11.11.11.11.11.1..11.11.11..1.11..1.11.11.11.11.1..11.11.11..1.11.1..11.
11.11.11.1...1.11.11..1.11.11.11.11.11.11.1..1..11.11..1.11.11..1.11.11.11
.1..11.11.11..1.11.11.1..11.11.11.1..11..1.11..1.11.11.1.
\end{Verbatim}

The concept of Carmichael numbers was named after
\href{https://en.wikipedia.org/wiki/Robert_Daniel_Carmichael}{Carmichael}
because of a 1910 paper he wrote.
However the defining property of a Carmichael number was studied as far back
as at least 1885.
For instance the first 6 Carmichael numbers (including 561) were
first discovered by
\href{https://en.wikipedia.org/wiki/V%C3%A1clav_%C5%A0imerka}{\v{S}imerka}
in 1885.
In his 1910 paper,
Carmichael conjectured that there are infinitely many Carmichael numbers.
This was not proven until 1994
(see \url{https://math.dartmouth.edu/~carlp/PDF/paper95.pdf}).
Here's the wikipedia entry for
\href{https://en.wikipedia.org/wiki/Carmichael_number}{Carmichael numbers}.


\begin{ex} 
  \label{ex:hill-10}
  \tinysidebar{\debug{exercises/{rsa-43/question.tex}}}
  What if you test for the condition $n = x^3 - y^3$?
  Is there are version of
  factorization using difference of cubes?
  %We have
  %\[
  %  x^3 - y^3 = (x - y)(x^2 + xy + y^2)
  %\]
  %Suppose $n = abc$.
  %\[
  %  \frac{}{}
  %\]

  \solutionlink{sol:hill-10}
  \qed
\end{ex} 
\begin{python0}
from solutions import *
add(label="ex:hill-10",
    srcfilename='exercises/hill-10/answer.tex') 
\end{python0}


The above gives us a probabilistic primality test:
if $n > 3$ and we pick $a \in [2, n-2]$ and we have
\[
a^{n-1} \equiv 1 \pmod{n}
\]
then $n$ is \lq\lq probably a prime".
Why?
Because
\[
a^{n-1} \equiv 1 \pmod{n}
\]
can occur in two ways:
\begin{enumerate}[nosep]
\item $n$ is a prime (the fact that $n \nmid a$ is guaranteed by the fact that
  $n$ is prime and $a \in [2, n-2]$)
\item $n$ is a Fermat pseudoprime and $a$ is Fermat liar for $n$
\end{enumerate}
From the first table, we see that most of the \lq\lq 1" occurs when
$n$ is a prime.
For a fixed $n$,
the more $a$'s we try, the higher the probability that $n$ is prime,
unless of course our $n$ has lots of liars.
In particular, in the worse scenario, if $n$ is a Carmichael number,
the test will say $n$ is very likely a prime when in fact it's not.

Combining the Fermat compositeness test with the \lq\lq probably prime test",
we now have:

\begin{prop} \textnormal{(\defone{Fermat Primality Test})}
  If there is some $a$ such that $1 \leq a \leq n - 1$ and
  \[
  a^{n-1} \not\equiv 1 \pmod{n}
  \]
  then $n$ is composite.
  If there is some randomly chosen $a \in [2, n-2]$ such that 
  \[
  a^{n-1} \equiv 1 \pmod{n}
  \]
  then $n$ is \lq\lq probably a prime".
\end{prop}

Here's the Fermat primality test algorithm:
\begin{Verbatim}[frame=single,fontsize=\footnotesize,commandchars=\\\{\}]
ALGORITHM: Fermat-primality-test
INPUTS: n -- number to be tested for compositeness/primeness. Assume n >= 2.
        t -- number of tries
OUTPUT: "n is composite" or "n is probably prime" (after t tries)

if n is 2: return "n is prime"
if n % 2 is 0 and n / 2 > 1: return "n is composite"

for i = 1, 2, 3, ..., t:
    pick random a in [2, n - 2]
    if \abcde return "n is composite"

return "n is probably prime"
\end{Verbatim}
Note that we have not included the handling of $n \leq 1$.
In general, primality tests are for handling cases when $n$ is large.
We have added a note that $n$ is assumed to be $\geq 2$.

Note that the first 2 checks (i.e., $n$ is even) can be omitted:
\begin{Verbatim}[frame=single,fontsize=\footnotesize,commandchars=\\\{\}]
ALGORITHM: Fermat-primality-test
INPUTS: n -- number to be tested for compositeness/primeness. Assume n >= 2.
        t -- number of tries
OUTPUT: "n is composite" or "n is probably prime" (after t tries)

for i = 1, 2, 3, ..., t:
    pick random a in [2, n - 2]
    if \abcde return "n is composite"

return "n is probably prime"
\end{Verbatim}
Why?
Because if $n = 2$, then $[2, n - 2]$ is empty, which means that
\texttt{"n is probably prime"} is returned.
And if $n = 2k$ where $k > 1$, then $\gcd(a, n) > 1$
whenever $a$ is even or when $a = k$.
This means that more than half of the values in $[2, n-2]$
are Fermat witnesses, so there is a strong likeliheed that
\verb!"n is composite"! will be returned especially if $t > 1$.

Note that when given an integer $n \geq 0$, either $n$ is a prime
or it is not.
The statement \lq\lq n is probably a prime" should actually be
\lq\lq there's probably a value in $[2, n-2]$
satisfying some condition".
But the phrase \lq\lq n is probably a prime" has been in use for a long
time and it's hard to break the usage.

When comparing number of Fermat witnesses against Fermat liar,
we can be a little bit more precise:

\begin{prop}
  Let $n$ be composite.
  If there is a Fermat witness $w$ for $n$ such that $\gcd(w, n) = 1$, then
  \[
  |\{a \in [1, n - 1] \mid a^{n-1} \not\equiv 1 \pmod{n} \}| > \frac{n - 1}{2}
  \]
\end{prop}
\proof
TODO
\qed


In the above proposition, the $a$ is an integer in $[1, n - 1]$.
Remember that we usually test $a \in [2, n-2]$.
Also, we usually assume $n > 3$ is odd.
Note that $1^{n-1} \equiv 1 \pmod{n}$.
Hence the above proposition implies
\[
|\{a \in [2, n - 1] \mid a^{n-1} \not\equiv 1 \pmod{n} \}| > \frac{n - 1}{2}
\]
If $n$ is odd.
Then $n - 1$ is even and $(n-1)^{n-1} \equiv (-1)^{n-1} \equiv 1 \pmod{n}$.
Hence
\[
|\{a \in [2, n - 2] \mid a^{n-1} \not\equiv 1 \pmod{n} \}| > \frac{n - 1}{2}
\]
Also, $(n - 1)/2$ is an integer.
Therefore
\[
|\{a \in [2, n - 2] \mid a^{n-1} \not\equiv 1 \pmod{n} \}| \geq \frac{n - 1}{2}
+ 1 = \frac{n + 1}{2} 
\]
The number of integer in $[2, n - 2]$ is $n - 2 - 1 = n - 3$.
Therefore $>50\%$ of the values in $[2, n - 2]$ are Fermat witnesses.
If $n$ is even.
Then $n - 1$ is odd and $(n-1)^{n-1} \equiv (-1)^{n-1} \equiv -1 \pmod{n}$.
\[
|\{a \in [2, n - 2] \mid a^{n-1} \not\equiv 1 \pmod{n} \}| + 1 > \frac{n - 1}{2}
\]
In this case $(n - 1)/2$ is not an integer.
\[
|\{a \in [2, n - 2] \mid a^{n-1} \not\equiv 1 \pmod{n} \}| + 1
\geq \frac{n - 1}{2} + \frac{1}{2}
\]
i.e.,
\[
|\{a \in [2, n - 2] \mid a^{n-1} \not\equiv 1 \pmod{n} \}|
\geq \frac{n - 1}{2} - \frac{1}{2}
= \frac{n - 2}{2}
\]
Together,
\[
|\{a \in [2, n - 2] \mid a^{n-1} \not\equiv 1 \pmod{n} \}| >
\begin{cases}
  \displaystyle \frac{n + 1}{2} & \text{ if $n$ is odd} \\
  \displaystyle \frac{n - 2}{2} & \text{ if $n$ is even}
\end{cases}
\]
In both of the above cases
\[
|\{a \in [2, n - 2] \mid a^{n-1} \not\equiv 1 \pmod{n} \}| > \frac{n - 3}{2}
\]
i.e., more than 1/2 of the values in $[2, n - 2]$ are witnesses.

Suppose $p$ is the probability of not finding a Fermat witness
$a$ in $[2, n-2]$ for $n$.
Assuming there is a Fermat witness $a$ with $\gcd(a, n) = 1$.
From the above, more than half of the $a$ in $[2, n - 2]$ are Fermat
witnesses for $n$.
This means that $p < 0.5$.
The probability of not finding a Fermat witness after $t$ tries is
\[
p^{t}
\]
Therefore the probability of finding a Fermat witness is
\[
1 - p^t
\]
For instance assuming $p = 0.5$, then $t=8$ tries,
if no Fermat witness is found, the probability that $n$ is prime is
\[
1 - p^t > 1 - 0.5^8 = 0.99609375
\]
If $t = 10$, we have
\[
1 - p^t > 1 - 0.5^{10} = 0.9990234375
\]
And if $t = 20$, we reach
\[
1 - p^t > 1 - 0.5^{10} = 0.9999990463256836
\]
Of course all the above is based on the assumption that there is
a Fermat witness $a$ such that $\gcd(a, n) = 1$.
You are out of luck if $n$ is a Carmichael number.
You can think of Carmichael numbers as extreme failure cases of Fermat
primality test.

The question is how common are Carmichael numbers?
Carmichael numbers are very rare.
At this point, we know that the density of Carmichael numbers
is about 1 in 50 trillion = $5 \times 10^{13}$.

\begin{eg}
In the following, we obtain a random integer with 10 digits
and found a Fermat witness with one try:
\begin{Verbatim}[fontsize=\footnotesize,frame=single]
import random; random.seed()
d = 10
n = random.randrange(10**(d - 1), 10**d)
print(n)
for i in range(20):
    a = random.randrange(2, n - 1)
    b = (pow(a, n - 1, n) != 1)
    print(i, a, b)
    if b: break
\end{Verbatim}
The output is
\begin{Verbatim}[fontsize=\footnotesize,frame=single]
1151731626
0 869958907 True
\end{Verbatim}
Fermat primality test says that 1151731626 is composite with only one try.
In fact $1151731626 = 2 \cdot 3 \cdot 19 \cdot 10102909$.
\end{eg}

\begin{eg}
Here's another run of the above code where no Fermat witness was found
after 20 tries:
\begin{Verbatim}[fontsize=\footnotesize,frame=single,commandchars=\\\{\}]
\userinput{3584990077}
0 3295070400 False
1 3215421426 False
2 262972142 False
3 1050903352 False
4 152804132 False
5 1015451650 False
6 885960417 False
7 720561088 False
8 1694137561 False
9 2065337998 False
10 3345601994 False
11 1535607663 False
12 3183174792 False
13 1772850385 False
14 2569697199 False
15 948739551 False
16 2148646472 False
17 1640965445 False
18 2024258764 False
19 3229113680 False
\end{Verbatim}
No Fermat witness was found after 20 tries.
Fermat primality test would return \lq\lq probably a prime".
In fact $3584990077$ is prime.
In this case $3584990077$ is small enough that a brute force
prime testing by brute force division can be used.
\end{eg}


\begin{ex} 
  \label{ex:hill-10}
  \tinysidebar{\debug{exercises/{rsa-43/question.tex}}}
  What if you test for the condition $n = x^3 - y^3$?
  Is there are version of
  factorization using difference of cubes?
  %We have
  %\[
  %  x^3 - y^3 = (x - y)(x^2 + xy + y^2)
  %\]
  %Suppose $n = abc$.
  %\[
  %  \frac{}{}
  %\]

  \solutionlink{sol:hill-10}
  \qed
\end{ex} 
\begin{python0}
from solutions import *
add(label="ex:hill-10",
    srcfilename='exercises/hill-10/answer.tex') 
\end{python0}


\begin{ex} 
  \label{ex:hill-10}
  \tinysidebar{\debug{exercises/{rsa-43/question.tex}}}
  What if you test for the condition $n = x^3 - y^3$?
  Is there are version of
  factorization using difference of cubes?
  %We have
  %\[
  %  x^3 - y^3 = (x - y)(x^2 + xy + y^2)
  %\]
  %Suppose $n = abc$.
  %\[
  %  \frac{}{}
  %\]

  \solutionlink{sol:hill-10}
  \qed
\end{ex} 
\begin{python0}
from solutions import *
add(label="ex:hill-10",
    srcfilename='exercises/hill-10/answer.tex') 
\end{python0}


\begin{ex} 
  \label{ex:hill-10}
  \tinysidebar{\debug{exercises/{rsa-43/question.tex}}}
  What if you test for the condition $n = x^3 - y^3$?
  Is there are version of
  factorization using difference of cubes?
  %We have
  %\[
  %  x^3 - y^3 = (x - y)(x^2 + xy + y^2)
  %\]
  %Suppose $n = abc$.
  %\[
  %  \frac{}{}
  %\]

  \solutionlink{sol:hill-10}
  \qed
\end{ex} 
\begin{python0}
from solutions import *
add(label="ex:hill-10",
    srcfilename='exercises/hill-10/answer.tex') 
\end{python0}


The following is a major theorem on Carmichael numbers:

\begin{thm} \textnormal{(\defone{Koselt's Theorem} 1899)}
  Let $n$ be a positive integer.
  Then $n$ is a Carmichael number iff $n$ is odd, squarefree, and
  if $p \mid n$, then $p-1 \mid n-1$.
\end{thm}
\proof
Omitted.
\qed

The conditions on $n$ in the above theorem is frequently called
\defone{Koselt's criterion}.

\begin{eg}
Consider 561.
From the above second table, we know that 561 is a Carmichael.
Also, as mentioned earlier \v{S}imerka was the first to discover this
Carmichael number.
561 is small and can be easily factorized: $561 = 3 \cdot 11 \cdot 17$.
Note that 561 is odd and squarefree.
Furthermore $2 \mid 560$, $10 \mid 560$, and $16 \mid 560$.
Therefore $561$ satisfies Koselt's criteria and therefore
$561$ is a Carmichael number.
\end{eg}


\begin{ex} 
  \label{ex:hill-10}
  \tinysidebar{\debug{exercises/{rsa-43/question.tex}}}
  What if you test for the condition $n = x^3 - y^3$?
  Is there are version of
  factorization using difference of cubes?
  %We have
  %\[
  %  x^3 - y^3 = (x - y)(x^2 + xy + y^2)
  %\]
  %Suppose $n = abc$.
  %\[
  %  \frac{}{}
  %\]

  \solutionlink{sol:hill-10}
  \qed
\end{ex} 
\begin{python0}
from solutions import *
add(label="ex:hill-10",
    srcfilename='exercises/hill-10/answer.tex') 
\end{python0}


Although Koselt proved his theorem in 1899,
he never mention any Carmichael number in his publications.

\begin{cor} \textnormal{(Cernick 1939)}\label{cor:cernick}
  If $6k + 1$, $12k + 1$, and $18k + 1$ are all primes, then the product
  $(6k + 1)(12k + 1)(18k + 1)$ is a Carmichael number.
\end{cor}
\proof
Let $n = (6k + 1)(12k + 1)(18k + 1) = 6 \cdot 12 \cdot 18 \cdot k^3
+ (6 \cdot 12 + 6 \cdot 18 + 12 \cdot 18)k^2
+ (6 + 12 + 18)k
+ 1$.
$n$ is clearly odd and since $n$ is a product of three distinct primes,
$4n$ is clearly squarefree.
Note that
$n - 1 = (6k + 1)(12k + 1)(18k + 1) = 6 \cdot 12 \cdot 18 \cdot k^3
+ (6 \cdot 12 + 6 \cdot 18 + 12 \cdot 18)k^2
+ (6 + 12 + 18)k$.
Clearly
$6k \mid n - 1$,
$12k \mid n - 1$,
$18k \mid n - 1$.
Therefore by Korselt's Theorem, $n$ is a Carmichael number.
\qed


\begin{ex} 
  \label{ex:hill-10}
  \tinysidebar{\debug{exercises/{rsa-43/question.tex}}}
  What if you test for the condition $n = x^3 - y^3$?
  Is there are version of
  factorization using difference of cubes?
  %We have
  %\[
  %  x^3 - y^3 = (x - y)(x^2 + xy + y^2)
  %\]
  %Suppose $n = abc$.
  %\[
  %  \frac{}{}
  %\]

  \solutionlink{sol:hill-10}
  \qed
\end{ex} 
\begin{python0}
from solutions import *
add(label="ex:hill-10",
    srcfilename='exercises/hill-10/answer.tex') 
\end{python0}


\begin{ex} 
  \label{ex:hill-10}
  \tinysidebar{\debug{exercises/{rsa-43/question.tex}}}
  What if you test for the condition $n = x^3 - y^3$?
  Is there are version of
  factorization using difference of cubes?
  %We have
  %\[
  %  x^3 - y^3 = (x - y)(x^2 + xy + y^2)
  %\]
  %Suppose $n = abc$.
  %\[
  %  \frac{}{}
  %\]

  \solutionlink{sol:hill-10}
  \qed
\end{ex} 
\begin{python0}
from solutions import *
add(label="ex:hill-10",
    srcfilename='exercises/hill-10/answer.tex') 
\end{python0}


\begin{ex} 
  \label{ex:hill-10}
  \tinysidebar{\debug{exercises/{rsa-43/question.tex}}}
  What if you test for the condition $n = x^3 - y^3$?
  Is there are version of
  factorization using difference of cubes?
  %We have
  %\[
  %  x^3 - y^3 = (x - y)(x^2 + xy + y^2)
  %\]
  %Suppose $n = abc$.
  %\[
  %  \frac{}{}
  %\]

  \solutionlink{sol:hill-10}
  \qed
\end{ex} 
\begin{python0}
from solutions import *
add(label="ex:hill-10",
    srcfilename='exercises/hill-10/answer.tex') 
\end{python0}

