\sectionthree{Fermat factorization}
\begin{python0}
from solutions import *; clear()
\end{python0}

Suppose you want to factorize $n$ and you know that
$n$ is a difference of two squares:
\[
n = x^2 - y^2
\]
then right away you know
\[
n = (x + y)(x - y)
\]
Of course if $x - y$ is 1, the factorization is not
helpful at all.

To achieve the goal of writing $n$ as a difference of squares,
you can do the following:
\begin{tightlist}
  \li Check if $1^2 - n$ is a square.
  \li Check if $2^2 - n$ is a square.
  \li Check if $3^2 - n$ is a square.
  \li Etc.
\end{tightlist}
Why?
Because if $x^2 - n$ is a square, say $y^2$, then
$x^2 - n = y^2$ gives us
\[
n = x^2 - y^2
\]

Note that if $n = ab$ is odd (which is the case for an RSA modulus),
then writing $n$ as a difference of two squares since
\[
  n
  =
  \left(\frac{a + b}{2}\right)^2
  -
  \left(\frac{a - b}{2}\right)^2
\]
If $n$ is odd, then $a$ and $b$ are odd and
therefore $a + b$ and $a - b$ are both even.
Hence every odd $n$ can be expressed as the
difference of two squares.
What if $n$ is even?
Then you can remove $2^k$ from $n$
where $k$ is maximal to get
$n = 2^k n'$ and proceed with the above
idea applied to $n'$ if $n' > 1$.
In the following, I will assume $n$ is odd.

Now we do not want the factorization $n = n \times 1$.
Since we are aiming for $n = (x-y)(x+y)$,
one way is to start $x$ at around $\sqrt{n}$.
Then $y$ will be close to $0$ and therefore $x+y$ and $x-y$ won't be close to 1.

Another thing to note is that if we start $x$ at $\sqrt{n}$,
if $n$ is squarefree (which is the case for RSA),
we will have $x = \floor{\sqrt{n}} < \sqrt{n}$,
which means $x^2 - n$ is negative and therefore cannot be $y^2$.
In this case, we might as well start with
$x = \floor{\sqrt{n}} + 1$.

To find $x,y$ such that $n = x^2 - y^2$,
we try different values of $x$ and then check if
\[
  x^2 - n
\]
is a square:
\[
  x^2 - n = y^2
\]
We can started with $x = \floor{\sqrt{n}}$ (or around there) and increment $x$ by 1 until
the square is found.
Note that if $n$ is itself a square (this won't happen for the RSA case),
then the above will end with $y = 0$.

\begin{console}[fontsize=\footnotesize]
ALGORITHM: Fermat-Factorization
INPUT: n -- an odd integer

let x = floor(sqrt(n)) + 1
while x * x - n is not a square:
    x = x + 1

y = sqrt(x * x - n)
return (x - y, x + y)    
\end{console}

Here's an example execution of my code
for $n = 11 \cdot 11 \cdot 37$:
\VerbatimInput[frame=single,fontsize=\footnotesize]{fermat_factorization_example.tex}

Note that Fermat factorization need not be faster than trial division,
but it is specifically crafted to handle the product of two primes which are
\lq\lq close".

I'll give you one optimization:


\begin{ex} 
  \label{ex:rsa-10}
  \tinysidebar{\debug{exercises/{nt-61/question.tex}}}
  In your \verb!Zmod.py!, complete the following methods:
  \begin{enumerate}[nosep]
    \li multiplicative inverse mod $N$ (i.e., \texttt{inv})
    \li invertibility mod $N$ (i.e., \texttt{is\_invertible})
    \li division (i.e., \texttt{\_\_div\_\_})
  \end{enumerate}

  \solutionlink{sol:rsa-10}
  \qed
\end{ex} 
\begin{python0}
from solutions import *
add(label="ex:rsa-10",
    srcfilename='exercises/rsa-10/answer.tex') 
\end{python0}


\begin{ex} 
  \label{ex:rsa-10}
  \tinysidebar{\debug{exercises/{nt-61/question.tex}}}
  In your \verb!Zmod.py!, complete the following methods:
  \begin{enumerate}[nosep]
    \li multiplicative inverse mod $N$ (i.e., \texttt{inv})
    \li invertibility mod $N$ (i.e., \texttt{is\_invertible})
    \li division (i.e., \texttt{\_\_div\_\_})
  \end{enumerate}

  \solutionlink{sol:rsa-10}
  \qed
\end{ex} 
\begin{python0}
from solutions import *
add(label="ex:rsa-10",
    srcfilename='exercises/rsa-10/answer.tex') 
\end{python0}


\begin{ex} 
  \label{ex:rsa-10}
  \tinysidebar{\debug{exercises/{nt-61/question.tex}}}
  In your \verb!Zmod.py!, complete the following methods:
  \begin{enumerate}[nosep]
    \li multiplicative inverse mod $N$ (i.e., \texttt{inv})
    \li invertibility mod $N$ (i.e., \texttt{is\_invertible})
    \li division (i.e., \texttt{\_\_div\_\_})
  \end{enumerate}

  \solutionlink{sol:rsa-10}
  \qed
\end{ex} 
\begin{python0}
from solutions import *
add(label="ex:rsa-10",
    srcfilename='exercises/rsa-10/answer.tex') 
\end{python0}


\begin{ex} 
  \label{ex:rsa-10}
  \tinysidebar{\debug{exercises/{nt-61/question.tex}}}
  In your \verb!Zmod.py!, complete the following methods:
  \begin{enumerate}[nosep]
    \li multiplicative inverse mod $N$ (i.e., \texttt{inv})
    \li invertibility mod $N$ (i.e., \texttt{is\_invertible})
    \li division (i.e., \texttt{\_\_div\_\_})
  \end{enumerate}

  \solutionlink{sol:rsa-10}
  \qed
\end{ex} 
\begin{python0}
from solutions import *
add(label="ex:rsa-10",
    srcfilename='exercises/rsa-10/answer.tex') 
\end{python0}


\begin{ex} 
  \label{ex:rsa-10}
  \tinysidebar{\debug{exercises/{nt-61/question.tex}}}
  In your \verb!Zmod.py!, complete the following methods:
  \begin{enumerate}[nosep]
    \li multiplicative inverse mod $N$ (i.e., \texttt{inv})
    \li invertibility mod $N$ (i.e., \texttt{is\_invertible})
    \li division (i.e., \texttt{\_\_div\_\_})
  \end{enumerate}

  \solutionlink{sol:rsa-10}
  \qed
\end{ex} 
\begin{python0}
from solutions import *
add(label="ex:rsa-10",
    srcfilename='exercises/rsa-10/answer.tex') 
\end{python0}


\begin{ex} 
  \label{ex:rsa-10}
  \tinysidebar{\debug{exercises/{nt-61/question.tex}}}
  In your \verb!Zmod.py!, complete the following methods:
  \begin{enumerate}[nosep]
    \li multiplicative inverse mod $N$ (i.e., \texttt{inv})
    \li invertibility mod $N$ (i.e., \texttt{is\_invertible})
    \li division (i.e., \texttt{\_\_div\_\_})
  \end{enumerate}

  \solutionlink{sol:rsa-10}
  \qed
\end{ex} 
\begin{python0}
from solutions import *
add(label="ex:rsa-10",
    srcfilename='exercises/rsa-10/answer.tex') 
\end{python0}


\begin{ex} 
  \label{ex:rsa-10}
  \tinysidebar{\debug{exercises/{nt-61/question.tex}}}
  In your \verb!Zmod.py!, complete the following methods:
  \begin{enumerate}[nosep]
    \li multiplicative inverse mod $N$ (i.e., \texttt{inv})
    \li invertibility mod $N$ (i.e., \texttt{is\_invertible})
    \li division (i.e., \texttt{\_\_div\_\_})
  \end{enumerate}

  \solutionlink{sol:rsa-10}
  \qed
\end{ex} 
\begin{python0}
from solutions import *
add(label="ex:rsa-10",
    srcfilename='exercises/rsa-10/answer.tex') 
\end{python0}


\begin{comment}
  Square check:

  n^2 \equiv 0,1 (4)               --> 50%                          00                   01
  n^2 \equiv 0,1,4 (8)             --> 3/8 = 37.5%               000   100              001
  n^2 \equiv 0,1,4,9 (16)          --> 4/16 = 25%               0000  0100         0001     1001
  n^2 \equiv 0,1,4,9,16,17,25 (32) --> 7/32 = 21.8%            00000 00100        *0001    *1001

  Therefore is 1st 5 bits not 00000 or 00100 or *0001 or *1001, n is not square
  
  x^2 - n ... find first z^2     ---> z^2 <= x^2 - n
  (x+1)^2-n = x^2 - n + 2*x + 1, (z+1)^2 = z^2 + 2z + 1 ---> (z+1)^2 <= (x+1)^2 - n

  x^2 - n increase by 2x + 1
  z^2     increase by 2z + 1
  DIFF = 2(z - x)
  
  DELTA = (x^2 - n) - z^2.


  x^2 -> (x+1)^2 = x^2 + 2x + 1
  For k steps:
  (x+k)^2 = x^2 + 2kx + 1
  Find best k.
  
\end{comment}


\begin{ex} 
  \label{ex:rsa-10}
  \tinysidebar{\debug{exercises/{nt-61/question.tex}}}
  In your \verb!Zmod.py!, complete the following methods:
  \begin{enumerate}[nosep]
    \li multiplicative inverse mod $N$ (i.e., \texttt{inv})
    \li invertibility mod $N$ (i.e., \texttt{is\_invertible})
    \li division (i.e., \texttt{\_\_div\_\_})
  \end{enumerate}

  \solutionlink{sol:rsa-10}
  \qed
\end{ex} 
\begin{python0}
from solutions import *
add(label="ex:rsa-10",
    srcfilename='exercises/rsa-10/answer.tex') 
\end{python0}


\begin{comment}
  It would terminate when $x = (p + 1)/2$ and $y = (p - 1)/2$
  so that $x - y = 1$, $x + y = p$, i.e., we get the factorization $n = 1 \times n$.

  In this case $x$ starts at around $\sqrt{p} + 1$ and reaches $(p + 1)/2$,
  roughly from $\sqrt{p}$ to $p/2$.
  For instance if $p$ is about $10000$, then $p$ runs from $100$ to $5000$, i.e.,
  $4900$ integers.
  Brute force test for primeness runs from $2$ to $\sqrt{p}$ which in this case if $2$ to $100$.
  Therefore Fermat Factorization would be a very bad prime testing algorithm.
\end{comment}





% http://www.ams.org/journals/mcom/1999-68-228/S0025-5718-99-01133-3/home.html


\begin{ex} 
  \label{ex:rsa-10}
  \tinysidebar{\debug{exercises/{nt-61/question.tex}}}
  In your \verb!Zmod.py!, complete the following methods:
  \begin{enumerate}[nosep]
    \li multiplicative inverse mod $N$ (i.e., \texttt{inv})
    \li invertibility mod $N$ (i.e., \texttt{is\_invertible})
    \li division (i.e., \texttt{\_\_div\_\_})
  \end{enumerate}

  \solutionlink{sol:rsa-10}
  \qed
\end{ex} 
\begin{python0}
from solutions import *
add(label="ex:rsa-10",
    srcfilename='exercises/rsa-10/answer.tex') 
\end{python0}


\begin{ex} 
  \label{ex:rsa-10}
  \tinysidebar{\debug{exercises/{nt-61/question.tex}}}
  In your \verb!Zmod.py!, complete the following methods:
  \begin{enumerate}[nosep]
    \li multiplicative inverse mod $N$ (i.e., \texttt{inv})
    \li invertibility mod $N$ (i.e., \texttt{is\_invertible})
    \li division (i.e., \texttt{\_\_div\_\_})
  \end{enumerate}

  \solutionlink{sol:rsa-10}
  \qed
\end{ex} 
\begin{python0}
from solutions import *
add(label="ex:rsa-10",
    srcfilename='exercises/rsa-10/answer.tex') 
\end{python0}

