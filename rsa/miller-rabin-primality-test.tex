\sectionthree{Miller-Rabin primality test}
\begin{python0}
from solutions import *; clear()
\end{python0}

Fermat Little Theorem says this:
Let $a \in [2, n - 2]$.
\[
\text{$n$ is prime} \implies a^{n-1} \equiv 1 \pmod{n}
\]
The basis of Fermat primality test is
\[
\text{$n$ is not prime} \impliedby a^{n-1} \not\equiv 1 \pmod{n}
\]
We can a say bit more.
Again we have
\[
\text{$n$ is prime} \implies a^{n-1} \equiv 1 \pmod{n}
\]
Suppose $n - 1 = 2^k m$ with $k \geq 0$ and $2 \nmid m$.
Then we have
\[
\text{$n$ is prime} \implies a^{2^k m} \equiv 1 \pmod{n}
\]
Now note that
\[
a^{2^k m} = (a^m)^{2^k}
\]
This can be computed as a sequence of squares.
For instance if $k = 3$, then
\[
a^{2^3 m} = (a^m)^{2^3} = (((a^m)^2)^2)^2 
\]
Note that following fact:

\begin{prop}
  If $p$ is a prime and $x^2 \equiv 1 \pmod{p}$ then $x \equiv \pm 1 \pmod{p}$.
\end{prop}
\proof
TODO
\qed

Applying this proposition to $x^2 = (a^m)^{2^k}$, we have the following fact:
if $n$ is prime, and $k > 0$, then
\[
(a^m)^{2^k} \equiv 1 \pmod{n} \implies (a^m)^{2^{k - 1}} \equiv \pm 1 \pmod{n}
\]
And if
\[
(a^m)^{2^{k - 1}} \equiv 1 \pmod{n}
\]
then
\[
(a^m)^{2^{k - 2}} \equiv \pm 1 \pmod{n}
\]
Etc.

All in all, assuming $n$ is prime,
writing $n - 1$ as $2^k m$ where $2^k$ is the highest
power of $2$ dividing $n - 1$, then the sequence
\begin{align*}
  & (a^m)^{2^0} \pmod{n} \\
  & (a^m)^{2^1} \pmod{n} \\
  & (a^m)^{2^2} \pmod{n} \\
  & \vdots \\
  & (a^m)^{2^{k-2}} \pmod{n} \\
  & (a^m)^{2^{k-1}} \pmod{n} \\
  & (a^m)^{2^k} \pmod{n}
\end{align*}
either the whole sequence is 1s or it
ends with a sequence of 1s (of length $\geq 1$)
and before this sequence there is a $-1$.
For instance if $k = 5$, the above sequence is a sequence of 6 numbers
and here are some possibilities:
\begin{enumerate}[nosep]
\li The last three numbers might be $-1, 1, 1 \pmod{n}$ (the first three are not -1 and not 1).
\li Or the last two might be $-1, 1 \pmod{n}$ (the first four are not -1 and not 1).
\li Or all 6 might be $1, 1, 1, 1, 1, 1 \pmod{n}$.
\end{enumerate}
Of course if $k = 0$, then there is only one number in the sequence and that number is $1 \pmod{n}$.
In general, the above is a sequence of $k + 1$ numbers
and ends with $1$s of length $\geq 1$ or ends with $-1$ followed by $1$s of length $\geq 1$.

Miller-Rabin primality test is similar to Fermat prime test.
For an integer $n$, we compute $m$ and $k$ such that $n - 1 = 2^k \cdot m$.
We randomly pick an $a$ in $[2, n-2]$
look at the values
\begin{align*}
  &a^m \pmod{n} \\
  &a^{2m} \pmod{n} \\
  &a^{2^2m} \pmod{n} \\
  &a^{2^3m} \pmod{n} \\
  &\vdots \\
  &a^{2^{k-2} m} \pmod{n} \\
  &a^{2^{k-1} m} \pmod{n} \\
  &a^{2^k m} \pmod{n}
\end{align*}
and if they are all 1s (i.e., the first is 1)
or ends with $-1,1,1,...1$, the algorithm (i.e., there's a $-1$)
returns \verb!"n is probably prime"!.
Otherwise it returns \verb!"n is composite"!.
The difference is Fermat primality test only looks at the last value.

\begin{eg}
As an example, note that $n = 561$ is a Carmichael number.
Fermat primality test reports $n$ as probably prime even though it is
a composite.
Using Miller-Rabin primality test, first $n - 1 = 560 = 35 \cdot 2^4$.
Let us use $a = 2$.
\begin{align*}
  2^{35}        &\equiv 263 \pmod{561} \\
  (2^{35})^2    &\equiv (263)^2 \equiv 166 \pmod{561} \\
  (2^{35})^{2^2} &\equiv (166)^2 \equiv 67 \pmod{561} \\
  (2^{35})^{2^3} &\equiv (67)^2 \equiv 1 \pmod{561} \\
  (2^{35})^{2^4} &\equiv (1)^2 \equiv 1 \pmod{561}
\end{align*}
The sequence is $263, 166, 67, 1, 1$.
And we see that $n$ cannot be prime, because if $n$ is prime, from line 3 above
\[
(2^{35})^{2^3} \equiv 1 \pmod{561}
\]
the previous line should have been $\pm 1 \pmod{561}$, but it is not.
Therefore Miller-Rabin primality test will report $561$ as composite.
\end{eg}

One can define
Miller-Rabin pseudoprimes (usually called \defone{strong pseudoprimes}),
Miller-Rabin witness,
and
Miller-Rabin liar.
While the failure case of Fermat primality test is on the average
less than 1/2, the
failure case of Miller-Rabin is less than 1/4.

Here's the Miller-Rabin primality test algorithm:

\newcommand\mathmode[1]{$ #1 $}
\begin{smallconsole}[commandchars=\\\{\}]
ALGORITHM: Miller-Rabin-primality-test
INPUTS: n -- integer to be tested for primeness/compositeness
        t -- number of passes

compute k and odd m such that n - 1 = 2^k * m
for pass = 1, 2, 3, ..., t:
    randomly select a in [2, n - 2]
    if Miller-Rabin-one-pass(n, k, m, a) returns "n is composite":
        return "n is a composite"
return "n is probably a prime"


ALGORITHM: Mill-Rabin-one-pass
INPUTS: n, k, m where n - 1 = 2^k * m
        a
OUTPUT: "n is composite" or "n is probably prime"
       
let b \mathmode{\equiv} a^m (mod n)
if b \mathmode{\equiv} 1 (mod n):
    return "n is probably prime"
for i = 0, 1, 2, ..., k - 1:
    if b \mathmode{\equiv} -1 (mod n):
        return "n is probably prime"
    b \mathmode{\equiv} b^2 (mod n)
return "n is composite"
\end{smallconsole}


\begin{ex} 
  \label{ex:rsa-10}
  \tinysidebar{\debug{exercises/{nt-61/question.tex}}}
  In your \verb!Zmod.py!, complete the following methods:
  \begin{enumerate}[nosep]
    \li multiplicative inverse mod $N$ (i.e., \texttt{inv})
    \li invertibility mod $N$ (i.e., \texttt{is\_invertible})
    \li division (i.e., \texttt{\_\_div\_\_})
  \end{enumerate}

  \solutionlink{sol:rsa-10}
  \qed
\end{ex} 
\begin{python0}
from solutions import *
add(label="ex:rsa-10",
    srcfilename='exercises/rsa-10/answer.tex') 
\end{python0}


\begin{ex} 
  \label{ex:rsa-10}
  \tinysidebar{\debug{exercises/{nt-61/question.tex}}}
  In your \verb!Zmod.py!, complete the following methods:
  \begin{enumerate}[nosep]
    \li multiplicative inverse mod $N$ (i.e., \texttt{inv})
    \li invertibility mod $N$ (i.e., \texttt{is\_invertible})
    \li division (i.e., \texttt{\_\_div\_\_})
  \end{enumerate}

  \solutionlink{sol:rsa-10}
  \qed
\end{ex} 
\begin{python0}
from solutions import *
add(label="ex:rsa-10",
    srcfilename='exercises/rsa-10/answer.tex') 
\end{python0}


\begin{ex} 
  \label{ex:rsa-10}
  \tinysidebar{\debug{exercises/{nt-61/question.tex}}}
  In your \verb!Zmod.py!, complete the following methods:
  \begin{enumerate}[nosep]
    \li multiplicative inverse mod $N$ (i.e., \texttt{inv})
    \li invertibility mod $N$ (i.e., \texttt{is\_invertible})
    \li division (i.e., \texttt{\_\_div\_\_})
  \end{enumerate}

  \solutionlink{sol:rsa-10}
  \qed
\end{ex} 
\begin{python0}
from solutions import *
add(label="ex:rsa-10",
    srcfilename='exercises/rsa-10/answer.tex') 
\end{python0}




