\sectionthree{Multiplication: Karatsuba algorithm}
\begin{python0}
from solutions import *; clear()
\end{python0}

Skip this section if you have already taken CISS358.

It turns out that there is a better way to multiply integers.
This was only discovered very recently in the 1960s. 

The idea is surprisingly simple.
Here's a basic formula:
\[
(aT + b)(cT + d) = acT^2 + (ad + bc)T + bd
\]
For now you can pretend that $T = 10$ and $a,b,c,d$ are digits.
For instance to multiply $23$ and $45$, you can view it as
\[
(2 \cdot 10 + 3)(4 \cdot 10 + 5)
= (2 \cdot 4) 10^2 + (2 \cdot 5 + 3 \cdot 4) 10 + 3 \cdot 5
\]
You can finish up this computation on your own.

But 
\[
(aT + b)(cT + d) = acT^2 + (ad + bc)T + bd
\]
doesn't really help!!!
Viewing $T = 10$ and the $a,b,c,d$ as digits, the above
multiplication of two 2-digit numbers, i.e. 
$(aT + b)$ and $(cT + d)$, we need
\textit{ four} multiplications
\[
ac, \,\,\, 
ad, \,\,\,
bc, \,\,\,
bd
\]
So it's still essentially an $n^2$ algorithm.
In general if $T(n)$ is runtime of multiplying 2 integers of length $n$,
then
\[
T(n) = 4T(n/2) + An + B
\]
If $n = 2^k$, then
\begin{align*}
T(2^k) &= 4T(2^{k-1}) + A2^k + B \\
&= 4(4T(2^{k-2}) + A2^{k-1} + B) + A2^k + B \\
&= 4^2T(2^{k-2}) + (4 \cdot 2^{k-1} + 2^k)A + (4 + 1) B \\
&= 4^2(4T(2^{k-3}) + A2^{k-2} + B) + (4 \cdot 2^{k-1} + 2^k)A + (4 + 1)B \\
&= 4^3T(2^{k-3}) + (4^2 \cdot 2^{k-2} + 4 \cdot 2^{k-1} + 2^k)A + (4^2 + 4 + 1)B \\
&= ... \\
&= 4^kT(2^{k-k}) + (4^{k-1} \cdot 2^{1} + ... + 4^2 \cdot 2^{k-2} + 4 \cdot 2^{k-1} + 2^k)A + (4^{k-1} + ... + 4^2 + 4 + 1)B
\end{align*}
The coefficient for $B$ is $\displaystyle \frac{4^k - 1}{4 - 1}$.
The coefficient for $A$ is
\[
4^{k-1} \cdot 2^{1} + ... + 4^2 \cdot 2^{k-2} + 4 \cdot 2^{k-1} + 2^k
=
2^{2k - 1}
+ 2^{k + 2}
+ 2^{k + 1}
+ 2^k
= 2^k(1 + 2 + ... + 2^{k-1})
= 2^k(2^k - 1)
\]
Hence
\[
T(2^k) = 4^k T(1) + 2^k(2^k - 1) A + \frac{4^k - 1}{3} B
\]
Hence $T(1) = C$:
\[
T(2^k) = 4^k C + 2^k(2^k - 1) A + \frac{4^k - 1}{3} B
\]
Since $n = 2^k$,
\[
T(n) = n^2 C + n(n - 1) A + \frac{n^2 - 1}{3} B
\]
Clearly $T(n) = O(n^2)$.

But wait ... here's the brilliant (but simple) idea from Karatsuba.
Note that 
\begin{align*}
            (a+b)(c+d) &= ac + ad + bc + bd \\
\THEREFORE\,\,\,  (a+b)(c+d) - ac - bd &= ad + bc
\end{align*}
In other words
\[
(aT + b)(cT + d) = (ac)T^2 + [(a+b)(c+d) - (ac) - (bd)]T + (bd)
\]
If you count all the operations on the right,
you would see that there are only
\textit{ three} multiplications!!!

So we can do this:
\begin{Verbatim}[frame=single]
1. Compute A = ac         ... 1 multiplication
2. Compute B = bd         ... 1 multiplication
3. Compute C = (a+b)(c+d) ... 2 additions, 1 multiplication
4. Compute D = C - A - B  ... 2 subtractions
5. Output AT^2 + DT + B
\end{Verbatim}

Note that the runtime for subtraction is like addition.
In other words addition and subtraction are \lq\lq cheap'':
they are both $O(n)$ where $n$ is the length of the integers.

Practically speaking, how is Karatsuba actually used? 
Suppose you have to multiply two 8-digit numbers:
\[
12345678 \times 13572468
\]
We split them up into this (with $T = 10$)
\[
(1234T^4 + 5678) \times (1357T^4 + 2468)
\]
by Karatsuba
\begin{align*}
(a T^4 + b) \times (c T^4 + d) &= A (T^4)^2 + D (T^4) + B \\
A &= a b \\
B &= b d \\
C &= (a + b)(c + d) \\
D &= C - A - B
\end{align*}
Next, we
\textit{ again} apply Karasuba but this time to the products
$a b, b d, (a + b)(c + d)$.
Etc. 
In other words you recursively use Karatsuba
until the products involve numbers which are small enough
that we can perform them quickly without Karatsuba,
If we starting with the multiplication of two 8-digit numbers.
We're left with three multiplications of 4-digit numbers.
On applying Karasuba to the three multiplications of 4-digits numbers,
each multiplication gives rise to 3 2-digit numbers.
There are now $3 \times 3$ multiplications of 2-digit
numbers.
Going further, we get $3 \times 3 \times 3$ multiplication of 1-digit
numbers.

In general you see the following.
Suppose we start off with multiplying two n-digit numbers
and $n = 2^m$.
We have:
\begin{align*}
&1 \text{ multiplication(s) of } 2^m\text{--digit numbers} \\
&3 \text{ multiplication(s) of } 2^{m-1}\text{--digit numbers} \\
&3^2 \text{ multiplication(s) of } 2^{m-2}\text{--digit numbers} \\
&3^3 \text{ multiplication(s) of } 2^{m-3}\text{--digit numbers} \\
&3^4 \text{ multiplication(s) of } 2^{m-4}\text{--digit numbers} \\
&\ldots \\
&3^m \text{ multiplication(s) of } 2^{m-m} = 1\text{--digit numbers}
\end{align*}
Note that we started with $n$-digit numbers, i.e., $n = 2^m$, 
i.e., $m = \log n$
($\log$ means $\log_2$, right?)
This gives rise to $3^m = 3^{\log n}$ multiplications, i.e.
\[
3^{\log n} 
= 3^\frac{\log_3 n}{\log_3 2} 
= \left(
3^{\log_3 n}
\right)^\frac{1}{\log_3 2}
= n^\frac{1}{\log_3 2}
= n^\frac{1}{\log 2/\log 3}
= n^{\log_2 3}
= n^{1.58496...}
\]
multiplications
which is a lot better than $n^2$ when $n$ is huge.
(The process of adding the 3 numbers for each stage is no big deal.)
Some details are missing in the above analysis.
You can check the details for yourself.
The big-O is correct. 

Here's an example.
Suppose we want to multiply
\[
1234 \times 1357
\]

\underline{Step 1: $(1234)(1357)$}.
\begin{align*}
(1234)(1357)
&= (12T^2 + 34) \times (13T^2 + 57) \\
&= A (T^2)^2 + D T^2 + B
\end{align*}
($T = 10$)
where
\begin{align*}
a &= 12, \,\,\, b = 34, \,\,\, c = 13, \,\,\, d = 57 \\
A &= a c = (12)(13)\\
B &= b d = (34)(57)\\
C &= (a + b) (c + d) = (46)(70)\\
D &= C - A - B
\end{align*}

\underline{Step 2: $(12)(34)$}.
\begin{align*}
(12)(13) 
&= (1T^1 + 2) \times (1T^1 + 3) \\
&= A (T^1)^2 + D T^1 + B 
\end{align*}
where
\begin{align*}
a &= 1, \,\,\, b = 2, \,\,\, c = 1, \,\,\, d = 3 \\
A &= a c = (1)(1) = 1\\
B &= b d = (2)(3) = 6\\
C &= (a + b) (c + d) = (3)(4) = 12\\
D &= C - A - B = 12 - 1 - 6 = 5
\end{align*}
So
\begin{align*}
(12)(34) = 1(T^1)^2 + 5T^1 + 6 = 100 + 50 + 6 = 156 
\end{align*}

\underline{Step 3: $(34)(57)$}
\[
(34)(57) = (3T^1 + 4) \times (5T^1 + 7) =  A(T^1)^2 + DT^1 + B 
\]
where
\begin{align*}
a &= 3, \,\,\, b = 4, \,\,\, c = 5, d = 7 \\
A &= a c = (3)(5) = 15\\
B &= b d = (4)(7) = 28\\
C &= (a + b) (c + d) = (7)(12) = 84\\
D &= C - A - B = 84 - 15 - 28 = 41
\end{align*}
So
\begin{align*}
(34)(57) = 15(T^1)^2 + 41T^1 + 28 = 1500 + 410 + 28 = 1938 
\end{align*}

\underline{Step 4: $(46)(70)$}.
\[
(46)(70) 
= (4T^1 + 6) \times (7T^1 + 0) = A(T^1)^2 + DT^1 + B 
\]
where
\begin{align*}
a &= 4, \,\,\, b = 6, \,\,\, c = 7, \,\,\, d = 0 \\
A &= a c = (4)(7) = 28\\
B &= b d = (6)(0) = 0\\
C &= (a + b) (c + d) = (10)(7) = 70\\
D &= C - A - B = 70 - 28 - 0 = 42
\end{align*}
So
\begin{align*}
(46)(70) = 28(T^1)^2 + 42T^1 + 0 = 2800 + 420 + 0 = 3220 
\end{align*}

Putting Steps 2,3,4 back into Step 1 ...

\underline{Continuing Step 1: (1234)(1357)}.
\begin{align*}
(1234)(1357)
&= (12T^2 + 34) \times (13T^2 + 57) \\
&= A (T^2)^2 + D T^2 + B
\end{align*}
where
\begin{align*}
a &= 12, \,\,\, b = 34, \,\,\, c = 13, \,\,\, d = 57 \\
A &= (12)(13) = 156    & & \text{from Step 1}\\
B &= (34)(57) = 1938   & & \text{from Step 2} \\
C &= (46)(70) = 3220   & & \text{from Step 3} \\
D &= C - A - B = 3220 - 156 - 1938 = 1126
\end{align*}
and
\begin{align*}
(1234)(1357)  
&= A(T^2)^2 + D T^2 + B \\
&= 156(T^2)^2 + 1126 T^2 + 1938 \\
&= 1560000 + 112600 + 1938 \\
&= 1674538
\end{align*}

Now if you go back and look for the multiplications you see these:
\begin{align*}
(1)(1) &= 1\\
(2)(3) &= 6\\
(3)(4) &= 12\\
(3)(5) &= 15\\
(4)(7) &= 28\\
(7)(12) &= 84\\
(4)(7) &= 28\\
(6)(0) &= 0\\
(10)(7) &= 70\\
\end{align*}
i.e., $9 = 3^2$ multiplications if we multiply two 
integers of length $n = 2^2$.
High-school multiplication would require $n^2 = (2^2)^2 = 16$
multiplications.


Now let us compute the runtime of Karatsuba's algorithm:

\begin{prop}
  The runtime of Karatsuba is $O(n^{\lg 3}) = O(n^{1.5849...})$
  where $n$ is the length of the integers to be multiplied.
\end{prop}
\proof
From
\[
(aT + b)(cT + d) = (ac)T^2 + [(a+b)(c+d) - (ac) - (bd)]T + (bd)
\]
if $T(n)$ is the runtime where $n$ is the number of digits in the
two integers to be multiplied, we have
\[
T(n) = 3T(n/2) + An + B
\]
Let $n = 2^k$. Then
\begin{align*}
  T(2^k)
  &= 3T(2^{k-1}) + A2^k + B \\
  &= 3(3T(2^{k-2}) + A2^{k-1} + B) + A2^k + B \\
  &= 3^2T(2^{k-2}) + (3\cdot 2^{k-1} + 2^k)A + (3 + 1)B \\
  &= 3^2(3T(2^{k-3}) + A2^{k-2} + B) + (3\cdot 2^{k-1} + 2^k)A + (3 + 1)B \\
  &= 3^3T(2^{k-3}) + (3^2\cdot 2^{k-2} + 3\cdot 2^{k-1} + 2^k)A + (3^2 + 3 + 1)B \\
  &= \cdots \\
  &= 3^kT(2^{k-k})
  + (3^{k-1}\cdot 2 + \cdots + 3^2\cdot 2^{k-2} + 3\cdot 2^{k-1} + 2^k)A
  + (3^{k-1} + \cdots + 3^2 + 3 + 1)B 
\end{align*}
The coefficient of $B$ is $\displaystyle \frac{3^k - 1}{2}$.
The coeffcient of $A$ is
\begin{align*}
  3^{k-1}\cdot 2 + \cdots + 3^2\cdot 2^{k-2} + 3\cdot 2^{k-1} + 2^k
  =
  2 ( 3^{k-1}\cdot 2^0 + \cdots + 3^2\cdot 2^{k-3} + 3^1\cdot 2^{k-2} + 3^0 \cdot 2^{k-1} )
\end{align*}
This is a convolution and is the coeffcient of $x^{k - 1}$ of
\begin{align*}
  2 \cdot \sum_{n \geq 0} 3^n x^n \cdot \sum_{n \geq 0} 2^n x^n
  &=
  2 \cdot \frac{1}{1 - 3x} \cdot \frac{1}{1 - 2x} \\
  &=
  2 \left( \frac{3}{1 - 3x} + \frac{-2}{1 - 2x} \right)\\
  &=
  2 \left( 3 \sum_{n \geq 0} 3^n x^n -2 \sum_{n \geq 0} 2^n x^n \right) \\
  &=
  \sum_{n \geq 0} 2(3^{n+1} - 2^{n+1}) x^n 
\end{align*}
Hence
\[
T(2^k) = 3^k C + 2(3^k - 2^k) A + \frac{3^k - 1}{2} B
\]
where $C = T(1)$.
Since $n = 2^k$, we have
$3^k = 2^{\log_2 3^k} = 2^{k \log_2 3} = (2^k)^{\log_2 3} = n^{\log_2 3}$.
Therefore
\begin{align*}
  T(n)
  &= n^{\log_2 3} C + 2(n^{\log_2 3} - n) A + \frac{n^{\log_2 3} - 1}{2} B \\
  &= (C + 2A + B/2) n^{\log_2 3} - 2A n - B/2 \\
  &= O(n^{\log_2 3}) \\
  &= O(n^{1.5849...})
\end{align*}
\qed

The above result is the same whether you view your integer as an array
of decimals or as an array of bits.

It's a good exercise to implement Karatsuba on your own.
I did some number theory crunching programming during the
summer of my freshman
year and a prof gave me a copy of the original 
paper published by Karatsuba.
(Karatsuba's paper has since inspired several improvements.)
As you see from the above, the amount of math that you need to know
is pretty minimal.
Basically the ingredients are
\[
(aT + b)(cT + d) = acT^2 + (ad + bc)T + bd
\]
(which is not new) and this (which is new):
\begin{align*}
(a+b)(c+d) &= ac + ad + bc + bd \\
\THEREFORE \,\,\, (a+b)(c+d) - ac - bd &= ad + bc
\end{align*}

There are various obvious optimizations which only tweaks the
constants of your big-O, but not the big-O itself. 
For instance if you look at this addition:
\begin{align*}
(1234)(1357)  
&= A(T^2)^2 + D T^2 + B \\
&= 156(T^2)^2 + 1126 T^2 + 1938 \\
&= 1560000 + 112600 + 1938 \\
&= 1674538
\end{align*}
you'll see that you are adding $1560000$ with another number $112600$.
There are some zeroes in $112600$.
So you might want to have a function that adds say $1126$ to an integer
starting at the 100s digit position.
This will speed up for instance adding to an integer $123456789123456789$, 
the number
$123000000000000$:
\begin{Verbatim}
          123456789123456789
        +    123000000000000
        --------------------

        --------------------
               ^
               |
               start column addition here
\end{Verbatim}

Another thing to note is that our computers are mostly 32 bit machines.
If you're using an array of integers to model long integers,
you only want to cut up the integers up to a certain point, and not
to the digit-level.
For instance you might want to model your long integers
with arrays of 16-bit integers.
Note that the process of cutting up the integers into pieces
takes time.
In fact when the number of digits is small, highschool multiplication
might be faster.
Therefore you want to write your code in such as way that 
if the length of the integers is less than a certain constant,
then highschool multiplication is used;
if the length is greater than this constant, then Karatsuba is used.
That's what I did.

(The language I used long long long time ago to do Karatsuba
was Pascal.
Each element of the array models 0 to 999.
I found through timed testing that if the array has length $\leq 5$,
then highschool multiplication was actually faster.
This is of course machine specific.
This explains why when you install some number crunching libraries,
before the installation is completed, it will tests the code to tweak
such constants for maximum performance.)

OK. 
Enough hints.
You should go ahead and write a long integer package that 
incorporates 
Karatsuba for multiplication.


\begin{ex} 
  \label{ex:hill-10}
  \tinysidebar{\debug{exercises/{rsa-43/question.tex}}}
  What if you test for the condition $n = x^3 - y^3$?
  Is there are version of
  factorization using difference of cubes?
  %We have
  %\[
  %  x^3 - y^3 = (x - y)(x^2 + xy + y^2)
  %\]
  %Suppose $n = abc$.
  %\[
  %  \frac{}{}
  %\]

  \solutionlink{sol:hill-10}
  \qed
\end{ex} 
\begin{python0}
from solutions import *
add(label="ex:hill-10",
    srcfilename='exercises/hill-10/answer.tex') 
\end{python0}


\begin{ex} 
  \label{ex:hill-10}
  \tinysidebar{\debug{exercises/{rsa-43/question.tex}}}
  What if you test for the condition $n = x^3 - y^3$?
  Is there are version of
  factorization using difference of cubes?
  %We have
  %\[
  %  x^3 - y^3 = (x - y)(x^2 + xy + y^2)
  %\]
  %Suppose $n = abc$.
  %\[
  %  \frac{}{}
  %\]

  \solutionlink{sol:hill-10}
  \qed
\end{ex} 
\begin{python0}
from solutions import *
add(label="ex:hill-10",
    srcfilename='exercises/hill-10/answer.tex') 
\end{python0}


\begin{ex} 
  \label{ex:hill-10}
  \tinysidebar{\debug{exercises/{rsa-43/question.tex}}}
  What if you test for the condition $n = x^3 - y^3$?
  Is there are version of
  factorization using difference of cubes?
  %We have
  %\[
  %  x^3 - y^3 = (x - y)(x^2 + xy + y^2)
  %\]
  %Suppose $n = abc$.
  %\[
  %  \frac{}{}
  %\]

  \solutionlink{sol:hill-10}
  \qed
\end{ex} 
\begin{python0}
from solutions import *
add(label="ex:hill-10",
    srcfilename='exercises/hill-10/answer.tex') 
\end{python0}


\begin{ex} 
  \label{ex:hill-10}
  \tinysidebar{\debug{exercises/{rsa-43/question.tex}}}
  What if you test for the condition $n = x^3 - y^3$?
  Is there are version of
  factorization using difference of cubes?
  %We have
  %\[
  %  x^3 - y^3 = (x - y)(x^2 + xy + y^2)
  %\]
  %Suppose $n = abc$.
  %\[
  %  \frac{}{}
  %\]

  \solutionlink{sol:hill-10}
  \qed
\end{ex} 
\begin{python0}
from solutions import *
add(label="ex:hill-10",
    srcfilename='exercises/hill-10/answer.tex') 
\end{python0}

