\sectionthree{Vigen\`ere cipher}
\begin{python0}
from solutions import *; clear()
\end{python0}

\newcommand\aaa{\texttt{a}}
\newcommand\yy{\texttt{y}}

The shift, affine, substitution ciphers are called
\defone{monoalphabetic}
ciphers --
each letter can be mapped to only one letter.

A
\defone{polyalphabetic}
ciphers is the opposite
of monoalphabetic ciphers.


The Vigenere cipher is basically a collection of shifts.
Here's an example.

My key is going to be the word \verb!fun!.
Note that when I translate \verb!fun! to numbers
(with
\verb!a! $\rightarrow 0$,
\verb!b! $\rightarrow 1$,
\verb!c! $\rightarrow 2$, ...)
I get $5, 20, 13$.
So all I need to do is to encrypt the characters by doing
\begin{tightlist}
  \li shift by 5
  \li shift by 20
  \li shift by 13
  \li shift by 5
  \li shift by 20
  \li shift by 13
  \li ...
\end{tightlist}
  For instance suppose the plaintext is

\verb!    It's a dangerous business, Frodo, going out your door.!

I change everything to lowercase and throwing away non a-z
and our plaintext \verb!x! is:

\verb!    itsadangerousbusinessfrodogoingoutyourdoor!

Suppose the encrypted text is
\[
  y =
  \texttt{y}_1
  \texttt{y}_2
  \texttt{y}_3
  \texttt{y}_4
  \texttt{y}_5
  \texttt{y}_6
  ...
\]
Then
\begin{tightlist}
  \li $\yy_1$ = shift \verb!i! by 5 = \verb!n!
  \li $\yy_2$ = shift \verb!t! by 20 = \verb!n!
  \li $\yy_3$ = shift \verb!s! by 13 = \verb!f!
  \li $\yy_4$ = shift \verb!a! by 5 = \verb!f!
  \li $\yy_5$ = shift \verb!d! by 20 = \verb!x!
  \li $\yy_6$ = shift \verb!a! by 13 = \verb!n!
  \li $\yy_4$ = shift \verb!n! by 5 = \verb!s!
  \li $\yy_5$ = shift \verb!g! by 20 = \verb!a!
  \li $\yy_6$ = shift \verb!e! by 13 = \verb!r!
  \li ...
\end{tightlist}
i.e.,
\[
  y =
  \texttt{nnffxnsar}...
\]


\begin{ex} 
  \label{ex:rsa-10}
  \tinysidebar{\debug{exercises/{nt-61/question.tex}}}
  In your \verb!Zmod.py!, complete the following methods:
  \begin{enumerate}[nosep]
    \li multiplicative inverse mod $N$ (i.e., \texttt{inv})
    \li invertibility mod $N$ (i.e., \texttt{is\_invertible})
    \li division (i.e., \texttt{\_\_div\_\_})
  \end{enumerate}

  \solutionlink{sol:rsa-10}
  \qed
\end{ex} 
\begin{python0}
from solutions import *
add(label="ex:rsa-10",
    srcfilename='exercises/rsa-10/answer.tex') 
\end{python0}


Now let's attack the Vignere cipher.

Suppose the length of the key is $m$.
Then you can think of your message as being chopped up
into $m$ strings and each is encrypted
by the shift of a character and then everything is
put together.
For instance if the ciphertext is
\[
  y =
  \texttt{y}_1
  \texttt{y}_2
  \texttt{y}_3
  ...
\]
and the key is \texttt{fun}, i.e. $m$ is 3, then the 3 strings are
\begin{align*}
  z_1 =
  \texttt{y}_1
  \texttt{y}_4
  \texttt{y}_7
  \ldots \\
  z_2 =
  \texttt{y}_2
  \texttt{y}_5
  \texttt{y}_8
  \ldots \\
  z_3 =
  \texttt{y}_3
  \texttt{y}_6
  \texttt{y}_9
  \ldots 
\end{align*}
Now, the above of course is the encryption process
and you know that the key has length 3, therefore
you are three different shifts.
If you do know the key has length 3, then each
of the above $z_i$ is encrypted using the same shift
and therefore you can use frequency analysis to decrypt
each of them separately.

The problem is when you do \textit{not} have the length of the
key.
So
the first step is always to figure out the length of the
key, i.e.
$m$.

\textsc{Method 1.}
Here's a strategy. Look at the substrings of the ciphertext of
length 3. Think about it.
Suppose you see the following in your
ciphertext:
\[
  \ldots \texttt{etu} \ldots \texttt{etu}\ldots
\]
What can you do? One might guess that if the distance between the
first \texttt{e} to the next is $d$, then $m$ divides
$d$ right?
Think about it. So you need to scan for all possible substrings of
length 3 and look for such distances. Suppose you have a bunch of
such distances $d_1, d_2, \ldots, d_k$. Then $m$ must divide all
these $d_i$'s. This means that $m$ divides
$\gcd(d_1,d_2,\ldots,d_k)$. Obviously the more $d_i$'s you have
the better.


\begin{ex} 
  \label{ex:rsa-10}
  \tinysidebar{\debug{exercises/{nt-61/question.tex}}}
  In your \verb!Zmod.py!, complete the following methods:
  \begin{enumerate}[nosep]
    \li multiplicative inverse mod $N$ (i.e., \texttt{inv})
    \li invertibility mod $N$ (i.e., \texttt{is\_invertible})
    \li division (i.e., \texttt{\_\_div\_\_})
  \end{enumerate}

  \solutionlink{sol:rsa-10}
  \qed
\end{ex} 
\begin{python0}
from solutions import *
add(label="ex:rsa-10",
    srcfilename='exercises/rsa-10/answer.tex') 
\end{python0}


\begin{ex} 
  \label{ex:rsa-10}
  \tinysidebar{\debug{exercises/{nt-61/question.tex}}}
  In your \verb!Zmod.py!, complete the following methods:
  \begin{enumerate}[nosep]
    \li multiplicative inverse mod $N$ (i.e., \texttt{inv})
    \li invertibility mod $N$ (i.e., \texttt{is\_invertible})
    \li division (i.e., \texttt{\_\_div\_\_})
  \end{enumerate}

  \solutionlink{sol:rsa-10}
  \qed
\end{ex} 
\begin{python0}
from solutions import *
add(label="ex:rsa-10",
    srcfilename='exercises/rsa-10/answer.tex') 
\end{python0}


\begin{ex} 
  \label{ex:rsa-10}
  \tinysidebar{\debug{exercises/{nt-61/question.tex}}}
  In your \verb!Zmod.py!, complete the following methods:
  \begin{enumerate}[nosep]
    \li multiplicative inverse mod $N$ (i.e., \texttt{inv})
    \li invertibility mod $N$ (i.e., \texttt{is\_invertible})
    \li division (i.e., \texttt{\_\_div\_\_})
  \end{enumerate}

  \solutionlink{sol:rsa-10}
  \qed
\end{ex} 
\begin{python0}
from solutions import *
add(label="ex:rsa-10",
    srcfilename='exercises/rsa-10/answer.tex') 
\end{python0}


\begin{ex} 
  \label{ex:rsa-10}
  \tinysidebar{\debug{exercises/{nt-61/question.tex}}}
  In your \verb!Zmod.py!, complete the following methods:
  \begin{enumerate}[nosep]
    \li multiplicative inverse mod $N$ (i.e., \texttt{inv})
    \li invertibility mod $N$ (i.e., \texttt{is\_invertible})
    \li division (i.e., \texttt{\_\_div\_\_})
  \end{enumerate}

  \solutionlink{sol:rsa-10}
  \qed
\end{ex} 
\begin{python0}
from solutions import *
add(label="ex:rsa-10",
    srcfilename='exercises/rsa-10/answer.tex') 
\end{python0}


Once you have a probabilistic guess of the length, say $m$,
you break up the encrypted string into $m$
pieces.
Each piece is a shift ciphertext.
So you just use whatever we talked about
in the sections on shift cipher.

Done!

\textsc{Method 2.}
Here's another technique. It involves some simple counting from
discrete math. Suppose again the length of the key of $m$
and you cut up your string into $m$ pieces so that the characters
in each substring is shifted by the same amount. Clearly the
probabilities of the letters for each substring is preserved. In
other words if \texttt{e} is encrypted as \texttt{g} for the first
string and to \texttt{n} for the second, etc., then the
probability of \texttt{g} in the first string and of \texttt{n} in
the second string must be approximately the same as the
probability of $\texttt{e}$. On the other hand, if we assumed that
the length wrongly, say $m+1$, then the probabilities must be
different. As a matter of fact the substrings would appear very
random. OK. So let's do some math.

Suppose the frequencies of the letters in a string $s$ are
$f_0$,
$f_1$, $\ldots$, $f_{25}$,
i.e., $f_0$ is the frequency of
\texttt{a}, etc.
Suppose the length of the $s$ is $n$.
Therefore
\[
\sum_{i=0}^{25} f_i = n
\]
In this string $s$, the probability $p_0$ that a randomly chosen
character of $s$ is the character \texttt{a} is just $f_0/n$.
Etc.
In the following I might write $f(0)$ instead of $f_0$.
So in the following discussion if you see $p_0$ you can think of $f_0/n$.

In python or C\texttt{++}, for statistical computation
of frequencies of characters (or substrings), you can use a hashtable.
(Details in assignment.)
For instance in python you can do this.
A hashtable in python is called a python dictionary.
In C\texttt{++} a hashtable is called an unordered map.
Try this for python:
\begin{Verbatim}[frame=single,fontsize=\small]
s = "tobeornottobethatisthequestionandtheanswerisfortytwo"
f = {}
for c in s:
    if c not in f:
        f[c] = 1
    else:
        f[c] += 1    
for c,count in f:
    print(c, count)    
\end{Verbatim}
or
\begin{Verbatim}[frame=single,fontsize=\small]
s = "tobeornottobethatisthequestionandtheanswerisfortytwo"
f = {}
for c in s:
    f[c] = f.get(c, 0) + 1
for c,count in f.items():
    print(c, count)    
\end{Verbatim}
Make sure you run the above.
(For those of you who have taken CISS350, 
check my notes on hashtables and review \verb!std::unordered_map!.)
A lot more information on python dictionary is found in the assignment.
Moving on ...

Now the
probability of two randomly chosen characters being the same must
be \sidenote{ $\binom{n}{k}$ is the $n$--choose--$k$ binomial
coefficient, i.e. $n!/((n-k)!k!)$. In particular $\binom{n}{2} =
n(n-1)/2$.}
\[
  I(s) = \frac{\binom{f_0}{2} + \ldots + \binom{f_{25}}{2}}{\binom{n}{2}} =
  \frac{\sum_{i=0}^{25}f_i(f_i-1)}{n(n-1)}
\]

One thing nice about the above formula is that if any substitution
has been applied to the string,
\textit{the above value remains the same}.


\begin{ex} 
  \label{ex:rsa-10}
  \tinysidebar{\debug{exercises/{nt-61/question.tex}}}
  In your \verb!Zmod.py!, complete the following methods:
  \begin{enumerate}[nosep]
    \li multiplicative inverse mod $N$ (i.e., \texttt{inv})
    \li invertibility mod $N$ (i.e., \texttt{is\_invertible})
    \li division (i.e., \texttt{\_\_div\_\_})
  \end{enumerate}

  \solutionlink{sol:rsa-10}
  \qed
\end{ex} 
\begin{python0}
from solutions import *
add(label="ex:rsa-10",
    srcfilename='exercises/rsa-10/answer.tex') 
\end{python0}


So $I(s)$ is approximately $\sum_{i=0}^{25} p_i^2$ where $p_i$ is
the probability of a character of the string being the $i$--th
letter.

On the other hand if there is no pattern and everything is random,
for instance you cut up your ciphertext, which was encrypted by a
substitution with a string of length $m$, into $m+1$ substrings,
then the probabilities of the characters would be almost the same
since the string is gibberish. In other words, the probabilities
would be about $f_i/n = 1/26$.


\begin{ex} 
  \label{ex:rsa-10}
  \tinysidebar{\debug{exercises/{nt-61/question.tex}}}
  In your \verb!Zmod.py!, complete the following methods:
  \begin{enumerate}[nosep]
    \li multiplicative inverse mod $N$ (i.e., \texttt{inv})
    \li invertibility mod $N$ (i.e., \texttt{is\_invertible})
    \li division (i.e., \texttt{\_\_div\_\_})
  \end{enumerate}

  \solutionlink{sol:rsa-10}
  \qed
\end{ex} 
\begin{python0}
from solutions import *
add(label="ex:rsa-10",
    srcfilename='exercises/rsa-10/answer.tex') 
\end{python0}


\begin{ex} 
  \label{ex:rsa-10}
  \tinysidebar{\debug{exercises/{nt-61/question.tex}}}
  In your \verb!Zmod.py!, complete the following methods:
  \begin{enumerate}[nosep]
    \li multiplicative inverse mod $N$ (i.e., \texttt{inv})
    \li invertibility mod $N$ (i.e., \texttt{is\_invertible})
    \li division (i.e., \texttt{\_\_div\_\_})
  \end{enumerate}

  \solutionlink{sol:rsa-10}
  \qed
\end{ex} 
\begin{python0}
from solutions import *
add(label="ex:rsa-10",
    srcfilename='exercises/rsa-10/answer.tex') 
\end{python0}


So the question is this: Is the $I$ value sufficiently different
for a meaningful string and a random string? Is so, then we have
an algorithm for determining $m$:

Test $m = 1$:
You get one piece (the complete string) from the ciphertext
$y$.
Compute $I(y)$.
If $I(y)$ is approximately $0.065$, $m = 1$.

Test $m = 2$:
Cut up your ciphertext $y$ into two pieces
$z_1, z_2$.
Compute
$I(z_1)$,
$I(z_1)$.
If they are approximately $0.065$ (you can use the average of
$I(z_1)$ and 
$I(z_1)$), then $m = 2$.

Test $m = 3$:
Cut up your ciphertext $y$ into three pieces
$z_1, z_2, z_3$.
Compute
$I(z_1)$,
$I(z_2)$,
$I(z_3)$.
If they are approximately $0.065$ (you can use the average of all three),
then $m = 3$.

Etc.

So now we know how to compute the length of the key.
Suppose the plaintext is $x$ and the $m$ pieces of $x$ are
$x_1, x_2, x_3, ..., x_m$.
Recall $y$ is the ciphertext and the $m$ pieces of $y$ are
$z_1, z_2, z_3, ..., z_m$.
Suppose the
shifts are by $k_1, k_2, \ldots k_m$.
So 
$z_1$ is a shift of $x_1$ by $k_1$, $z_2$ is a shift of
$x_2$ by $k_2$, etc.

Suppose $p_0$ is the probability of \texttt{a}, $p_1$ is the
probability of \texttt{b}, etc. in unencrypted English.  Look at
$y_1$. If you look at the probability of $\texttt{a}$ in $z_1$,
then you're really looking at the $p_\alpha$ where $\alpha$ is
encrypted as $\texttt{a}$. For instance suppose the shift is by 3.
In other words \text{x} is encrypted as \texttt{a}. So the
probability of \texttt{a} in $z_1$ is the same as the probability
of \texttt{x} which is $p_{26-3}$. For simplicity, from now on all the indices
will be considered mod 26. So when I write $p_{-3}$ I really mean
$p_{26-3}$. In general, if the shift is $k$, then the probability
must be $p_{-k}$. Similarly, the probability of \texttt{b} in
$y_1$ must be $p_{1-k}$. In general, the probability of the
$i$--th character in $z_1$ must be $p_{i-k}$.

Now we make the following observation: What is the probability
that a randomly chosen character from $z_1$ and a randomly chosen
character from $z_2$ are both \texttt{a}? It is approximately
\[
 M(y_i,y_j) = \sum_{\ell=0}^{25} p_{\ell-k_1} p_{\ell-k_2}
\]
For any $z_i$ and $z_j$, this is then
\[
 \sum_{\ell=0}^{25} p_{\ell-k_i} p_{\ell-k_j}
\]
You now notice that the above is the same as
\[
 \sum_{\ell=0}^{25} p_{\ell} p_{\ell-k_j+k_i}
\]
or
\[
 \sum_{\ell=0}^{25} p_{\ell} p_{\ell+k_i-k_j}
\]

The number $k_i - k_j$ is called the
\defone{relative shift}
of $z_i$ and
$z_j$.


\begin{ex} 
  \label{ex:rsa-10}
  \tinysidebar{\debug{exercises/{nt-61/question.tex}}}
  In your \verb!Zmod.py!, complete the following methods:
  \begin{enumerate}[nosep]
    \li multiplicative inverse mod $N$ (i.e., \texttt{inv})
    \li invertibility mod $N$ (i.e., \texttt{is\_invertible})
    \li division (i.e., \texttt{\_\_div\_\_})
  \end{enumerate}

  \solutionlink{sol:rsa-10}
  \qed
\end{ex} 
\begin{python0}
from solutions import *
add(label="ex:rsa-10",
    srcfilename='exercises/rsa-10/answer.tex') 
\end{python0}


You notice that when there is no shift, the $M$ value is
approximately $0.065$. Otherwise it is much smaller, around 0.04.
So this what you can do. Take $z_1$ and $z_2$ as examples. Compute
the $M$ value for $z_1$ and $z_2$. Next you shift $z_2$ by 1 to
get $E_1(z_2)$ ($E_1$ is the encryption for the shift cipher). Now
compute the $M$ value for $y_1$ and $E_1(y_2)$, i.e. $M(y_1,
E_1(y_2))$. Next you compute $M(y_1, E_2(y_2))$. Continue until
you have the $M$ value of $y_1$ and $E_{25}(y_1)$. If the $M$
value for $y_1$ and $E_{17}(y_2)$ is approximately $0.065$, then
the relative shift is very likely 17. In other words
\[
  k_1 - k_2 = 17
\]

Now perform the same for different pairs of $y_i$, $y_j$. You have
$\binom{m}{2}$ pairs and you hope to get as many equations as
possible.

You should be able to write your shifts in terms of one shift, say
the first $k_1$. For instance the second shift might be $k_1+3$,
the third is $k_1+2$, etc. You still have the unknown $k_1$.

But that's only one unknown!!!

You can then easily solve it with 26
different values for $k_1$. Alternatively, for a Vigenere cipher,
the shifts come from a sequence of letters and the letters usually
make up a meaningful word. You can scan for meaningful words using
your sequence of shifts. For instance if the shifts are
$(k_1,k_1+17,k_1+4,k_1+21,k_1+10)$, then when $k_1=9$ you get the
word \texttt{janet}.

Notice that the cryptanalysis as above already hints at the fact
that algebra and probability theory are extremely important in
this area of study. In particular, there is an area of study
called information theory that studies the amount of randomness or
lack of information in random variables encoded within something
called the entropy of random variables.  Using information theory
one can prove that a substitution ciphertext has a unique
decryption if the string has a length of at least 25. 

In case the above proof hurts your head too much, let
me give you an informal proof.

Suppose I have a sequence of numbers:
\[
[0.1, 0.2, 0.3, 0.4]
\]
Think of these as the probabilities of a language that involves
only 4 letters, say the probabilities of the symbols $a,b,c,d$.
Now I rotate the numbers in a circle by 1 step to get this:
\[
[0.4, 0.1, 0.2, 0.3]
\]
and by 3 steps to get this:
\[
[0.2, 0.3, 0.4, 0.1]
\]
These corresponds to a shift cipher of 1 and of 3.
The relative shift between the 1st and 2nd is 2 steps:
you can see that 0.1 is 2 steps away between the two sequences.
One way to compute the relative shift (i.e. 2) is to 
compute the sum of products of the corresponding terms:
\begin{enumerate}
\li Relative shift the second by 0:
\begin{align*}
&[0.4, 0.1, 0.2, 0.3] \\
&[0.2, 0.3, 0.4, 0.1]
\end{align*}
and compute:
\[
0.4 \times 0.2 + 0.1 \times 0.3 + 0.2 \times 0.4 + 0.3 \times 0.1 = 0.2200...
\]
\li Relative shift the second by -1:
\begin{align*}
&[0.4, 0.1, 0.2, 0.3] \\
&[0.3, 0.4, 0.1, 0.2]
\end{align*}
and compute:
\[
0.4 \times 0.3 + 0.1 \times 0.4 + 0.2 \times 0.1 + 0.3 \times 0.2 = 0.2399...
\]
\li Relative shift the second by -2
\begin{align*}
&[0.4, 0.1, 0.2, 0.3]\\
&[0.4, 0.1, 0.2, 0.3]
\end{align*}
and compute
\[
0.4 \times 0.4 + 0.1 \times 0.1 + 0.2 \times 0.2 + 0.3 \times 0.2 = 0.3000...
\]
\li Relative shift the second by -3:
\begin{align*}
&[0.4, 0.1, 0.2, 0.3]\\
&[0.1, 0.2, 0.3, 0.4]
\end{align*}
and compute
\[
0.4 \times 0.1 + 0.1 \times 0.2 + 0.2 \times 0.3 + 0.3 \times 0.4 = 0.2399...
\]
\end{enumerate}
You see that the largest value gives you the correct relative shift.

Of course if the probabilities were from ciphertexts,
the numbers will only be
\textit{close} to $0.1,0.2,0.3,0.4$
if the lengths are large enough.
   
(By the way, the reason why you get the largest value
when the numbers line up nicely is because the
above sum of product of corresponding term is called
the inner product and the inner product
is largest when the two sequence of positive numbers, 
think vectors, are pointing in the same direction.)


\begin{ex} 
  \label{ex:rsa-10}
  \tinysidebar{\debug{exercises/{nt-61/question.tex}}}
  In your \verb!Zmod.py!, complete the following methods:
  \begin{enumerate}[nosep]
    \li multiplicative inverse mod $N$ (i.e., \texttt{inv})
    \li invertibility mod $N$ (i.e., \texttt{is\_invertible})
    \li division (i.e., \texttt{\_\_div\_\_})
  \end{enumerate}

  \solutionlink{sol:rsa-10}
  \qed
\end{ex} 
\begin{python0}
from solutions import *
add(label="ex:rsa-10",
    srcfilename='exercises/rsa-10/answer.tex') 
\end{python0}


\begin{ex} 
  \label{ex:rsa-10}
  \tinysidebar{\debug{exercises/{nt-61/question.tex}}}
  In your \verb!Zmod.py!, complete the following methods:
  \begin{enumerate}[nosep]
    \li multiplicative inverse mod $N$ (i.e., \texttt{inv})
    \li invertibility mod $N$ (i.e., \texttt{is\_invertible})
    \li division (i.e., \texttt{\_\_div\_\_})
  \end{enumerate}

  \solutionlink{sol:rsa-10}
  \qed
\end{ex} 
\begin{python0}
from solutions import *
add(label="ex:rsa-10",
    srcfilename='exercises/rsa-10/answer.tex') 
\end{python0}


\begin{ex} 
  \label{ex:rsa-10}
  \tinysidebar{\debug{exercises/{nt-61/question.tex}}}
  In your \verb!Zmod.py!, complete the following methods:
  \begin{enumerate}[nosep]
    \li multiplicative inverse mod $N$ (i.e., \texttt{inv})
    \li invertibility mod $N$ (i.e., \texttt{is\_invertible})
    \li division (i.e., \texttt{\_\_div\_\_})
  \end{enumerate}

  \solutionlink{sol:rsa-10}
  \qed
\end{ex} 
\begin{python0}
from solutions import *
add(label="ex:rsa-10",
    srcfilename='exercises/rsa-10/answer.tex') 
\end{python0}


