%-*-latex-*-
\sectionthree{Attacking the shift cipher}
\begin{python0}
from solutions import *; clear()
\end{python0}

Recall that for the shift cipher, we have
\begin{align*}
 E_k(x) &\equiv (x+k) \,\,\,(\operatorname{mod} 26) \\
 D_k(x) &\equiv (x-k) \,\,\,(\operatorname{mod} 26)
\end{align*}

Since $0 \leq k \leq 25$, there are not many possible values for
$k$. So you can try to apply
\textit{all} the possible decryptions
$D_0$, $D_1$, $D_2$, $\ldots$, $D_{25}$. Assuming the message is
something Eve can read, all that is required is that Eve has
enough time to read 26 messages. Easy!
Since Eve is trying all keys, this is a brute force attack.

But in fact Eve can do better, by using
a heuristic approach based on probability.
If the message is long enough, based
on letter frequencies, heuristically, she can try to tabulate the
frequencies of all the characters and then assume the most common
occurring character is decrypted as \texttt{e} which is
statistically the most commonly occurring letter in English.
\sidenote{Too bad if the message is in Russian.} Of course knowing
how to decrypt to \texttt{e} is sufficient for Eve to decrypt all
the letters right?

The following is a table of probabilities for each letter used in
English.
\[
 \begin{tabular}{|c|c|}
   \hline
   % after \\: \hline or \cline{col1-col2} \cline{col3-col4} ...
   Letter & Probability \\
   \hline
   \texttt{e} &  0.127 \\ \hline
   \texttt{t} &  0.091 \\
   \texttt{a} &  0.082 \\
   \texttt{o} &  0.075 \\
   \texttt{i} &  0.070 \\
   \texttt{n} &  0.067 \\
   \texttt{s} &  0.063 \\
   \texttt{h} &  0.061 \\
   \texttt{r} &  0.060 \\ \hline
   \texttt{d} &  0.043 \\
   \texttt{l} &  0.040 \\ \hline
   \texttt{c} &  0.028 \\
   \texttt{u} &  0.028 \\
   \texttt{m} &  0.024 \\
   \texttt{w} &  0.023 \\
   \texttt{f} &  0.022 \\
   \texttt{g} &  0.020 \\
   \texttt{y} &  0.020 \\
   \texttt{p} &  0.019 \\
   \texttt{b} &  0.015 \\ \hline
   \texttt{v} &  0.010 \\
   \texttt{k} &  0.008 \\
   \texttt{j} &  0.002 \\
   \texttt{x} &  0.001 \\
   \texttt{q} &  0.001 \\
   \texttt{z} &  0.001 \\
   \hline
 \end{tabular}
\]

Of course these are probabilities. It does not mean that the
second most frequently occurring letter
\textit{must} be \texttt{t}!
I've divided up the probabilities into groups according to jumps
in the values.

It is also useful to know that besides commonly occurring letters,
which \textit{pairs} of letters occurring frequently next to each
other. These are called
\defone{digrams}
(or 2--grams). For three,
they are called
\defone{trigrams}
(or 3--grams).
The following is a
table of commonly occurring digrams and trigrams listed in
decreasing order of frequencies:

\[
 \begin{tabular}{|c|l|}
   \hline
   $n$ & $n$-grams (in decreasing order)\\
   \hline
   2 & \texttt{th he in er an re ed on es st en at to nt ha nd} \\
     & \texttt{ou ea ng as or ti is et it ar te se hi of} \\
     \hline
   3 & \texttt{the ing and her ere ent tha nth was eth for dth} \\
   \hline
 \end{tabular}
\]

Here are the frequencies of the 2-grams (of course not all 2-grams, just the
top few):
\[
 \begin{tabular}{|c|c|}
   \hline
   % after \\: \hline or \cline{col1-col2} \cline{col3-col4} ...
   2-gram & Probability \\
   \hline
\texttt{th} &  0.0271\\
\texttt{he} &  0.0233\\
\texttt{in} &  0.0203\\
\texttt{er} &  0.0178\\
\texttt{an} &  0.0161\\
\texttt{re} &  0.0141\\
\texttt{es} &  0.0132\\
\texttt{on} &  0.0132\\
\texttt{st} &  0.0125\\
\texttt{nt} &  0.0117\\
\texttt{en} &  0.0113\\
\texttt{at} &  0.0112\\
\texttt{ed} &  0.0108\\
\texttt{nd} &  0.0107\\
\texttt{to} &  0.0107\\
\texttt{or} &  0.0106\\
\texttt{ea} &  0.0100\\
\texttt{ti} &  0.0099\\
\texttt{ar} &  0.0098\\
\texttt{te} &  0.0098\\
\texttt{ng} &  0.0089\\
\texttt{al} &  0.0088\\
\texttt{it} &  0.0088\\
\texttt{as} &  0.0087\\
\texttt{is} &  0.0086\\
\texttt{ha} &  0.0083\\
\texttt{et} &  0.0076\\
\texttt{se} &  0.0073\\
\texttt{ou} &  0.0072\\
\texttt{of} &  0.0071\\
   \hline
 \end{tabular}
\]
and the 3-grams:
\[
 \begin{tabular}{|c|c|}
   \hline
   % after \\: \hline or \cline{col1-col2} \cline{col3-col4} ...
   3-gram & Probability \\
   \hline
\texttt{the} &  0.0181\\                
\texttt{and} &  0.0073\\                
\texttt{ing} &  0.0072\\                
\texttt{ent} &  0.0042\\                
\texttt{ion} &  0.0042\\                
\texttt{her} &  0.0036\\                
\texttt{for} &  0.0034\\                
\texttt{tha} &  0.0033\\                
\texttt{nth} &  0.0033\\                
\texttt{int} &  0.0032\\                
\texttt{ere} &  0.0031\\
\texttt{tio} &  0.0031\\
\texttt{ter} &  0.0030\\
\texttt{est} &  0.0028\\
\texttt{ers} &  0.0028\\
\texttt{ati} &  0.0026\\
\texttt{hat} &  0.0026\\
\texttt{ate} &  0.0025\\
\texttt{all} &  0.0025\\
\texttt{eth} &  0.0024\\
\texttt{hes} &  0.0024\\
\texttt{ver} &  0.0024\\
\texttt{his} &  0.0024\\
\texttt{oft} &  0.0022\\
\texttt{ith} &  0.0021\\
\texttt{fth} &  0.0021\\
\texttt{sth} &  0.0021\\
\texttt{oth} &  0.0021\\
\texttt{res} &  0.0021\\
\texttt{ont} &  0.0020\\
   \hline
 \end{tabular}
\]
and the 4-grams:
\[
 \begin{tabular}{|c|c|}
   \hline
   % after \\: \hline or \cline{col1-col2} \cline{col3-col4} ...
   4-gram & Probability \\
   \hline
\texttt{tion} &  0.31\\        
\texttt{nthe} &  0.27\\        
\texttt{ther} &  0.24\\        
\texttt{that} &  0.21\\        
\texttt{ofth} &  0.19\\        
\texttt{fthe} &  0.19\\        
\texttt{thes} &  0.18\\        
\texttt{with} &  0.18\\        
\texttt{inth} &  0.17\\        
\texttt{atio} &  0.17\\        
\texttt{othe} &  0.16\\        
\texttt{tthe} &  0.16\\        
\texttt{dthe} &  0.15\\        
\texttt{ingt} &  0.15\\        
\texttt{ethe} &  0.15\\        
\texttt{sand} &  0.14\\        
\texttt{sthe} &  0.14\\        
\texttt{here} &  0.13\\        
\texttt{thec} &  0.13\\        
\texttt{ment} &  0.12\\        
\texttt{them} &  0.12\\
\texttt{rthe} &  0.12\\
\texttt{thep} &  0.11\\
\texttt{from} &  0.10\\
\texttt{this} &  0.10\\
\texttt{ting} &  0.10\\
\texttt{thei} &  0.10\\
\texttt{ngth} &  0.10\\
\texttt{ions} &  0.10\\
\texttt{andt} &  0.10\\
   \hline
 \end{tabular}
\]

So a slight improve to brute force search of trying
$k=0,1,2,\ldots,25$, is to try encrypt \texttt{e} to the most
commonly occurring letter, the second, the third, etc.

(Some authors use $M$ for their set of plaintexts instead of $P$.
In that case they might call their plaintexts messages.)


\begin{ex} 
  \label{ex:rsa-10}
  \tinysidebar{\debug{exercises/{nt-61/question.tex}}}
  In your \verb!Zmod.py!, complete the following methods:
  \begin{enumerate}[nosep]
    \li multiplicative inverse mod $N$ (i.e., \texttt{inv})
    \li invertibility mod $N$ (i.e., \texttt{is\_invertible})
    \li division (i.e., \texttt{\_\_div\_\_})
  \end{enumerate}

  \solutionlink{sol:rsa-10}
  \qed
\end{ex} 
\begin{python0}
from solutions import *
add(label="ex:rsa-10",
    srcfilename='exercises/rsa-10/answer.tex') 
\end{python0}


\begin{ex} 
  \label{ex:rsa-10}
  \tinysidebar{\debug{exercises/{nt-61/question.tex}}}
  In your \verb!Zmod.py!, complete the following methods:
  \begin{enumerate}[nosep]
    \li multiplicative inverse mod $N$ (i.e., \texttt{inv})
    \li invertibility mod $N$ (i.e., \texttt{is\_invertible})
    \li division (i.e., \texttt{\_\_div\_\_})
  \end{enumerate}

  \solutionlink{sol:rsa-10}
  \qed
\end{ex} 
\begin{python0}
from solutions import *
add(label="ex:rsa-10",
    srcfilename='exercises/rsa-10/answer.tex') 
\end{python0}

