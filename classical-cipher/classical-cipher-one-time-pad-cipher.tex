\sectionthree{One-time pad cipher}
\begin{python0}
from solutions import *; clear()
\end{python0}

The one time pad is easy:
Suppose Bob wants to send a message.
Bob will need to translate this into bits.
(For instance use the ASCII code or some other agreed upon format.)
Say the plaintext (in bits) is $00101011$.

The key is a very long sequence of random bits that
Bob and Alice has agreed upon.
Suppose the sequence is $0101101011101110001010101100110101001$.

Bob takes his message $00101011$ and exclusive-or with the first eight bits of
the key $0101101011101110001$:
\begin{align*}
  &\texttt{00101011} \\
  &\texttt{\underline{01011010}11101110001010101100110101001} \\
  &\texttt{\underline{01110001}}
\end{align*}
He then removes the 8 bits used:
\[
  \texttt{\textso{01011010}11101110001010101100110101001} \\
\]
and sends the ciphertex (in bits) 01110001 to Alice.

Once Alice received 01110001, she exclusive-or with her key:
\begin{align*}
  &\texttt{01110001} \\
  &\texttt{\underline{01011010}11101110001010101100110101001} \\
  &\texttt{\underline{00101011}}
\end{align*}
which is the plaintext.
She also removes the bits in her key used in the decryption:
\texttt{\textso{01011010}11101110001010101100110101001}.

The next time the communicate, they will start with the remaining bits
of their key.

That's it.

It's a very simple cipher.
We don't have enough math (yet) to prove it, but you can sense the this
cipher is extremely secure.
Why?


\begin{ex} 
  \label{ex:rsa-10}
  \tinysidebar{\debug{exercises/{nt-61/question.tex}}}
  In your \verb!Zmod.py!, complete the following methods:
  \begin{enumerate}[nosep]
    \li multiplicative inverse mod $N$ (i.e., \texttt{inv})
    \li invertibility mod $N$ (i.e., \texttt{is\_invertible})
    \li division (i.e., \texttt{\_\_div\_\_})
  \end{enumerate}

  \solutionlink{sol:rsa-10}
  \qed
\end{ex} 
\begin{python0}
from solutions import *
add(label="ex:rsa-10",
    srcfilename='exercises/rsa-10/answer.tex') 
\end{python0}


There's a rumor that during the Cold War, Washington D.C. communicates with
Moscow using one time pad.

To be secure the key must be a random sequence.
Furthermore, the key cannot be reused.
The other problem is that the key is really long.

(Using the concept of entropy of information theory, Claude Shannon can prove that
the ciphertext contains no information about the plaintext, other than the length.)



