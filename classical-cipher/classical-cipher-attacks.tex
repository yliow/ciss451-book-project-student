\sectionthree{Attacks}
\begin{python0}
from solutions import *; clear()
\end{python0}

OK. Now it's time to attack some of the classical cryptosystems.
Before we do that we want to know exactly what is it we want to
achieve.


Recall that a cryptosystems is made up of the encryption and
decryption function $E$ and $D$.

Suppose Alice wants to send a message $m$ to Bob. She encrypts the
message to get $E_k(m)$ and delivers $E_k(m)$ to Bob. Bob decrypts
$E_k(m)$ by applying $D_k$ and get $D_k(E_k(m)) = m$.

Now let's think about the rogue agent Eve. What does she want?

The most common mode of attack assumes Eve receives a copy of
$E_k(m)$ and she wants to derive $m$. It's even better if she can
derive $k$ because in that case she actually has $D_k$ and hence
can decrypt all future ciphertexts. (Don't forget we assume that
the encryption and decryption algorithm is known. That means if
Eve has $k$, she has $D_k$ and in fact also $E_k$. Only the key(s)
is secret). Not only that. If she has $k$, she also has $E_k$ and
therefore she can actually impersonate Alice!

The various assumptions of what Eve has are called
\defone{attack models}
or
\defone{attack modes}
Here are some standard ones.

\begin{myenum}

\item The
  \defone{known ciphertext attack}
  is an attack where the
  ciphertext of a plaintext is available to the attacker. In other words, this
  attack involves computing $k$ from $E_k(m)$:
  \[
  E_k(m) \mapsto k
  \]
  This include the case when there is more than one ciphertext.
  
\item
  The
  \defone{known plaintext attack}
    is an attack where the
  Eve has some messages and their ciphertext.
  So her goal is this:
  \[
  m_1, m_2, ..., m_n, E_k(m_1), E_k(m_2), ...,  E_k(m_n)  \mapsto k
  \]
  This does happen.
  For instance the breaking of Germany's Enigma ciphers during WWII
  is based on this assumption.
  
\item
  The
  \defone{chosen plaintext attack}
   is an attack where the
  Eve can actually encrypt messages/plaintexts that she
  chooses.
  Make sure you see that this is different from known plaintext.
  
\item
  The
  \defone{chosen ciphertext attack}
  is an attack where the
  Eve has access to the decryptions of ciphertexts that she chooses.
  Her goal is to compute the key.
  Make sure you see that this is different from known plaintext.
  For instance suppose Eve chooses (a meaningless) ciphertext $x$
  and sends it to the decryption machine.
  Suppose the decryption of $x$ is $y$.
  It's very likely that $y$ is totally meaningless.
  Bob then calls Alice and ask her why she sends the gibberish $y$ message.
  On listening to their phone conversation, she now knows that
  \[
  D_k(x) = y
  \]
  Eve then attempts to find $k$.
  
  
\end{myenum}
