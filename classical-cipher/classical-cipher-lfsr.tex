\sectionthree{Linear feedback shift register}
\begin{python0}
from solutions import *; clear()
\end{python0}

Recall that the
one time pad uses exclusive-or to operator on
bit sequences.
Note that the exclusive-or is the same as
addition mod 2!!!
In other words addition \textit{in mod 2}
\begin{align*}
  0 + 0 \equiv 0 \pmod{2} \\
  0 + 1 \equiv 1 \pmod{2} \\
  1 + 0 \equiv 1 \pmod{2} \\
  1 + 0 \equiv 0 \pmod{2} \\
\end{align*}
is the same as exclusive-or operation \textit{on bits}:
\begin{align*}
  0 \oplus 0 = 0  \\
  0 \oplus 1 = 1  \\
  1 \oplus 0 = 1  \\
  1 \oplus 0 = 0 
\end{align*}
(I'm using $\oplus$ for exclusive-or bit operator -- that's pretty standard.)

Now I'm going to this:
first I define the following bit sequence of length 5:
\[
  x_1x_2x_3x_4x_5 = 10110
\]
which is the same as defining integer $x_1, ..., x_5$ in $\Z/2$.
Then I define
\[
  x_{n + 6} \equiv x_{n+1} + x_{n+2} + x_{n+4} \pmod{2}
\]
for $n \geq 0$.
For instance
\[
  x_{6} \equiv x_{1} + x_{2} + x_{4} \equiv 1 + 0 + 1 \equiv 0 \pmod{2} 
\]
Here are next about 15: 
\begin{align*}
  x_{6} &\equiv x_{1} + x_{2} + x_{4} \equiv 1 + 0 + 1 \equiv 0  & \pmod{2} \\
  x_{7} &\equiv x_{2} + x_{3} + x_{5} \equiv 0 + 1 + 0 \equiv 1  & \pmod{2} \\
  x_{8} &\equiv x_{3} + x_{4} + x_{6} \equiv 1 + 1 + 0 \equiv 0  & \pmod{2} \\
  x_{9} &\equiv x_{4} + x_{5} + x_{7} \equiv 1 + 0 + 1 \equiv 0  & \pmod{2} \\
  x_{10} &\equiv x_{5} + x_{6} + x_{8} \equiv 0 + 0 + 0 \equiv 0 & \pmod{2} \\
  x_{11} &\equiv x_{6} + x_{7} + x_{9} \equiv 0 + 1 + 0 \equiv 1 & \pmod{2} \\
  x_{12} &\equiv x_{7} + x_{8} + x_{10} \equiv 1 + 0 + 1 \equiv 1 & \pmod{2} \\
  x_{13} &\equiv x_{8} + x_{9} + x_{11} \equiv 0 + 0 + 1 \equiv 1 & \pmod{2} \\
  x_{14} &\equiv x_{9} + x_{10} + x_{12} \equiv 0 + 0 + 1 \equiv 1 & \pmod{2} \\
  x_{15} &\equiv x_{10} + x_{11} + x_{13} \equiv 0 + 1 + 1 \equiv 0 & \pmod{2} \\
  x_{16} &\equiv x_{11} + x_{12} + x_{14} \equiv 1 + 1 + 1 \equiv 1 & \pmod{2} \\
  x_{17} &\equiv x_{12} + x_{13} + x_{15} \equiv 1 + 1 + 0 \equiv 0 & \pmod{2} \\
  x_{18} &\equiv x_{13} + x_{14} + x_{16} \equiv 1 + 1 + 1 \equiv 1 & \pmod{2} \\
  x_{19} &\equiv x_{14} + x_{15} + x_{17} \equiv 1 + 0 + 0 \equiv 1 & \pmod{2} \\
  x_{20} &\equiv x_{15} + x_{16} + x_{18} \equiv 0 + 1 + 1 \equiv 0 & \pmod{2} \\
\end{align*}
More generally, after defining $x_1, ..., x_5$ (the initial conditions) you can
generated the sequence $x_i$ using
\[
  x_{n + 6} \equiv c_1 x_{n+1} + c_2 x_{n+2} + c_3 x_{n+3} + c_4x_{n+4} + c_5x_{n+5} \pmod{2}
\]
for $n \geq 0$ for constants $c_1, ..., c_5$ in $\Z/2$.
I will say that this is linear relation has \defone{degree 5}.
Even more generally, you can have any number of bits for the initial condition.
Say you begin with $x_1, ..., x_k$ (the initial condition) and
the relation is
\[
  x_{n + k + 1} \equiv c_1 x_{n+1} + c_2 x_{n+2} + c_3 x_{n+3} + \cdots + c_kx_{n+k} \pmod{2}
\]

LSRFs are very easy to implement in both hardware and software
and they are extremely fast
(they just access bits and XOR them).
The LSRF generator itself need to remember the $k$ bits
$c_1, ..., c_k$ (which is fixed)
and the $k$ bits of the sequence so far
$x_{n + 1}, ..., x_{n + k}$ in order to generate the next
bit $x_{n + k + 1}$.

\begin{comment}
x = [1,0,1,1,0]
c = [1,1,0,1,0]
# x_{n + 6} = 1x_{n+1} + 1x_{n+2} + 0x_{n+3} + 1x_{n+4} + 0x_{n+5}
def f(c, x):
    y = x[-5:]
    s = sum([a*b for (a,b) in zip(c,y)]) % 2
    x.append(s)

#print x
for i in range(1000):
    f(c, x)

found = False
y = None   
for length in range(1, 100):
    y = x[:length]
    z = x[length:]
    numtimes = 0
    while 1:
        y0 = z[:length]
        if y0 != y:
            break
        numtimes += 1
        z = z[length:]
        if len(z) < length:
            found = True
    if found: break

if found:
    print y
    print length
    print numtimes
\end{comment}

In the above example, the sequence from $x_1$ to $x_{20}$ is
\[
10110010001111010110010001111010110...
\]
Notice that the pattern repeats itself:
\[
101100100011110\ \ 101100100011110\ \ 10110...
\]
The period is 15.

The problem with the one-time pad is that you need to
generate a random sequence of 0s and 1s.
You can see that with 5 bits
\[
  x_1 x_2 x_3 x_4 x_5 = 10110
\]
and the relation
\[
  x_{n + 6} \equiv x_{n+1} + x_{n+2} + x_{n+4} \pmod{2}
\]
which involves (1,1,0,1,0) (5 bits), a total of 10 bits)  we can generate
\[
  101100100011110
\]
which has length 15.
The 15 bits is somewhat random -- we say that the 15 bits are
\defone{pseudorandom}.
Therefore LFSR can be used to generate a pseudorandom bit sequence,
which can 
used, for instance, as a key for the one-time pad.
Of course you want to find a LSFR with extremely long periods.


\begin{ex} 
  \label{ex:rsa-10}
  \tinysidebar{\debug{exercises/{nt-61/question.tex}}}
  In your \verb!Zmod.py!, complete the following methods:
  \begin{enumerate}[nosep]
    \li multiplicative inverse mod $N$ (i.e., \texttt{inv})
    \li invertibility mod $N$ (i.e., \texttt{is\_invertible})
    \li division (i.e., \texttt{\_\_div\_\_})
  \end{enumerate}

  \solutionlink{sol:rsa-10}
  \qed
\end{ex} 
\begin{python0}
from solutions import *
add(label="ex:rsa-10",
    srcfilename='exercises/rsa-10/answer.tex') 
\end{python0}


\begin{ex} 
  \label{ex:rsa-10}
  \tinysidebar{\debug{exercises/{nt-61/question.tex}}}
  In your \verb!Zmod.py!, complete the following methods:
  \begin{enumerate}[nosep]
    \li multiplicative inverse mod $N$ (i.e., \texttt{inv})
    \li invertibility mod $N$ (i.e., \texttt{is\_invertible})
    \li division (i.e., \texttt{\_\_div\_\_})
  \end{enumerate}

  \solutionlink{sol:rsa-10}
  \qed
\end{ex} 
\begin{python0}
from solutions import *
add(label="ex:rsa-10",
    srcfilename='exercises/rsa-10/answer.tex') 
\end{python0}


\begin{ex} 
  \label{ex:rsa-10}
  \tinysidebar{\debug{exercises/{nt-61/question.tex}}}
  In your \verb!Zmod.py!, complete the following methods:
  \begin{enumerate}[nosep]
    \li multiplicative inverse mod $N$ (i.e., \texttt{inv})
    \li invertibility mod $N$ (i.e., \texttt{is\_invertible})
    \li division (i.e., \texttt{\_\_div\_\_})
  \end{enumerate}

  \solutionlink{sol:rsa-10}
  \qed
\end{ex} 
\begin{python0}
from solutions import *
add(label="ex:rsa-10",
    srcfilename='exercises/rsa-10/answer.tex') 
\end{python0}


\begin{ex} 
  \label{ex:rsa-10}
  \tinysidebar{\debug{exercises/{nt-61/question.tex}}}
  In your \verb!Zmod.py!, complete the following methods:
  \begin{enumerate}[nosep]
    \li multiplicative inverse mod $N$ (i.e., \texttt{inv})
    \li invertibility mod $N$ (i.e., \texttt{is\_invertible})
    \li division (i.e., \texttt{\_\_div\_\_})
  \end{enumerate}

  \solutionlink{sol:rsa-10}
  \qed
\end{ex} 
\begin{python0}
from solutions import *
add(label="ex:rsa-10",
    srcfilename='exercises/rsa-10/answer.tex') 
\end{python0}


\begin{ex} 
  \label{ex:rsa-10}
  \tinysidebar{\debug{exercises/{nt-61/question.tex}}}
  In your \verb!Zmod.py!, complete the following methods:
  \begin{enumerate}[nosep]
    \li multiplicative inverse mod $N$ (i.e., \texttt{inv})
    \li invertibility mod $N$ (i.e., \texttt{is\_invertible})
    \li division (i.e., \texttt{\_\_div\_\_})
  \end{enumerate}

  \solutionlink{sol:rsa-10}
  \qed
\end{ex} 
\begin{python0}
from solutions import *
add(label="ex:rsa-10",
    srcfilename='exercises/rsa-10/answer.tex') 
\end{python0}


\begin{ex} 
  \label{ex:rsa-10}
  \tinysidebar{\debug{exercises/{nt-61/question.tex}}}
  In your \verb!Zmod.py!, complete the following methods:
  \begin{enumerate}[nosep]
    \li multiplicative inverse mod $N$ (i.e., \texttt{inv})
    \li invertibility mod $N$ (i.e., \texttt{is\_invertible})
    \li division (i.e., \texttt{\_\_div\_\_})
  \end{enumerate}

  \solutionlink{sol:rsa-10}
  \qed
\end{ex} 
\begin{python0}
from solutions import *
add(label="ex:rsa-10",
    srcfilename='exercises/rsa-10/answer.tex') 
\end{python0}

