\section{Classical ciphers: instructions}

\begin{enumerate}
\item
  At my website, in the Tutorials section, you'll find \verb!latex.pdf!.
  Post \LaTeX\ questions in CCCS discord.
\item
  In \verb!thispreamble.tex!, change \verb!AUTHOR! and \verb!SHORTAUTHOR!
  to your name.
\item
  To speed up compilation, in \verb!chap-classical-ciphers.tex!, you
  might want to comment out some sections using \verb!%!.
\item
  Rewrite the contents of this chapter in your own words, otherwise your book
  is considered plagiarized.
  (You probably want to make a copy of this directory.)
  Note that you need not rewrite the questions in the exercises.
  You may retain the chapter and section organization (and their titles).
\item
  Every cipher in my notes must be present in your notes.
  You can add extra ciphers not found in my notes.
  (Example: enigma, playfair, etc.)
\item
  For each cipher, have a complete definition of each cipher
  and then have at least one example on encryption and decryption.
  Include definitions of terms.
  Your example(s) must be different from the examples in my notes.
\item
  You must write in proper English and using proper mathematical style.
\item
  Think of your notes as the only notes you can use in an open-book test or
  open-book final exam.
  Therefore you need not include historical or pedagogical remarks
  (but that's up to you).
\item
  Solve as many exercises as you can.
  The exercises are stored in directory \verb!exercises!.
  For instance if you see \verb!
\begin{ex} 
  \label{ex:rsa-10}
  \tinysidebar{\debug{exercises/{nt-61/question.tex}}}
  In your \verb!Zmod.py!, complete the following methods:
  \begin{enumerate}[nosep]
    \li multiplicative inverse mod $N$ (i.e., \texttt{inv})
    \li invertibility mod $N$ (i.e., \texttt{is\_invertible})
    \li division (i.e., \texttt{\_\_div\_\_})
  \end{enumerate}

  \solutionlink{sol:rsa-10}
  \qed
\end{ex} 
\begin{python0}
from solutions import *
add(label="ex:rsa-10",
    srcfilename='exercises/rsa-10/answer.tex') 
\end{python0}
!, this means
  the question of this exercise is stored in
  \verb!\tinysidebar{\debug{exercises/{nt-61/question.tex}}}
  In your \verb!Zmod.py!, complete the following methods:
  \begin{enumerate}[nosep]
    \li multiplicative inverse mod $N$ (i.e., \texttt{inv})
    \li invertibility mod $N$ (i.e., \texttt{is\_invertible})
    \li division (i.e., \texttt{\_\_div\_\_})
  \end{enumerate}
!
  and the answer should be written in
  \verb!\tinysidebar{\debug{exercises/{hill-9/answer.tex}}}

    Solution not provided.
    !
\item
  In terms of writing style, technically speaking, in formal
  writings, you should not use personal noun like \lq\lq I".
  Instead, \lq\lq we" should be used.
  For instance instead of saying
  \[
  \text{\lq\lq I will now prove my theorem."}
  \]
  you should write
  \[
  \text{\lq\lq We will now prove the (or our) theorem."}
  \]
  I use \lq\lq I" just to make my notes informal.
  For your book, you should use the formal writing style.
\item
  When you are done with this chapter, comment out this section of
  instructions.
\end{enumerate}
