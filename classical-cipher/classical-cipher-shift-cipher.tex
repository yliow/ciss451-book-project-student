\sectionthree{Shift cipher}
\begin{python0}
from solutions import *; clear()
\end{python0}

This is one of the earliest cryptosystems and apparently Julius Caesar used it 
(high tech, eh?).
To encrypt a message, you simply do the following:
\[
\alpha \mapsto \delta, \ ...
\]
or in our alphabet system:
\begin{align*}
a &\mapsto d \\
b &\mapsto e \\
c &\mapsto f \\
\vdots &\textwhite{\mapsto d} \vdots 
\end{align*}
i.e. $a$ is replaced by $d$, $b$ is replaced by $e$, etc.
And of course you \lq\lq go around in a circle'': $x$ is replaced by $a$,
$y$ is replaced by $b$, $z$ is replaced by $c$.
\begin{align*}
a &\mapsto d \\
b &\mapsto e \\
c &\mapsto f \\
\vdots &\textwhite{\mapsto d} \vdots \\
x &\mapsto a \\
y &\mapsto b \\
z &\mapsto c
\end{align*}
This is known as the
\defone{Caesar cipher}.

Here's a simple Python code to encrypt a character:
\begin{console}[fontsize=\footnotesize]
def E(x):
    i = ord(x) - ord('a')
    i = (i + 3) % 26
    return chr(ord('a') + i)
\end{console}
and for \cpp:
\begin{console}[fontsize=\footnotesize]
char E(char x)
{
    return (x - 'a' + 3) % 26 + 'a';
}
\end{console}

Of course there's no reason why you must \lq\lq shift by 3''.
You can also shift by 7.
Right?
Note that the encryption input is only a single character.
Of course for a string, you just encrypt character-by-character.
Also, uppercase is replaced by lowercase.
Furthermore, 
anything not $a$-$z$ (example: punctuations, spaces) is thrown away.

Easy, right?
OK. Break the following the code:

\begin{center}
\includegraphics[width=5in]{dino-ciphertext.jpg}
\end{center}

\begin{center}
*** WARNING: SPOILERS ON THE NEXT PAGE *** 
\end{center}

\newpage
\begin{center}
\includegraphics[width=5in]{dino-plaintext.jpg}
\end{center}


\begin{ex} 
  \label{ex:rsa-10}
  \tinysidebar{\debug{exercises/{nt-61/question.tex}}}
  In your \verb!Zmod.py!, complete the following methods:
  \begin{enumerate}[nosep]
    \li multiplicative inverse mod $N$ (i.e., \texttt{inv})
    \li invertibility mod $N$ (i.e., \texttt{is\_invertible})
    \li division (i.e., \texttt{\_\_div\_\_})
  \end{enumerate}

  \solutionlink{sol:rsa-10}
  \qed
\end{ex} 
\begin{python0}
from solutions import *
add(label="ex:rsa-10",
    srcfilename='exercises/rsa-10/answer.tex') 
\end{python0}


Let $P$ and $C$ be two sets.
A
\defone{cipher}
is just a pair of functions
$E: P \rightarrow C$ and $D: C \rightarrow P$
where $E$ is called the
\defone{encryption}
and $D$ is called the
\defone{decryption}
such that
\[
D(E(x)) = x \,\,\, \text{ for all } x \in P
\]
In other words
if you decrypt what you have encrypted, you get back the same data.
(It'd better be so!)
An element of $P$ is called a
\defone{plaintext} --
it's what you encrypt.
An element of $C$ is called a
\defone{ciphertext}
--
it's what you get when you encrypt.

Caesar cipher is an example of a cipher.
For Caesar cipher $P = C = \{a, b, c, ..., z\}$.
Also, although the encryption function maps one character to another,
it's understood that if you want to encrypt a string, you
simply encrypt each character of the string and join them up into
a string.

But if you allow the shift amount in the Caesar cipher to change,
then the encryption and decryption depends on the shift amount -- the key.
A general principle in cryptography is the following
concept due to
\href{https://en.wikipedia.org/wiki/Auguste_Kerckhoffs}{Auguste Kerckhoffs}:

\defone{Kerckhoffs' principle} (1883):
A secure cipher should not depend on the secrecy of the encryption
and decryption algorithm, but rather on the secrecy of the key.


The opposite and a really bad idea is called 
\defone{security through obscurity},
i.e., it's the hope that your
encrypted messages are safe as long as the encryption and decryption
algorithm are kept secret.

Why is this important?
Because it's easy to change the key while changing the encryption
and decryption algorithm might not be that easy.
If a worker who performs the encryption or decryption process is captured,
then he/she can be made (tortured?) to reveal the algorithm.
On the other hand, if the key is stolen, then we can simply change the key.
So in cryptography, it is always assumed that the algorithms (the cipher)
cannot be kept secret for long.

In fact in modern cryptography, once a cipher is designed,
the cryptography researcher(s) is expected to publish the cipher
so that other researchers can check if the cipher is actually secure.

So we just need to modify the definition of our cipher:

\begin{defn}
A
\defone{cipher}
is
a pair of functions
$E: K \times P \rightarrow C$ and
$D: K \times C \rightarrow P$ such that
if $k \in K$,
\[
D(k, E(k, x)) = x \,\,\, \text{ for all } x \in P
\]
$P$ is the set of plaintexts, $C$ is the set of ciphertexts,
and $K$ is the set of keys.
Notice that in the above the key used for encryption $k$
is the same as the key used for decryption.
\end{defn}

Instead of writing $E(k, x)$ and $D(k, x)$,
it's also common to write $E_{k}(x)$ and $D_{k}(x)$.
Depending on which book you read, the encryption and decryption functions
can also be written $e$ instead of $E$ and $d$ instead of $D$.

Humans have used ciphers for thousands of years.
The early ciphers always use the same key for encryption and decryption.
It was only very recently in 1970 that
\href{https://en.wikipedia.org/wiki/James_H._Ellis}{James H.~Ellis},
asked if it's possible to have a cipher that uses two distinct keys,
one for encryption and one for decryption.
Ellis was a British cryptographer at the
\href{https://en.wikipedia.org/wiki/GCHQ}{GCHQ }
(UK Government Communications
Headquarters).
If this is possible, then only the decyption key has to be kept secret.
Why do you want to a use such a cipher?

Well, I can publish the encryption key for such a cipher on a
website, you encrypt with the encryption key and send me the ciphertext by
email.
On receiving the ciphertext, I decrypt it using the decryption key.
Note that I can make the encryption key public, but I must keep
the decryption key a secret.
On the other hand for a symmetric key cipher, we would have to meet secretly
and decide on the common key.

Surprisingly such a cipher exists.

So now I have to modify our definition of ciphers ...

\begin{defn}
A
\defone{symmetric cipher}
is
a pair of functions
$E: K \times P \rightarrow C$ and
$D: K \times C \rightarrow P$ such that
if $k \in K$,
\[
D_k(E_k(x)) = x \,\,\, \text{ for all } x \in P
\]
$P$ is the set of plaintexts, $C$ is the set of ciphertexts,
and $K$ is the set of keys.
A symmetric cipher is also called a \defone{private key cipher}
because the key used must be kept private.
\end{defn}

And of course we also must have 

\begin{defn}
An
\defone{asymmetric cipher}
is a cipher where there are two distinct keys,
one for encryption and one for decryption.
An asymmetric cipher is also called a
\defone{public key cipher}
because the encryption key can be
made public (but the decryption key has to kept secret).
In this case, if $k,k'$ are the encryption and decryption keys,
then the cipher must satisfy
\[
D_{k'}(E_k(x)) = x
\]
for $x \in P$.
The encryption key $k$ is called the \defone{public key} (because it can be
made public)
while the decryption key is called the \defone{private key}.
\end{defn}


\begin{ex} 
  \label{ex:rsa-10}
  \tinysidebar{\debug{exercises/{nt-61/question.tex}}}
  In your \verb!Zmod.py!, complete the following methods:
  \begin{enumerate}[nosep]
    \li multiplicative inverse mod $N$ (i.e., \texttt{inv})
    \li invertibility mod $N$ (i.e., \texttt{is\_invertible})
    \li division (i.e., \texttt{\_\_div\_\_})
  \end{enumerate}

  \solutionlink{sol:rsa-10}
  \qed
\end{ex} 
\begin{python0}
from solutions import *
add(label="ex:rsa-10",
    srcfilename='exercises/rsa-10/answer.tex') 
\end{python0}


Public key ciphers use quite a bit of math.
So we won't see public key ciphers for a while.

Let's go back to our Caesar cipher.
You can think of the Caesar cipher as a special case of a symmetric cipher that
uses the key 3:
\begin{myenum}
\li encryption is \lq\lq shift forward by 3''
\li decryption is \lq\lq shift backward by 3''.
\end{myenum}
In other words, generalizing the Caesar cipher, we get the shift cipher:
\begin{myenum}
\li encryption is \lq\lq shift forward by $k$''
\li decryption is \lq\lq shift backward by $k$''.
\end{myenum}
where $k$ is the key.
I hope it's clear that the shift cipher with key 27 is
the same as the shift cipher with key 1.
Effectively speaking there are only 26 shifts, including
the very bad key of 0.
Hence for the shift cipher, $K = \{0, 1, 2, ..., 25\}$.

For classical ciphers, assuming we're only interested in English,
the plaintexts are
strings involving $a$-$z$.
I will write $\{a,b,c,...,z\}^*$ for the set of all strings with
characters from $\{a,b,c,...,z\}$.
If $n$ is a positive integer, I will also write $\{a,b,c,...,z\}^n$
for the set of strings with length $n$ and with characters
from the set $\{a,b,c,...,z\}$.
For instance
\[
\{a,b,c,...,z\}^2 = \{aa, ab, ac, ..., zx, zy, zz\}
\]


\begin{ex} 
  \label{ex:rsa-10}
  \tinysidebar{\debug{exercises/{nt-61/question.tex}}}
  In your \verb!Zmod.py!, complete the following methods:
  \begin{enumerate}[nosep]
    \li multiplicative inverse mod $N$ (i.e., \texttt{inv})
    \li invertibility mod $N$ (i.e., \texttt{is\_invertible})
    \li division (i.e., \texttt{\_\_div\_\_})
  \end{enumerate}

  \solutionlink{sol:rsa-10}
  \qed
\end{ex} 
\begin{python0}
from solutions import *
add(label="ex:rsa-10",
    srcfilename='exercises/rsa-10/answer.tex') 
\end{python0}


In modern day cryptography, we frequently work with bit strings.
The set of all bit strings is denoted by $\{0,1\}^*$.
Bit strings of length exactly 8 is denoted by $\{0,1\}^8$ --
these would be bytes.
For instance you might have heard of the SHA2 family of hash function.
SHA256 takes in bit strings and spits out bit strings of length 256.
So SHA256 is a function of type
\[
\{0,1\}^* \rightarrow \{0,1\}^8
\]
(Technically speaking SHA256 inputs do have a maximum limit in length,
but it's so huge that for practical purposes it's as good as all possible bit
strings.)

You know this is coming ... we'll be using
\textit{lots} of math to do 
encryption and decryption.
In particular, for this notes, we associate letters $a$ to $z$ with numbers.
The encryption and decryption function will work with either numbers of 
letters.
Specifically we have the following correspondence:
\sidebar{
In math, \lq\lq correspondence'' is the same as 
\lq\lq 1-1 correspondence'' which is the same as
\lq\lq bijection''.
Remember bijection?
It's time to check your discrete math notes}
\begin{align*}
a &\leftrightarrow 0 \\
b &\leftrightarrow 1 \\
c &\leftrightarrow 2 \\
\vdots  &\textwhite{\leftarrow 2} \vdots \\
z &\leftrightarrow 25
\end{align*}

if $E$ encrypts $a$ to $c$, I will say either 
\[
E(a) = c
\]
or 
\[
E(0) = 2
\]

Now you might say ... \lq\lq so what's the big deal? 
Why rewrite $a$ as $0$,  $b$ as $1$, etc.
I can also come up with some secret encoding for instance
why can't I rewrite $a$ as a square, $b$ as a triangle, etc.?''

Well ... the reason is because $0, 1, 2, ...$ are numbers ... and ...
\textit{they have operations (addition, subtraction, multiplication, division).}

With the above in mind, instead of describing the shift cipher as
functions on $a$-$z$, I'll describe it as function on $\Z/26$:

\begin{defn}
The \defone{shift cipher} $(E, D)$ is given by
\[
  E_k(x) = x + k \pmod{26}
\]
and
\[
  D_k(x) = x - k \pmod{26}
\]
\end{defn}

It's clear that for the shift cipher $(E, D)$,
\[
D_k(E_k(x)) \equiv x \pmod{26}
\]

And of course the shift cipher with key $k = 3$ is the \defone{Caesar cipher}.
