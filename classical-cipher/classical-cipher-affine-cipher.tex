\sectionthree{Affine cipher}
\begin{python0}
from solutions import *; clear()
\end{python0}

Suppose we use $\Z/26$ instead of \verb!'a'! to \verb!'z'! again.
Recall that the shift cipher is
\[
  E(k, x) = (x + k) \pmod{26}
\]
and
\[
    D(k, x) = (x - k) \pmod{26}
\]

The benefit of translating our encryption/decryption
\lq\lq shift up" and \lq\lq shift down" into
mathematical operations in $\Z/26$, is that
we can now generalize and use different mathematical formulas!

Here's the affine cipher.
The encryption algorithm for the affine cipher looks like this:
\[
E((a,b), x) = (ax + b) \pmod{26}
\]
Note that the key is not one single number -- the key
is $(a, b)$ mod 26 which is made up of two numbers.
Why is that important?

Because this means that there are more key values!
Which means that Eve has to try more key!!!!
Get it??

But what is the decryption function?
Of course we know that
\[
D((a,b), E((a,b), x)) = x \pmod{26}
\]
This means that
\[
D((a,b), ax + b) = x \pmod{26}
\]
Suppose I make a guess ...
When I look at the shift cipher, I notice that the
decryption function is similar in form to the encryption function.
Maybe the decryption function for the affine cipher
is similar in form to the encryption function???

Let's try
\[
  D((a,b), x) = cx + d \pmod{26}
\]
In that case, from 
\[
D((a,b), ax + b) \equiv x \pmod{26}
\]
we would get
\[
c(ax + b)) + d \equiv x \pmod{26}
\]
which gives us
\[
cax + cb + d \equiv x \pmod{26}
\]
Now what?
Well maybe
\[
cax \equiv x \pmod{26}
\]
and
\[
cb + d \equiv 0 \pmod{26}
\]

The first condition
\[
cax \equiv x \pmod{26}
\]
is achieved if we have
\[
ca \equiv 1 \pmod{26}
\]

Can this be done?


\begin{ex} 
  \label{ex:rsa-10}
  \tinysidebar{\debug{exercises/{nt-61/question.tex}}}
  In your \verb!Zmod.py!, complete the following methods:
  \begin{enumerate}[nosep]
    \li multiplicative inverse mod $N$ (i.e., \texttt{inv})
    \li invertibility mod $N$ (i.e., \texttt{is\_invertible})
    \li division (i.e., \texttt{\_\_div\_\_})
  \end{enumerate}

  \solutionlink{sol:rsa-10}
  \qed
\end{ex} 
\begin{python0}
from solutions import *
add(label="ex:rsa-10",
    srcfilename='exercises/rsa-10/answer.tex') 
\end{python0}



Given an integer $a$, if $c$ satisfies
\[
  ca \equiv 1 \pmod{26}
\]
we say that $c$ is a
\defone{multiplicative inverse}
of $a$ mod 26.
We usually write $c$ as $a^{-1} \mod 26$.
Note that $a^{-1} \mod 26$ is NOT a fraction!!!
It's a whole number.
If $a$ has a multiplicative inverse mod 26, we say that
$a$ is
\defone{invertible}
mod 26.

Recall from a previous section, you already
have a table of $a^{-1} \mod 26$ for all $a$'s in mod 26.


\begin{ex} 
  \label{ex:rsa-10}
  \tinysidebar{\debug{exercises/{nt-61/question.tex}}}
  In your \verb!Zmod.py!, complete the following methods:
  \begin{enumerate}[nosep]
    \li multiplicative inverse mod $N$ (i.e., \texttt{inv})
    \li invertibility mod $N$ (i.e., \texttt{is\_invertible})
    \li division (i.e., \texttt{\_\_div\_\_})
  \end{enumerate}

  \solutionlink{sol:rsa-10}
  \qed
\end{ex} 
\begin{python0}
from solutions import *
add(label="ex:rsa-10",
    srcfilename='exercises/rsa-10/answer.tex') 
\end{python0}



Therefore to
satisfy
\[
ca \equiv 1 \pmod{26}
\]
we can't just pick any $a$.
We have to pick an $a$ with a multiplicative inverse mod 26.

After we have chosen a good $a$, what do we do?
We then have
\[
  D((a,b), x) = cx + d \pmod{26}
\]
where $c$ is the multiplicative inverse of $a$ mod 26.
But what about $d$???

Remember we still have the condition
\[
cb + d \equiv 0 \pmod{26}
\]
Writing $a^{-1} \mod 26$ for $c$, we get
\[
a^{-1}b + d \equiv 0 \pmod{26}
\]
we get
\[
d \equiv -a^{-1}b \pmod{26}
\]

Therefore we have the following:
The affine cipher is
\[
E((a,b), x) = (ax + b) \pmod{26}
\]
where $a$ is invertible mod 26 and
\begin{align*}
  D((a,b), x)
  \equiv a^{-1}x - a^{-1}b\pmod{26} \\
  \equiv a^{-1}(x - b)\pmod{26} \\
\end{align*}


\begin{ex} 
  \label{ex:rsa-10}
  \tinysidebar{\debug{exercises/{nt-61/question.tex}}}
  In your \verb!Zmod.py!, complete the following methods:
  \begin{enumerate}[nosep]
    \li multiplicative inverse mod $N$ (i.e., \texttt{inv})
    \li invertibility mod $N$ (i.e., \texttt{is\_invertible})
    \li division (i.e., \texttt{\_\_div\_\_})
  \end{enumerate}

  \solutionlink{sol:rsa-10}
  \qed
\end{ex} 
\begin{python0}
from solutions import *
add(label="ex:rsa-10",
    srcfilename='exercises/rsa-10/answer.tex') 
\end{python0}



The above however uses a lot of \lq\lq iffy'' math.
For instance we used the fact
\[
  a(b + c) \equiv ab + ac \pmod{26}
\]
(where?)
We also use the fact that if 
\[
  a + b \equiv 0 \pmod{26}
\]
then
\[
  a \equiv -b \pmod{26}
\]
We seem to be treated math in mod 26 like math in $\Z$
and $\R$!!!
Is that justifiable?
It turns out that the above algebra is actually correct.
I'll have to come back to that later otherwise people will think
we are rambling nonsense and making things up.


\begin{ex} 
  \label{ex:rsa-10}
  \tinysidebar{\debug{exercises/{nt-61/question.tex}}}
  In your \verb!Zmod.py!, complete the following methods:
  \begin{enumerate}[nosep]
    \li multiplicative inverse mod $N$ (i.e., \texttt{inv})
    \li invertibility mod $N$ (i.e., \texttt{is\_invertible})
    \li division (i.e., \texttt{\_\_div\_\_})
  \end{enumerate}

  \solutionlink{sol:rsa-10}
  \qed
\end{ex} 
\begin{python0}
from solutions import *
add(label="ex:rsa-10",
    srcfilename='exercises/rsa-10/answer.tex') 
\end{python0}


Now let's look at attacking the affine cipher.

Recall the encryption and decryption of the affine cipher looks
like
\[
 E_{a,b}(x) \equiv (ax + b) \,\,\,(\operatorname{mod} 26),
 \,\,\,\,\,\,\,\,\,\,
 D_{a,b}(x) \equiv a^{-1}(x - b) \,\,\,(\operatorname{mod} 26)
\]
Note that the key is $(a,b)$. 
Note also that $a$ must be invertible mod 26.

Therefore (by the multiplication principle in discrete mathematics),
the total numbers of keys is
\[
\phi(26) \cdot 26 = 312
\]
This is not that big, but it's definitely bigger than the number of keys
for the shift cipher (which is 26).
This means that to carry out a brute force attack on an affine
cipher, assume the attacker has the cipher,
he/she must try 312 possible keys.


\begin{ex} 
  \label{ex:rsa-10}
  \tinysidebar{\debug{exercises/{nt-61/question.tex}}}
  In your \verb!Zmod.py!, complete the following methods:
  \begin{enumerate}[nosep]
    \li multiplicative inverse mod $N$ (i.e., \texttt{inv})
    \li invertibility mod $N$ (i.e., \texttt{is\_invertible})
    \li division (i.e., \texttt{\_\_div\_\_})
  \end{enumerate}

  \solutionlink{sol:rsa-10}
  \qed
\end{ex} 
\begin{python0}
from solutions import *
add(label="ex:rsa-10",
    srcfilename='exercises/rsa-10/answer.tex') 
\end{python0}


Again you can do a brute force search for $a,b$. After all there are
not that many possibilities for $a$ and $b$. But we can do better if
we use letter (1--gram) frequencies again. Again suppose you have
computed the frequencies of the letters of the ciphertext and say
that \texttt{g} is the most common letter. So you assume \texttt{e}
is encrypted as \texttt{g}. This is the same as saying $4$ is
encrypted as $6$, i.e.,
\[
 E_{a,b}(4) = 6
\]
right? Now using the formula for $E_{a,b}$ we get
\[
 4a + b = 6
 \]
 To be accurate the equation should be
\[
 4a + b \equiv 6 \pmod{26}
 \]
 
Now suppose the second most common letter in the ciphertext is
\texttt{y}. So you assume that \texttt{t} is encrypted as
\texttt{y}. This means
\[
 20a + b \equiv 24 \pmod{26}
\]
Right? Yes, no? Think about it. So you can solve for $a$ and $b$
from the linear equations
\begin{alignat*}
 4a + b  &\equiv   6 &&\pmod{26}\\
 20a + b &\equiv 24  &&\pmod{26}
\end{alignat*}


\begin{ex} 
  \label{ex:rsa-10}
  \tinysidebar{\debug{exercises/{nt-61/question.tex}}}
  In your \verb!Zmod.py!, complete the following methods:
  \begin{enumerate}[nosep]
    \li multiplicative inverse mod $N$ (i.e., \texttt{inv})
    \li invertibility mod $N$ (i.e., \texttt{is\_invertible})
    \li division (i.e., \texttt{\_\_div\_\_})
  \end{enumerate}

  \solutionlink{sol:rsa-10}
  \qed
\end{ex} 
\begin{python0}
from solutions import *
add(label="ex:rsa-10",
    srcfilename='exercises/rsa-10/answer.tex') 
\end{python0}


\begin{ex} 
  \label{ex:rsa-10}
  \tinysidebar{\debug{exercises/{nt-61/question.tex}}}
  In your \verb!Zmod.py!, complete the following methods:
  \begin{enumerate}[nosep]
    \li multiplicative inverse mod $N$ (i.e., \texttt{inv})
    \li invertibility mod $N$ (i.e., \texttt{is\_invertible})
    \li division (i.e., \texttt{\_\_div\_\_})
  \end{enumerate}

  \solutionlink{sol:rsa-10}
  \qed
\end{ex} 
\begin{python0}
from solutions import *
add(label="ex:rsa-10",
    srcfilename='exercises/rsa-10/answer.tex') 
\end{python0}


\begin{ex} 
  \label{ex:rsa-10}
  \tinysidebar{\debug{exercises/{nt-61/question.tex}}}
  In your \verb!Zmod.py!, complete the following methods:
  \begin{enumerate}[nosep]
    \li multiplicative inverse mod $N$ (i.e., \texttt{inv})
    \li invertibility mod $N$ (i.e., \texttt{is\_invertible})
    \li division (i.e., \texttt{\_\_div\_\_})
  \end{enumerate}

  \solutionlink{sol:rsa-10}
  \qed
\end{ex} 
\begin{python0}
from solutions import *
add(label="ex:rsa-10",
    srcfilename='exercises/rsa-10/answer.tex') 
\end{python0}

