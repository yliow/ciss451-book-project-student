\sectionthree{Euclidean algorithm -- GCD}
\begin{python0}
from solutions import *; clear()
\end{python0}

Now let me use the Euclidean property to compute the gcd of two integers.

Let's use the division algorithm on $20$ and $6$.
\[
20 = 6 \cdot 3 + 2, \,\,\,\,\, 0 \leq 2 < 6
\]
Suppose I want to compute $\gcd(20, 6)$.
Of course the example is small enough that we know that it is $2$.
But notice something about this:
\[
20 = 6 \cdot 3 + 2, \,\,\,\,\, 0 \leq 2 < 6
\]
If $d$ is a divisor of $20$ and $6$, then it must also divide
$2$.
Therefore $\gcd(20, 6)$ must divide $2$.
The converse might not be true.
In general, we have this crucial
bridge between Euclidean property and 
common divisors:

\begin{ex}
Let $A + B + C = 0$ where $A, B, C \in \Z$ where $(A, B) \neq (0,0)$
and $(B,C) \neq (0,0)$.
Prove that $\gcd(A, B) = \gcd(B, C)$.
\end{ex}

\begin{ex}
Let $A, B, C, x \in \Z$ with $(A,B) \neq (0,0)$ and $(B,C) \neq (0,0)$.
If $A + Bx + C = 0$,
then $\gcd(A, B) = \gcd(B, C)$.
\end{ex}

\begin{lem}
  \textnormal{(\defone{GCD Lemma})}
  Let $a,b,q,r \in \Z$ such that
  not both $a,b$ are zero and not both $b,r$ are zero.
  If
  \[
  a = bq + r
  \]
  then
  \[
  \gcd(a,b) = \gcd(b, r)
  \]
\end{lem}
\proof
TODO
\qed


In particular, 
given $a, b \in \Z$ where $a > b > 0$.
By the Euclidean property of $\Z$, there exist $q,r \in \Z$ such that
\[
a = bq + r, \,\,\, 0 \leq r < b
\]
Hence
\[
\gcd(a,b) = \gcd(b, r)
\]

Note that in the above, I only require $a = bq + r$.
For instance for to $\gcd(120, 15)$, I can use
$120 = 1\cdot 15 + (120 - 15)$, i.e., $a = 120, b = 15, q = 1, r = 120-15$.
Then $\gcd(120, 15) = \gcd(15, 120-15) = \gcd(15, 105)$.

However if I use the division algorithm, then 
$r$ is \lq\lq small'':
\[
0 \leq r < b
\]
So if you want to compute $\gcd(a,b)$, make sure $a \geq b$ (otherwise
swap them).
Then $\gcd(a, b) = \gcd(b, r)$ and you would have $a \geq b > r$.
So instead of computing $\gcd(a, b)$, you are better off
computing $\gcd(b, r)$.

But like I said, we do not need the $q$ and $r$ to be the quotient
and remainder.
For instance suppose I want to compute the GCD of 514 and 24.
\[
514 = 24 \cdot 1 + (514 - 24)
\]
Then 
\[
\gcd(514, 24) = \gcd(24, 514 - 24)
\]
which gives us
\[
\gcd(514, 24) = \gcd(24, 490)
\]

Note that $\gcd(0, n) = n$ for any positive integer $n$.

Of course this gives rise to the following algorithm
\begin{Verbatim}[frame=single,fontsize=\footnotesize]
ALGORITHM: GCD
INPUTS:    a, b
OTPUT:     gcd(a, b)

if b > a:
    swap a, b

if b == 0:
    return a
else:
    return GCD(a - b, b) 
\end{Verbatim}

This only subtracts one copy of $b$ from $a$.
Suppose we can compute
\[
a = bq + r, \,\,\,\,\, 0 \leq r < b
\]
Then
\[
\gcd(a, b) = \gcd(b, r) 
\]
Of course $r$ is the remainder when $a$ is divided by $b$.
Using this we rewrite the above code to get
the
\defterm{Euclidean Algorithm}\index{Euclidean Algorithm}\tinysidebar{Euclidean Algorithm}:
\begin{Verbatim}[frame=single,fontsize=\footnotesize]
ALGORITHM: GCD (Euclidean algorithm)
INPUTS:    a, b
OUTPUT:    gcd(a, b)

if b > a:
    # To make sure that for gcd(a,b), a >= b
    swap a, b

if b == 0: 
    return a
else:
    return GCD(b, a % b)
\end{Verbatim}
Note that if \verb!a! $<$ \verb!b!, then
\[
\texttt{GCD(a, b) = GCD(b, a \% b) = GCD(b, a)}
\]
Therefore the swap is not necessary:
\begin{Verbatim}[frame=single,fontsize=\footnotesize]
ALGORITHM: GCD (Euclidean algorithm)
INPUTS:    a, b
OUTPUT:    gcd(a, b)

if b == 0: 
    return a
else:
    return GCD(b, a % b)
\end{Verbatim}
In this case, I'm assuming that \verb!a % b! is available.
As an example:
\begin{align*}
  \gcd(514, 24)
  &= \gcd(24, 514 \% 24) = \gcd(24, 10) \\
  &= \gcd(10, 24 \% 10) = \gcd(10, 4) \\
  &= \gcd(10, 10 \% 4) = \gcd(10, 2) \\
  &= \gcd(2, 10 \% 2) = \gcd(2, 0) \\
  &= 2
\end{align*}

The above can also be done in a loop:
\begin{Verbatim}[frame=single,fontsize=\footnotesize]
ALGORITHM: GCD (Euclidean algorithm)
INPUTS:    a, b
OUTPUT:    gcd(a, b)

while 1:
    if b == 0: 
        return a
    else:
        a, b = b, a % b
\end{Verbatim}


\begin{ex} 
  \label{ex:hill-10}
  \tinysidebar{\debug{exercises/{rsa-43/question.tex}}}
  What if you test for the condition $n = x^3 - y^3$?
  Is there are version of
  factorization using difference of cubes?
  %We have
  %\[
  %  x^3 - y^3 = (x - y)(x^2 + xy + y^2)
  %\]
  %Suppose $n = abc$.
  %\[
  %  \frac{}{}
  %\]

  \solutionlink{sol:hill-10}
  \qed
\end{ex} 
\begin{python0}
from solutions import *
add(label="ex:hill-10",
    srcfilename='exercises/hill-10/answer.tex') 
\end{python0}


\begin{ex} 
  \label{ex:hill-10}
  \tinysidebar{\debug{exercises/{rsa-43/question.tex}}}
  What if you test for the condition $n = x^3 - y^3$?
  Is there are version of
  factorization using difference of cubes?
  %We have
  %\[
  %  x^3 - y^3 = (x - y)(x^2 + xy + y^2)
  %\]
  %Suppose $n = abc$.
  %\[
  %  \frac{}{}
  %\]

  \solutionlink{sol:hill-10}
  \qed
\end{ex} 
\begin{python0}
from solutions import *
add(label="ex:hill-10",
    srcfilename='exercises/hill-10/answer.tex') 
\end{python0}


Despite the fact that the Euclidean algorithm is one of the fastest
algorithm to compute the GCD of two numbers and has been discovered
by
\href{https://en.wikipedia.org/wiki/Euclid}{Euclid} a long time ago (BC 300),
the actual runtime was not known until
\href{https://en.wikipedia.org/wiki/Gabriel_Lam%C3%A9}{Lam\'e}
proved in 1844 that the number of steps to compute
$\gcd(a, b)$ using the Euclidean algorithm
is $\leq 5$ times the number of digits (in base 10 notation)
of $\min(a, b)$.
For instance for the example above
of $\gcd(514, 24)$, the number of digits of $\min(514,24)$ is $2$.
Lam\'e theorem says that the number of steps made by the
Euclidean algorithm in the computation of $\gcd(514, 24)$ is
at most $5 \times 2 = 10$.
The actual number of steps in the earlier computation
\begin{align*}
  \gcd(514, 24)
  &= \gcd(24, 514 \% 24) = \gcd(24, 10) \\
  &= \gcd(10, 24 \% 10) = \gcd(10, 4) \\
  &= \gcd(10, 10 \% 4) = \gcd(10, 2) \\
  &= \gcd(2, 10 \% 2) = \gcd(2, 0) \\
  &= 2
\end{align*}
is 4 (not counting the base case step), i.e.,
\[
  \gcd(514, 24)
 = \gcd(24, 10) \\
 = \gcd(10, 4) \\
 = \gcd(10, 2) \\
 = \gcd(2, 0) \\
\]
Lam\'e's work is generally considered the beginning of
computational complexity theory, which is the study of
resources needed (time or space) to execute an algorithm.
Another fascinating fact about Lam\'e's theorem is that historically
the proof below is the first ever \lq\lq use" of the Fibonacci sequence.

\begin{thm} \textnormal{(Lam\'e 1844)}
  Let $a > b > 0$ be integers.
  If the GCD computation of $a,b$ 
  using Euclidean algorithm results in $n$ steps:
  \[
  \gcd(a_{n+1}, b_{n+1})
  = \gcd(a_{n}, b_{n})
  = \cdots
  = \gcd(a_1, b_1), \,\,\, b_1 = 0
  \]
  where $(a_{n+1}, b_{n+1}) = (a, b)$, and $a_i > b_i$, then
  \begin{enumerate}[nosep,label=\textnormal{(\alph*)}]
  \item  $a \geq F_{n+2}$ and
    $b \geq F_{n+1}$, where $F_n$ are the Fibonacci numbers
    ($F_1 = 1, F_2 = 1, F_3 = 2, F_4 = 3, F_5 = 5$, etc.
    Note that the index starts with $1$.)
  \item $n$ is at most 5 times the number of digits in $b$.
  \end{enumerate}
\end{thm}
\proof
TODO
\qed



\begin{prop}
  The number of digits of a positive integer $b$ is
  \[
  \floor{\log_{10} b + 1}
  \]
\end{prop}
\proof
TODO
\qed

On analyzing the proof, the above is in fact true for any base $> 1$.
In other words the number of base--$B$ symbols to represent $b$ is
\[
\floor{\log_{B} b + 1}
\]
where $B > 1$.
For instance the number of bits needed to represent $b$ is
\[
\floor{\log_{2} b + 1}
\]
For instance, $b = 9_{10} = 1001_2$ which has 4 bits and
\[
\floor{\log_{2} 9 + 1}
= \floor{3.1699... + 1}
= \floor{4.1699...}
= 4
\]

\begin{prop}
  Let positive integer $b$ be written in base $B$ where $B > 1$ is an integer.
  Then the number of base-$B$ symbols used to represent $b$ is
  \[
  \floor{\log_{B} b + 1}
  \]
\end{prop}
\proof
TODO


\begin{ex} 
  \label{ex:hill-10}
  \tinysidebar{\debug{exercises/{rsa-43/question.tex}}}
  What if you test for the condition $n = x^3 - y^3$?
  Is there are version of
  factorization using difference of cubes?
  %We have
  %\[
  %  x^3 - y^3 = (x - y)(x^2 + xy + y^2)
  %\]
  %Suppose $n = abc$.
  %\[
  %  \frac{}{}
  %\]

  \solutionlink{sol:hill-10}
  \qed
\end{ex} 
\begin{python0}
from solutions import *
add(label="ex:hill-10",
    srcfilename='exercises/hill-10/answer.tex') 
\end{python0}


\begin{ex} 
  \label{ex:hill-10}
  \tinysidebar{\debug{exercises/{rsa-43/question.tex}}}
  What if you test for the condition $n = x^3 - y^3$?
  Is there are version of
  factorization using difference of cubes?
  %We have
  %\[
  %  x^3 - y^3 = (x - y)(x^2 + xy + y^2)
  %\]
  %Suppose $n = abc$.
  %\[
  %  \frac{}{}
  %\]

  \solutionlink{sol:hill-10}
  \qed
\end{ex} 
\begin{python0}
from solutions import *
add(label="ex:hill-10",
    srcfilename='exercises/hill-10/answer.tex') 
\end{python0}


\begin{ex} 
  \label{ex:hill-10}
  \tinysidebar{\debug{exercises/{rsa-43/question.tex}}}
  What if you test for the condition $n = x^3 - y^3$?
  Is there are version of
  factorization using difference of cubes?
  %We have
  %\[
  %  x^3 - y^3 = (x - y)(x^2 + xy + y^2)
  %\]
  %Suppose $n = abc$.
  %\[
  %  \frac{}{}
  %\]

  \solutionlink{sol:hill-10}
  \qed
\end{ex} 
\begin{python0}
from solutions import *
add(label="ex:hill-10",
    srcfilename='exercises/hill-10/answer.tex') 
\end{python0}

