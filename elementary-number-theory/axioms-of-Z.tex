\sectionthree{Axioms of $\Z$}
\begin{python0}
from solutions import *; clear()
\end{python0}

We will assume that $(\Z, +, \cdot, 0, 1)$ satisfies the following
axioms.
\begin{enumerate}[nosep]
  \li \textsc{Properties of $+$:}
  \begin{enumerate}[nosep]
    \li Closure: If $x, y \in \Z$, then $x + y \in \Z$.
    \li Associativity: If $x, y, z \in \Z$, then $(x + y) + z = x + (y + z)$.
    \li Inverse: If $x \in \Z$, then there is some $y$ such that $x + y = 0 = y + x$.
      The $y$ in the above is an \defone{additive inverse} of $x$.
    \li Neutrality: If $x \in \Z$, then $0 + x = x = x + 0$.
    \li Commutativity: If $x, y \in \Z$, then $x + y = y + x$.
  \end{enumerate}
  (Memory aid for the first four: CAIN.)
  \li \textsc{Properties of $\cdot$:}
  \begin{enumerate}[nosep]
  \li Closure: If $x, y \in \Z$, then $x \cdot y \in \Z$.
  \li Associativity: If $x, y, z \in \Z$, then $(x \cdot y) \cdot z = x \cdot (y \cdot z)$.
  \li Neutrality: If $x \in \Z$, then $1 \cdot x = x = x + 1$.
  \li Commutativity: If $x, y \in \Z$, then $x \cdot y = y \cdot x$.
  \end{enumerate}
  It is common to write $xy$ instead of $x \cdot y$.

  \li \textsc{Distributivity}:
  If $x, y, z \in \Z$, then
  $x \cdot (y + z) = x \cdot y + x \cdot z$
  and
  $(y + z) \cdot x = y \cdot x + z \cdot x$
    
\end{enumerate}
(In the Inverse axiom above, we say that $y$ is \textit{an} additive inverse of
$x$.
Later we will show that $x$ has only one additive inverse.
Therefore we can say $y$ is \textit{the} additive invers of $x$.)
Also, it is conventional to write $xy$ for $x \cdot y$.

A set $R$ with operations $+_R, \cdot_R$ and elements $0_R, 1_R$ satisfying
the above properties is called a \defone{commutative ring}.
This is an important generalization because there are many
very useful commutative rings and we want to prove results about
commutative rings so that these results can be applied to all
commutative rings, including but not restricted to $\Z$.
If the commutativity of multiplication is left out, then we have the
concept of a \defone{non-commutative rings}.
This is also an important concept since $n \times n$ matrices with $\R$
entries, $\operatorname{M}_{n \times n}(\R)$, form a non-commutative ring.
In fact, more generally,
the set of $n\times n$ matrices with entries in a commutative ring $R$,
$\operatorname{M}_{n \times n}(R)$,
is itself a non-commutative ring.
By convention, \defone{ring} means commutative ring.
We will return to the concept of commutative and non-commutative rings later.

The next property of $\Z$, integrality,
is very special and does not apply to all communtative rings and is therefore
left out of the definition of a commutative ring:
\begin{enumerate}[nosep]
  \li \textsc{Integrality}:
  If $x, y \in \Z$, then
  $xy = 0 \implies x = 0$ or $y = 0$.
\end{enumerate}
In general a ring satisfying the integrality condition is called an
\defone{integral domain}.
Therefore $\Z$ is not just a commutative ring -- it is an integral domain.
Another property of $\Z$ that we will assume is
\begin{enumerate}[nosep]
  \li \textsc{Nontriviality}:
  $0 \neq 1$
\end{enumerate}
The nontriviality axiom of $\Z$ is extremely simple, but cannot be deduced from
the previous axioms.

The above forms the algebraic properties of $\Z$, i.e., properties involving
addition and multiplication.

It is actually possible to first define axioms for
$\N = \{0, 1, 2, ...\}$ and then define $\Z$ in terms of $\N$.
(See my Calculus notes.)
We will not do that except to mention that the
axioms for $\N$ are called the
\href{https://en.wikipedia.org/wiki/Giuseppe_Peano}{Peano}-\href{https://en.wikipedia.org/wiki/Richard_Dedekind}{Dedekind} axioms
and that one very important Peano-Dedekind axiom of $\N$
\begin{enumerate}[nosep]
  \li \textsc{Induction} of $\N$:
  If $X$ is a subset of $\N$ satisfying
  \begin{enumerate}[nosep]
  \li $0 \in X$
  \li If $n \in X$, then $n + 1 \in X$
  \end{enumerate}
  Then $X = \N$.
\end{enumerate}
It can be proven that as a consequence of the induction axiom of $\N$,
we have
\begin{enumerate}[nosep]
  \li \textsc{Well-ordering principle (WOP)} for $\N$:
  If $X$ is a nonempty subset of $\N$, then $X$ contains a
  \defone{least element},
  i.e., there is some $m \in X$ such that
  \[
  m \leq x
  \]
  for all $x \in X$.
\end{enumerate}
Without going into details, it can be shown that for $\N$, the WOP
is equivalent to each of the following axioms:
\begin{enumerate}[nosep]
  \li \textsc{Weak Mathematical Induction} for $\N$:
  Let $X$ be a subset of $\N$ satisyfing the following two conditions:
  \begin{enumerate}[nosep]
    \li $0 \in X$ and
    \li Let $n \in \N$. If $n \in X$, then $n + 1 \in X$.
  \end{enumerate}
  Then $X = \N$.
\end{enumerate}
and
\begin{enumerate}[nosep]
  \li \textsc{Strong Mathematical Induction} for $\N$:
  Let $X$ be a subset of $\N$ satisyfing the following two conditions:
  \begin{enumerate}[nosep]
    \li $0 \in X$ and
    \li Let $n \in \N$. If $k \in X$ for all $0 \leq k \leq n$,
    then $n + 1 \in X$.
  \end{enumerate}
  Then $X = \N$.
\end{enumerate}
  
In the above two induction properties, if we write $X = \{n \mid P(n)\}$ where
$P(n)$ is a propositional formula,
then the induction axioms can be rewritten in the
following way:

\begin{enumerate}[nosep]
  \li \textsc{Weak Mathematical Induction} for $\N$:
  Let $P(n)$ be a proposition for $n \in \N$ satisyfing the following two
  conditions:
  \begin{enumerate}[nosep]
    \li $P(0)$ is true and
    \li Let $n \in \N$. If $P(n)$ is true, then $P(n+1)$ is true.
  \end{enumerate}
  Then $P(n)$ is true for all $n \in \N$.
\end{enumerate}
and
\begin{enumerate}[nosep]
  \li \textsc{Strong Mathematical Induction} for $\N$:
  Let $P(n)$ be a proposition for $n \in \N$ satisyfing the following two
  conditions:
  \begin{enumerate}[nosep]
    \li $P(0)$ is true and
    \li Let $n \in \N$. If $k \in X$ for all $0 \leq k \leq n$,
    then $n + 1 \in X$.
    \li Let $n \in \N$. If $P(k)$ is true for $0 \leq k \leq n$,
    then $P(n+1)$ is true.
  \end{enumerate}
  Then $P(n)$ is true for all $n \in \N$.
\end{enumerate}

There are statements analogous to the above, but for $\Z$:

\begin{enumerate}[nosep]
  \li WOP for $\Z$:
  If $X$ is a nonempty subset of $\Z$ that is bounded below, then $X$ contains a
  \defone{least element},
  i.e., there is some $m \in X$ such that
  \[
  m \leq x
  \]
  for all $x \in X$.
  Here, $X$ is \defone{bounded below} means there is some $b \in \Z$ such that
  \[
  b \leq x
  \]
  for all $x \in X$.
  Note that $b$ in the above is in $\Z$ and need not be in $X$.
  For instance if $X = {-1, 1, 42}$, then the least element of $X$ is $-1$
  and $b = -3\in Z$ is a lower bound of $X$.
  However the least element of $X$ must be an element of $X$.
\end{enumerate}
\begin{enumerate}[nosep]
  \li \textsc{Weak Mathematical Induction} for $\Z$:
  Let $n_0 \in \Z$.
  Let $P(n)$ be a proposition for $n \in \Z$ and $n \geq n_0$
  satisyfing the following two conditions:
  \begin{enumerate}[nosep]
    \li $P(n_0)$ is true and
    \li Let $n \in \Z$ with $n \geq n_0$.
    If $P(n)$ is true, then $P(n+1)$ is true.
  \end{enumerate}
  Then $P(n)$ is true for all $n \in \Z$, $n \geq n_0$.
\end{enumerate}
\begin{enumerate}[nosep]
  \li \textsc{Strong Mathematical Induction} for $\Z$:
  Let $P(n)$ be a proposition for $n \in \Z, n \geq n_0 \in Z$
  satisyfing the following two
  conditions:
  \begin{enumerate}[nosep]
    \li $P(n_0)$ is true and
    \li Let $n \in \Z$. If $k \in X$ for all $0 \leq k \leq n$,
    then $n + 1 \in X$.
    \li Let $n \in \Z$. If $P(k)$ is true for $0 \leq k \leq n$,
    then $P(n+1)$ is true.
  \end{enumerate}
  Then $P(n)$ is true for all $n \in \Z$, $n \geq n_0$.
\end{enumerate}

It can be shown that from the induction axiom of $\N$,
we can prove that the WOP of $\N$, the weak induction of $\N$, and the
strong induction of $\N$ hold.
Furthermore, these implies the WOP for $\Z$, the weak induction of $\Z$,
and the strong induction of $\Z$ holds.

Note that
obvious statement of \lq\lq WOP for $\R$" does not hold.
For instance the open interval $X = (0, 1)$ is bounded below
(for instance by $-42$).
However there is no $m$ in $X$ such that $m \leq x$ for all $x$ in $X$.
For instance $m = 0.01 \in X$ is not a least element of $X$
since $0.0001 \in X$ is smaller than $m$.
Also, $m = 0.0000001 \in X$ is also not a least element of $X$ since
$0.0000000001 \in X$ is less than $m$.
In fact for any $m \in X$, $(1/2)m$ is in $X$ and is less than $m$.
In other words no value in $X$ can be a minimum value of $X$.

(See end of this section for exercises on WOP and induction.)

The above are the algebraic and inductive-type axioms of $\Z$.
There are also the order relations such as $<$ and $\leq$
of $\Z$.
Technically speaking, the order relation should come before
the inductive axiom, WOP, and induction (weak and strong) since they use
order relations.

The following are the order axioms of $\Z$.
We assume there is a subset $\Z^+$ \index{$\Z^+$}\tinysidebar{$\Z^+$}
of $\Z$ satisfying the following:
\begin{enumerate}[nosep]
  \li \textsc{Trichotomy for $\Z$}:
  If $x \in \Z$, then exactly one of the following holds:
  \[
  -x \in \Z^+,  \,\,\, x = 0, \,\,\, x \in \Z^+
  \]
  \li \textsc{Closure of $+$ for $\Z^+$}:
  If $x,y \in \Z^+$, then $x + y \in \Z^+$.
  \li \textsc{Closure of $\cdot$ for $\Z^+$}: If $x,y \in \Z^+$, then $x \cdot y \in \Z^+$.
\end{enumerate}
With the above, we define
$<$ \index{$<$}\tinysidebar{$<$} on $\Z$
as follows:
If $x, y \in \Z$, then we write $x < y$ if
\[
y - x \in \Z^+
\]
i.e., there is some $z \in \Z^+$ such that
\[
y - x = z
\]
i.e.,
\[
y = x + z
\]
Since $<$ is defined (on $\Z$), we can define
$x \leq y$ (on $\Z$) \index{$\leq$}\tinysidebar{$\leq$}
to mean \lq\lq either $x < y$ or $x = y$".
The above order relation is expressed abstractly without referring to the fact
that
$\Z^+$ is $\{1, 2, 3, ...\}$, i.e., the set of positive integers.
In fact, you can prove $\Z^+ = \{1, 2, 3, ...\}$ from the above axioms.
Of course you define $x > y$ to be the same as $y < x$ and
$x \geq y$ to be $y \leq x$.
Also, we say $x \in \Z$ is \defone{positive} is $x > 0$.

The three axioms in the order relation on $\Z$ can be derived from the order
relation axioms of $\N$.
The order relation on $\N$ is defined as follows:
For $\N = \{0, 1, 2, 3, ...\}$, we simply define
\[
x \geq 0
\]
for all $x \in \N$.
For $x, y \in \N$, we define
\[
x \geq y
\]
if there is some $z \in \N$ such that
\[
x = y + z
\]
Finally $x > y$ is defined to be $x \geq y$ and $x \neq y$.
If $x > 0$, we say that $x$ is \defone{positive}.
Just like positivity in $\Z$,
if $x \in \N$, we say that $x$ if \defone{positive} if $x > 0$.

Also, we will define $|x|$ in the usual way:
\[
|x| =
\begin{cases}
x  & \text{if } x \geq 0 \\
-x & \text{otherwise}
\end{cases}
\]

There is one more axiom of $\Z$ that is related to the
\lq\lq topology" of $\Z$ and uses the order relation:

\begin{enumerate}[nosep]
  \li \textsc{Topology} for $\Z$:
  Given any $x \in \Z$, there is no $y \in \Z$ such that
  \[
  x < y < x + 1
  \]
\end{enumerate}

This can derived from this:

\begin{enumerate}[nosep]
  \li \textsc{Topology} for $\N$:
  There is no $x \in \N$ such that
  \[
  0 < x < 1
  \]
\end{enumerate}

You can think of topology of a set as study of \lq\lq closeness" of values
in that set.
For $\Q$, given any two distinct
rational values $x < y$, there is also some $z \in \Q$ such that
$x < z < y$.
This is the same for $\R$.
Therefore the topology of $\Z$ is very different from the topology
of $\Q$ and $\R$ because there are \lq\lq holes" in $\Z$ where
there are no $\Z$ values.
$\Z$ has what is called a \defone{discrete topology}.
On the other hand $\Q$ and $\R$ are \lq\lq dense".
(I won't go further into topology since we won't need it.
See my notes on topology.)
The above assumes the existence of an order relation of $\Z$, i.e., $<$.

Now let's prove some algebraic facts about $\Z$ that can be deduced
from the fact that $\Z$ is a commutative ring.

\begin{prop}
\label{prop:Z-uniqueness-of-additive-inverse}
\textnormal{(\defone{Uniqueness of additive inverse})}.
  The additive inverse for $x$ is unique.
  In other words, if $y, y'$ satisfy
  \begin{align*}
    x + y &= 0 = y + x \tag{$1$}\\
    x + y' &= 0 = y' + x \tag{$2$}
  \end{align*}
  then $y = y'$.
\end{prop}
\proof
\begin{align*}
y &= 0 + y        & & \text{by Identity of $+$} \\
  &= (y' + x) + y & & \text{by RHS of $(2)$} \\
  &= y' + (x + y) & & \text{by Associativity of $+$} \\
  &= y' + 0       & & \text{by LHS of $(1)$} \\
  &= y'           & & \text{by Neutrality of $+$}
\end{align*}
\qed

Since the additive inverse of $x$ is unique, we can choose to write
the additive inverse of $x$ in terms of $x$.
This is usually written $-x$.
(Had it been the case that the additive inverse of $x$ is not unique,
we would have to denote the additive inverses of $x$ by
$(-x)_1$, $(-x)_2$, etc.)

We define the operator $-$ in terms of the additive inverse:

\begin{defn}
  Let $x,y \in \Z$. We define the subtraction operator as
  \[
  x - y = x + (-y)
  \]
\end{defn}

Note that every value in $\Z$ has an additive inverse, but we do not
require values of $\Z$ to have multiplicative inverses:

\begin{defn}
  Let $x \in \Z$. Then $y$ is a \defone{multiplicative inverse} of $x$ if
  \[
  x\cdot y = 1 = y\cdot x
  \]
  We say that $x$ is a \defone{unit} if $x$ has a multiplicative inverse.
  We can also say that $x$ is (multiplicatively) \defone{invertible}.
\end{defn}

Intuitively, you know that the only values of $\Z$ with multiplicative inverses
are $1$ and $-1$.
We will prove this later.

\begin{prop}
\label{prop:Z-uniqueness-of-multiplicative-inverse}
\textnormal{(\defone{Uniqueness of multiplicative inverse})}.
  Let $x \in \Z$.
  If $x$ is a unit, then the multiplicative inverse of $x$ is unique.
  In other words if $y,y'$ satisfy
  \begin{align*}
    x  y &= 1 = y x \tag{1}\\
    x  y' &= 1 = y' x \tag{2}
  \end{align*}
  then $y = y'$.
\end{prop}
\proof
\begin{align*}
y &= 1 \cdot y            & & \text{by Identity axiom of $\cdot$} \\
  &= (y' \cdot x) \cdot y & & \text{by RHS of (1)} \\
  &= y' \cdot (x \cdot y) & & \text{by Associativity of $\cdot$} \\
  &= y' \cdot 1           & & \text{by LHS of (2)} \\
  &= y'                   & & \text{by Neutrality of $\cdot$ }
\end{align*}
\qed

\begin{defn}
  The multiplicative inverse of $x$, if is exists, is denoted by $x^{-1}$.
  The set of all units of $\Z$ is denoted by either $U(\Z)$ or $\Z^\times$.
  We will show later that $U(\Z) = \{-1, 1\}$.
\end{defn}

\begin{prop}
\label{prop:Z-cancellation-law-for-addition}
  \textnormal{(\defone{Cancellation law for addition})}.
  Let $x, y, z \in \Z$.
  \begin{enumerate}[nosep,label=\textnormal{(\alph*)}]
    \item If $x + z = y + z$, then $x = y$.
    \item If $z + x = z + y$, then $x = y$.
  \end{enumerate}
\end{prop}
\proof
TODO
\qed

\begin{prop}
\label{prop:Z-properties-of-zero}
  Let $x \in \Z$.
  \begin{enumerate}[nosep,label=\textnormal{(\alph*)}]
  \item $0x = 0 = x0$
  \item $-0 = 0$
  \item $x - 0 = x$. 
  \end{enumerate}
\end{prop}
\proof
(a) 
We will first prove $0x = 0$:
\begin{align*}
  0x &= (0 + 0)x     & & \text{by Neutrality of $+$} \\
     &= 0x + 0x      & & \text{by Distributivity} \\
\THEREFORE  0 + 0x &= 0x + 0x  & & \text{by Neutrality of $+$} \\
  0  &= 0x       & & \text{by Cancellation Law of Addition} 
\end{align*}
To prove $0 = x0$, from above
\begin{align*}
  0 &= 0x \\
    &= x0 & & \text{ by Commutativity of $\cdot$}
\end{align*}

(b)
TODO

(c)
TODO
\qed

\begin{prop}
\label{prop:Z-properties-of-negative}
  Let $x, y, c \in \Z$.
  \begin{enumerate}[nosep,label=\textnormal{(\alph*)}]
  \item $-(-1) = 1$
  \item $-(-x) = x$
  \item $x(-1) = -x = (-1)x$
  \item $(-1)(-1) = 1$
  \item $(-x)(-y) = xy$
  \item $-(x + y) = -x + -y$
  \item $-(x - y) = -x + y$
  \end{enumerate}
\end{prop}
\proof
TODO
\qed

\begin{prop}
\label{prop:Z-cancellation-law-for-multiplication}
  \textnormal{(\defone{Cancellation law for multiplication})}.
  Let $x, y, z \in \Z$.
  \begin{enumerate}[nosep,label=\textnormal{(\alph*)}]
    \item If $xz = yz$ and $z \neq 0$, then $x = y$.
    \item If $zx = zy$ and $z \neq 0$, then $x = y$.
  \end{enumerate}
\end{prop}
\proof
TODO
\qed

For general expressions involving $n$ terms, instead of writing
$x_1 + \cdots + x_n$, we formally define
\index{$\sum_{i=1}^n$}\tinysidebar{$\sum_{i=1}^n$}
\[
\sum_{i=1}^n x_i =
\begin{cases}
0 & \text{if } n = 0 \\
\displaystyle\sum_{i=1}^{n-1} x_i + x_n & \text{if } n > 0
\end{cases}
\]
Note that the definition above implies that our summation
is left associative, i.e.,
\begin{align*}
\sum_{i=1}^{3} x_i
&= \sum_{i=1}^{2} x_i + x_3  \\
&= \left( \sum_{i=1}^{1} x_i + x_2 \right) + x_3 \\
&= \left( \left( \sum_{i=1}^{0} x_i + x_1  \right) + x_2 \right) + x_3 \\
&= \left( \left( 0 + x_1  \right) + x_2 \right) + x_3 \\
&= \left( x_1 + x_2 \right) + x_3
\end{align*}
This conforms with the standard practice.
Likewise, we define \index{$\prod_{i=1}^n$}\tinysidebar{$\prod_{i=1}^n$}
\[
\prod_{i=1}^n x_i =
\begin{cases}
1 & \text{if } n = 0 \\
\displaystyle\prod_{i=1}^{n-1} x_i + x_n & \text{if } n > 0
\end{cases}
\]

\begin{prop} \textnormal{(\defone{General associativity})}
Let $x_1, \ldots, x_n \in \Z$.
\begin{enumerate}[nosep,label=\textnormal{(\alph*)}]
\item Different fully parenthesized expressions of
$x_1 + \cdots + x_n$ evaluates to the same value.
\item Different fully parenthesized expressions of
$x_1 \cdot \cdots \cdot x_n$ evaluates to the same value.
\end{enumerate}
\end{prop}
\proof
The proof for both statements are similar.
We will only prove (b).

TODO
\qed

Now for the case when the summation or product involves the same value,
we have the following.
For convenience, I will write $x^2 = xx$ and in general \index{$()^n$}\tinysidebar{$()^n$}
\[
x^n =
\begin{cases}
  1       &\text{if } n = 0 \\
  x^{n-1}x &\text{if } n > 0 
\end{cases}
\]
Note that
\[
x^3 = x^2 x = (x^1x) x = ((x^0x)x)x = ((1x)x)x = (xx)x
\]
i.e., the above recursive definition of
$x^n$ is left recursive.
If $x$ has a multiplicative inverse, i.e., if $x^{-1}$ exists,
then, for $n \geq 0$, I will define
\[
x^{-n} = \left( x^{-1} \right)^n
\]

Note that $nx$ has two meanings: $nx$ can be the multiplication of $n$
and $x$ and it can also be $x + \cdots + x$ where the expression
contains $n$ copies of $x$.
Of course you would expect them to be the same.
For now define
\[
  [n]x =
  \begin{cases}
    0       &\text{if } n = 0 \\
    [n-1]x + x &\text{if } n > 0 
  \end{cases}
\]
$[n]x$ is just a notation that is easier to write than $\sum_{i=1}^n x$.
And if $n$ is negative, we define
\[
[n]x = -([-n]x)
\]
I'll write $[\bullet]$ for the function $\Z \times \Z \rightarrow \Z$
as defined above.

\begin{prop}
  Let $x \in \Z$.
  Then $[n]x = n \cdot x$, i.e.,
  \lq\lq add $n$ copies of $x$" is the same as \lq\lq multiply $n$ and $x$".
\end{prop}
\proof
We will prove two proofs, using weak induction and WOP.

First we proof the above for $n \geq 0$ using weak induction.
For $n \geq 0$, let $P(n)$ be the statement that $[n]x = n \cdot x$.

\textsc{Base case:}
We now prove $P(0)$ holds.
By definition, $[0]x = 0$.
By Proposition \ref{prop:Z-properties-of-zero}
(page \pageref{prop:Z-properties-of-zero}), $0 \cdot x = 0$.
Hence $[0]x = 0 \cdot x$, i.e., $P(0)$ holds.

\textsc{Inductive case:}
Now assume $P(n)$ holds, i.e, $[n]x = n \cdot x$.
We will prove that $P(n + 1)$ holds.
We have
\begin{align*}
[n + 1]x &= [n]x + x        & & \text{by definition of $[\bullet]$} \\
         &= n \cdot x + x   & & \text{by inductive hypothesis $P(n)$} \\
         &= n \cdot x + 1 \cdot x & & \text{by Neutrality of $\cdot$} \\
         &= (n + 1)\cdot x  & & \text{by Distributivity} 
\end{align*}
Hence $P(n + 1)$ holds.

Therefore by weak induction $P(n)$ holds for all integers $n \geq 0$.

We now prove $P(n)$ holds for integers $n < 0$:
\begin{align*}
[n]x &= -([-n]x)        & & \text{by definition of $[ \bullet ]$} \\
     &= -((-n) \cdot x) & & \text{by $P(-n)$} \\
     &= (-1)\cdot ((-n) \cdot x) & & \text{by Proposition \ref{prop:Z-properties-of-negative}(c) (page \pageref{prop:Z-properties-of-negative})} \\
     &= ((-1)\cdot (-n)) \cdot x) & & \text{by Associativity of $\cdot$} \\
     &= (1\cdot n) \cdot x & & \text{\text{by Proposition \ref{prop:Z-properties-of-negative}(e) (page \pageref{prop:Z-properties-of-negative})}} \\
     &= n \cdot x       & & \text{by Neutrality of $\cdot$}
\end{align*}

Next, we proof the above for $n \geq 0$ using WOP.

TODO
\qed

Now let us look at order relations on $\Z$.

\begin{prop}
Let $a, b, c, d \in \Z$ with $x > 0$.
\begin{enumerate}[nosep,label=\textnormal{(\alph*)}]
\item If $a < b$, then $a + c < b + c$. 
\item If $a < b$, then $ax < bx$.
\item If $a < b$, then $a(-x) > b(-x)$.
\item $a < a + c$
\item $a \leq ac$
\end{enumerate}
There are analogous statements for (a)-(c) where $<$ and $>$
are replaced by $\leq$ and $\geq$.
\end{prop}
\proof
(a)
Note that
\begin{align*}
(b + c) - (a + c)
&= (b + c) + (-(a + c)) & & \text{by definition of $-$}\\ 
&= b + (c + (-(a + c))) & & \text{by Associativity of $+$}\\
&= b + (c + (-a + -c))  & & \text{by Proposition \ref{prop:Z-properties-of-negative} (page \pageref{prop:Z-properties-of-negative})}\\
&= b + (c + (-c + -a))  & & \text{by Commutativity of $+$}\\
&= b + ((c + -c) + -a)  & & \text{by Associativty of $+$}\\
&= b + (0 + -a))        & & \text{by Inverse of $+$}\\
&= b + -a               & & \text{by Neutrality of $+$}\\
&= b - a                & & \text{by definition of $-$} \tag{1}
\end{align*}
We have
\begin{align*}
           & a < b \\
\THEREFORE & b - a \in \Z^+                     & & \text{by definition of $<$} \\
\THEREFORE & (b + c) - (a + c) = b - a \in \Z^+ & & \text{by (1)}\\
\THEREFORE & b + c > a + c                      & & \text{by definition of $<$}
\end{align*}
\qed

\begin{prop}
$1 \in \Z^+$, i.e., $1$ is positive.
\end{prop}
\proof
TODO
\qed

\begin{prop}
$\Z^+ = \{1, 2, 3, ...\} = \N - \{0\}$.
\end{prop}
\proof
TODO
\qed

\begin{prop}
The only units of $\Z$ are $-1, 1$, $U(\Z) = \{-1, 1\}$.
\end{prop}
\proof
TODO
\qed

The following are examples of proofs using WOP and induction.

\begin{ex}
For $n \in \N$,
\[
1 + 2 + \cdots + n = \frac{n(n+1)}{2}
\]
Here, when $n = 0$, $1 + 2 + \cdot + n$ is defined to be $0$.
\begin{enumerate}[nosep]
\item[(a)] Prove the above using weak induction.
\item[(b)] Prove the above using WOP.
\end{enumerate}
\end{ex}
\proof
TODO
\qed


\begin{ex}
For $n \in \N$,
\[
1^2 + 2^2 + \cdots + n^2 = \frac{1}{6}n(n+1)(2n + 1)
\]
Here, when $n = 0$, $1 + 2 + \cdot + n$ is defined to be $0$.
\begin{enumerate}[nosep]
\item[(a)] Prove the above using weak induction.
\item[(b)] Prove the above using WOP.
\end{enumerate}
\end{ex}
\proof
TODO
\qed


\begin{ex}
What is wrong with this proof?
I claim that every man wears brown pants.
I will prove this by weak induction.
Let $P(n)$ be the statement that every man in a set of $n$
men wears brown pants.
Clearly $P(0)$ is true.
For, if $P(0)$ does not hold, then there is a man in an empty set
that does not wear brown pants.
But an empty set cannot contain a man.
Now, let $n \geq 0$ and assume $P(n)$ hold.
I will prove that $P(n + 1)$ holds.
Let $X$ be a set with $n + 1$ men.
Let $Y$ be a subset of $X$ with exactly $n$ men.
Since there are $n$ men in $Y$, by induction hypothesis,
every men in $Y$ wears brown pants.
Of course $Y$ misses one man from $X$, say John.
Now form another subset $Z$ of $X$ with exactly $n$ men that contains John.
Again by induction hypothesis $P(n)$ hold and therefore every men in $Z$
wears brown pants, including John.
Since $X$ is made up of men in $Y$ together with John, we have shown
that every men in $X$ wears brown pants.
Therefore by weak induction, every men wears brown pants.

(Of course the above cannot possible be true.
In fact: you can change \lq\lq every man wears brown pants" to
\lq\lq every man wears the same pair of pants".)
\end{ex}
TODO


\begin{ex}
What is wrong with this proof?
I claim that every man has the same favorite number.
I will prove this by weak induction.
Let $P(n)$ be the statement that every man in a set of $n$
men has the same favorite number as the rest.
Clearly $P(0)$ is true vacuously like the previous exercise.
Now, let $n \geq 0$ and assume $P(n)$ hold.
I will prove that $P(n + 1)$ holds.
Let $X$ be a set of $n + 1$ men.
Line the men in $X$ in a line.
The set $Y$ of the first $n$ men of $X$ must all share the same favorite number
since there are $n$ men in $Y$.
Let $Z$ be the set of $n$ men of $X$ that excludes the first man.
All the men in $Z$ must have the same favorite number as well.
Clearly every men in $Y$ has the same favorite number as the man in
the middle of the line and every men in $Z$ also has the same favorite number
as this man in the middle.
Therefore all men in $X$ have the same favorite number.
Hence every men shared the same favorite number.
\end{ex}
TODO

\begin{ex}
What is wrong with this proof?
I claim that every day is Monday.
I will prove this by strong induction.
Let $P(n)$ be the statement that every day in a set of $n$
days is Monday.
Clearly $P(0)$ is true.
For, if $P(0)$ does not hold, then there is a day in an empty set
that is not Monday.
But an empty set cannot have a day.
Now, let $n \geq 0$ and assume $P(0), P(1), ..., P(n)$ hold.
I will prove that $P(n + 1)$ holds.
Let $X$ be a set with $n + 1$ days.
Let $Y$ be a subset of $X$ with exactly $n$ days.
Since there are $n$ days in $Y$, by induction hypothesis,
every day in $Y$ in Monday.
$Y$ misses one day in $X$, say $x$.
Also by $P(1)$, since $\{x\}$ is a set of size $1$, $x$ is also Monday. 
Since $X = Y \cup \{x\}$, we have shown
that every day in $X$ is Monday,
Therefore by strong induction, every day is Monday.
\end{ex}
TODO

\begin{ex}
What is wrong with this proof?
I will prove by strong induction that $e^n = 1$ for all $n \geq 0$.
Let $P(n)$ be the statement that $e^n = 1$.
Clearly $e^0 = 1$.
Therefore $P(0)$ holds.
Now assume $P(0), P(1), ..., P(n)$ holds.
Since $P(n)$ and $P(1)$ holds,
we have $e^{n + 1} = e^n \cdot e^1 = 1 \cdot 1 = 1$. 
Hence $P(n + 1)$ holds.
Therefore, by strong induction, $P(n)$ holds for all $n$, i.e.,
$e^n = 1$ for all $n \geq 0$.
\end{ex}
TODO

\begin{ex}
What is wrong with this proof?
I will prove by strong induction that given $x \in \R$,
$\sum_{i = 1}^n x = 0$ for $n \geq 0$.
Let $P(n)$ be the above statement.
$P(0)$ holds since $\sum_{i = 1}^0 x$ is a sum with no terms and by
convention is $0$.
Now assume $P(0), P(1), ..., P(n)$ holds.
Let $n + 1 = a + b$ with both $a,b$ approximately equal.
(For instance if $n + 1$ is even, $a = b = (n + 1)/2$.
And if $n + 1$ is odd, $a = \floor{(n + 1)/2}$ and $b = n + 1 - a$.)
Since $P(a)$ and $P(b)$ holds, 
$\sum_{i = 1}^a x = 0$
and
$\sum_{i = 1}^b x = 0$.
Clearly
$\sum_{i = 1}^{n + 1} x = \sum_{i = 1}^{a} x + \sum_{i = 1}^{b} x$
since the LHS is a sum of $n + 1$ $x$'s and there are
$a + b$ $x$'s on the RHS.
Hence
\[
\sum_{i = 1}^{n + 1} x = \sum_{i = 1}^{a} x + \sum_{i = 1}^{b} x = 0 + 0 = 0
\]
Therefore $P(n + 1)$ holds.
By strong induction $P(n)$ holds for all $n \geq 0$, i.e.,
$\sum_{i = 1}^n x = 0$ for $n \geq 0$.
\end{ex}
TODO
