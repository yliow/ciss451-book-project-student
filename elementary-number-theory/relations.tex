\sectionthree{Relations}
\begin{python0}
from solutions import *; clear()
\end{python0}

Let $X, Y$ be two set.
A \defone{relation} $R$ on $X,Y$ is a subset of $X \times Y$.
For instance for $X = \{0, 1, 2\}$ and $Y = \{a, b\}$,
\[
R = \{(0, b), (2, a), (2, b)\}
\]
is a relation on $X \times Y$.
Instead of writing
\[
(2, a) \in R
\]
and say \lq\lq $(2, a)$ is an $R$",
I can also write
\[
2 R a
\]
and say
\lq\lq"$2$ is related to $a$ (for relation $R$)".
Note that $(0, a) \not\in R$.
In this case, I will also say \lq\lq $0$ is not related to $a$" (through $R$)
and will write
\[
0 \not \mathrel{R} a
\]
It's convenient to draw a relation as a directed graph.
For instance for the above relation
\[
R = \{(0, b), (2, a), (2, b)\}
\]
I can draw
\begin{python}
from latextool_basic import *
p = Plot()
def node(x, y, s):
    return Graph.node(x=x, y=y, r=0.25, label=r'$%s$' % s, name=s)

p += node(x=0, y=0,  s='0')
p += node(x=0, y=-1, s='1')
p += node(x=0, y=-2, s='2')
p += node(x=2, y=0,  s='a')
p += node(x=2, y=-1, s='b')

for (a,b) in ["0b", "2a", "2b"]:
    p += Graph.arc(names=[a,b])

print(p)
\end{python}
You can represent $R$ as a 2D array (matrix):
\begin{python}
from latextool_basic import *
p = Plot()
m = [[0,1],
     [0,0],
     [1,1],
]
table2(p, m, width=0.7, height=0.7,
rownames=[0,1,2],
colnames=['$a$', '$b$'],
rowlabel=None, collabel=None)
print(p)
\end{python}
This version of the $R$ is more suitable for computations
with a program, especially after the column names $a$,$b$ are changed to
$0$, $1$.
A $0$ in the matrix at row $r$, column $c$ means \lq\lq not related"
while a $1$ means \lq\lq is related".

If $X = Y$, then
instead of saying
\lq\lq $R$ is a relation on $X$ and $X$",
I will just say \lq\lq $R$ is a relation on $X$".

For instance $\equiv \pmod{N}$ is a relation on $\Z$.
This is (of course) defined by
\[
a \equiv b \pmod{N} \text{ if } N \mid a - b
\]

\begin{defn}
Let $R$ be a relation on set $X$.
\begin{myenum}
\item $R$ is \defone{reflexive} if for all $x \in X$, we have $x R x$.
\item $R$ is \defone{symmetric} if for all $x, y \in X$, if $x R y$, then $y R x$.
\item $R$ is \defone{transitive} if for all $x, y,z \in X$, if $x R y$ and $y R z$, then $x R z$.
\end{myenum}
$R$ is said to be \defone{transitive} if $R$ is reflexive, symmetric and transitive.
\end{defn}

The following are examples of relations.

\begin{eg}
  For a general set $X$ you have the \lq\lq $=$" and \lq\lq $\neq$" relation.
  A function $f:X \rightarrow Y$ can be viewed as a relation since $f$
  gives us the set
  \[
  \{ (x, f(x)) \mid x \in X \}
  \]
  This set is usually called the \defone{function graph} of $f$.
\end{eg}

\begin{eg}
  For a set of sets (for instance a powerset), besides $\lq\lq ="$
  and \lq\lq$\neq$", you also have the following relations: 
  \begin{myenum}
    \item  $\subseteq$, $\subsetneq$
    \item \lq\lq intersects (nontrivally)".
      $X$ intersects $Y$ means $X \cap Y \neq \emptyset$.
    \item \lq\lq disjoint from" where
    $X$ is disjoint from $Y$ means $X \cap Y = \emptyset$.
    \item \lq\lq same cardinality" where
    $|X| = |Y|$ means there is a bijection $X \rightarrow Y$.
  \end{myenum}
\end{eg}

\begin{eg}
  Besides \lq\lq $=$" and \lq\lq $\neq$",
  here are some relations on $\R$:
  \begin{myenum}
  \item the \lq\lq $<$" and the \lq\lq $\leq$" relation.
  \item the \lq\lq $>$" and the \lq\lq $\geq$" relation.
  \item \lq\lq less than 1 apart" is a relation, i.e., for $x,y \in \R$,
    define $x R y$ if $|x - y| < 1$.
    Of course changing \lq\lq $1$" in the above to another fixed real number
    also
    gives you a relation.
  \item \lq\lq Has the same fractional part as" is a relation.
    For instance $1.25$ is related to $7.25$ since they both have fractional
    part of $0.25$.
  \item \lq\lq Has the same integer part as" is also a relation. 
  \end{myenum}
\end{eg}

\begin{eg}
On $\Z$ here are some relations:
\begin{myenum}
\item Divisibility relation 
  $a \mid b$ if $ax = b$ for some $x$.
\item For a fixed $N > 0$, congruence relation mod $N > 0$, i.e.,
  $a \equiv b \pmod{N}$ if $N \mid a - b$.
\item
  \lq\lq has same number of distinct prime factors" is a relation.
  For instance $100$ has as many prime factors (i.e., $2$) as $35$.
\item
  Fix a prime $p$.
  Then \lq\lq has the same highest power of $p$ factor" is a relation.
  For instance fixing $p = 5$, $75 = 3 \cdot 5^2$ is related to
  $3502 \cdot 5^2 \cdot 7$.
  Sometimes the highest power of $p$ dividing $n$ is
  denoted by $v_p(n)$.
\end{myenum}
\end{eg}

\begin{eg}
  On the set of strings,
  \begin{myenum}
  \item  \lq\lq has the same length" is a relation.
    For instance \verb!cat! has the same length as \verb!pie!.
  \item
    For bitstrings, \lq\lq same number of 1s" is a relation.
  \item
    For bitstrings, \lq\lq has the same parity for the number of 1s"
    is a relation.
    For instance $10010101000$ and $00001101111$ are related
    since both have an even number of $1$s.
  \end{myenum}
\end{eg}

\begin{eg}
  From basic geometry you have
  \begin{myenum}
    \item
      The \lq\lq is parallel to" is a relation on the set of lines in $\R^2$.
    \item
      The \lq\lq is similar to" is a relation on the set of triangles in $\R^2$.
    \item
      The \lq\lq is congruent to" is a relation on the set of triangles in
      $\R^2$.
  \end{myenum}
\end{eg}

\begin{eg}
  From graph theory, you have the following examples:
  \begin{myenum}
  \item \lq\lq Reachability" is relation on the vertex set of a graph:
    $v R v'$ if there is a path from $v$ to $v'$.
  \item \lq\lq is isomorphic to" is a relation on graphs.
  \item \lq\lq is subgraph of" is a relation on graphs.
  \item \lq\lq is homeomorphic to" is a relation on graphs.
  \end{myenum}
\end{eg}



\begin{defn}
  Let $R$ be a relation on $X$.
  Let $a \in X$.
  The \defone{left relation class} of $a$ is
  \[
  aR = aR\bullet = \{x \in X \mid aRx\}
  \]
  The \defone{right relation class} of $a$ is
  \[
  Ra = \bullet Ra = \{x \in X \mid xRa\}
  \]
  Let $X/R$ denote the set of left relation classes of $X$:
  \[
  X/R = \{aR \mid a \in X \}
  \]
  and $R \backslash X$ denote the set of right relation classes of $X$:
  \[
  R\backslash X = \{Ra \mid a \in X \}
  \]
\end{defn}


\begin{ex} 
  \label{ex:rsa-10}
  \tinysidebar{\debug{exercises/{nt-61/question.tex}}}
  In your \verb!Zmod.py!, complete the following methods:
  \begin{enumerate}[nosep]
    \li multiplicative inverse mod $N$ (i.e., \texttt{inv})
    \li invertibility mod $N$ (i.e., \texttt{is\_invertible})
    \li division (i.e., \texttt{\_\_div\_\_})
  \end{enumerate}

  \solutionlink{sol:rsa-10}
  \qed
\end{ex} 
\begin{python0}
from solutions import *
add(label="ex:rsa-10",
    srcfilename='exercises/rsa-10/answer.tex') 
\end{python0}


\begin{prop}
  Let $R$ is an equivalence relation on $X$.
  Let $a,a'\in X$.
  \begin{myenum}
  \item $aR = Ra$
  \item $a \in aR$
  \item $aR = a'R$ iff $aRa'$
  \item $aR \cap a'R = \emptyset$ iff $a \not \mathrel{R} a'$
  \end{myenum}
\end{prop}
\proof
(a) We have
\begin{align*}
  x \in aR
  &\implies aRx \\
  &\implies xRa & & \text{because $R$ is symmetric} \\
  &\implies x \in Ra
\end{align*}
Hence $aR \subseteq Ra$.
Also,
\begin{align*}
  x \in Ra
  &\implies xRa \\
  &\implies aRx & & \text{because $R$ is symmetric} \\
  &\implies x \in aR
\end{align*}
Hence $Ra \subseteq Ra$.
Therefore $aR = Ra$.

(b) $aRa$ since $R$ is reflexive. Hence $a \in aR$ and $a \in Ra$.

(c)
We have
\begin{align*}
  aR = a'R
  &\implies a \in aR = a'R & & \text{by (a)} \\
  &\implies a \in a'R \\
  &\implies a'Ra
\end{align*}
Now assume $a'Ra$.
Since $R$ is symmetric, we also have $aRa'$.
\begin{align*}
  x \in Ra
  &\implies x R a \\
  &\implies x R a' & & \text{by $xRa, aRa'$ and transitivity}\\  
  &\implies x \in Ra'
\end{align*}
Hence $Ra \subseteq Ra'$.
Also,
\begin{align*}
  x \in Ra'
  &\implies xRa' \\
  &\implies xRa & & \text{by $xRa', a'Ra$ and transitivity}\\
  &\implies x \in Ra
\end{align*}
Hence $Ra'\subseteq Ra$.
Therefore $Ra = Ra'$.
\qed

From (b) above,
a left equivalence class is a right equivalence class
and vice versa. 
Therefore for equivalence relation, we only need the term
\defone{equivalence class} (no left or right).
In this case $aR$ and $Ra$ is denoted by
$[a]_R$\index{$[]_R$@equivalence class}\tinysidebar{$[]_R$, $[]$}
or simply
$[a]$ if $R$ is understood.

\begin{prop}
  \mbox{}
  \begin{myenum}
  \item
    If $R$ is an equivalence relation, then there is a partition
    of $X$ by equivalence classes.
  \item
    If $X$ has a partition $X = \dot{\bigcup}_{i \in I} X_i$, then
    $X$ has an equivalence relation $R$ such that
    $X_i$ are equivalence classes, i.e., $R$ defined by
    \[
    x R y \text{ if $x,y$ are in the same $X_i$}
    \]
  \end{myenum}
\end{prop}

\begin{prop}
  Let $N > 0$.
  Then the relation $\equiv \pmod{N}$ on $\Z$ is an equivalance relation
\end{prop}


Let $R$ be an equivalence relation on $X$.
Define the function
\begin{align*}
  f_R: X &\rightarrow X/R \\
  f(x) &= [x]_R
\end{align*}

\begin{prop}
  \mbox{}
  \begin{myenum}
  \item
    If $x R y$, then $f_R(x) = f_R(y)$.
 \item
    $f_R$ is a well-defined onto function.
  \end{myenum}
\end{prop}

Let $R, R'$ be two relations on $X$ such that
$R'$ is \defone{finer} in the sense that
\[
x R' y \implies x R y
\]
Here are simple examples of the concept of finer relation:
\begin{myenum}
\item
  \lq\lq sibling" is finer than \lq\lq have the same great-grandfather".
\item
  \lq\lq divisible by 100" is finer than \lq\lq divisible by $5$".
\item
  \lq\lq $\equiv \pmod{100}$" is finer than \lq\lq $\equiv \pmod{5}$".  
\end{myenum}
If $R$ is an equivalence relation, we have equivalence classes (through $R$)
for $X$.
You can then define a relation $R'$ on $X/R$ as follows:
\[
[x]_R R' [y]_R \text{ if } x R' y
\]
Watch out: Note that $R'$ is the given relation $X$
and also $R'$ denotes the new relation on $X/R$.
Here are simple examples of the concept of finer relation:
\begin{myenum}
\item
  \lq\lq sibling" is finer than \lq\lq have the same great-grandfather".
  Let $[\text{John}]_{\text{sibling}} = \{\text{John, Sue, Tom}\}$,
    i.e., John, Sue, Tom are siblings.
  Also, let $[\text{Mary}]_{\text{sibling}} = \{\text{Bob, Mary}\}$,
    i.e., Bob, Mary are siblings.
    We can talk about whether
    $[\text{John}]$ and $[\text{Mary}]$ are related or not
    through the same great-grandfather relation.
    This is determined by whether John and Mary have the same
    great-grandfather.
    If John and Mary do have the same great-grandfather,
    then every $x \in [\text{John}]$ and $y \in [\text{Mary}]$
    also have the same great-grandfather.
\item
  \lq\lq $\equiv \pmod{100}$" is finer than \lq\lq $\equiv \pmod{5}$".
  Therefore for $x, y \in \Z$, I can define
  \[
  [x]_{5} \equiv [y]_5 \pmod{100} \text{ if } x \equiv y \pmod{100}
  \]
\end{myenum}

\begin{prop}
  If $R'$ is finer than $R$, then $R' \subseteq R$.
\end{prop}

If $R$ is a relation on $X$, you can enlarge the relation by \lq\lq taking
closure".
For instance the \defone{reflexive closure} of $R$ is the relation $R$, by
you add $(x, x)$ to $R$:
\[
R' = \{(x, x) \mid x \in X\} \cup R
\]
Then $R'$ is reflexive.
It's the smallest subset of $X \times X$ that is reflexive and contains $R$.

The \defone{symmetric closure} R' of $R$ is the smallest relation
of $X \times X$ that contains $R$ and is symmetric:
\[
R' = R \cup \{(y, x) \mid (x, y) \in R \text{ for } x,y \in X\}
\]

The \defone{transitive closure} $R'$ of $R$ is the smallest relation
of $X \times X$ that contains $R$ and is transitive.
Note that whereas reflexive closure and symmetric closures are easy to
understand,
transitive closure is a little bit more complicated.
For instance the following $R'$ is \textit{not} the transitive closure of $R$:
\[
R' = R \cup \{(x, z) \mid (x, y), (y, z) \in R \text{ for } x,y,z \in X\}
\]


\begin{ex} 
  \label{ex:rsa-10}
  \tinysidebar{\debug{exercises/{nt-61/question.tex}}}
  In your \verb!Zmod.py!, complete the following methods:
  \begin{enumerate}[nosep]
    \li multiplicative inverse mod $N$ (i.e., \texttt{inv})
    \li invertibility mod $N$ (i.e., \texttt{is\_invertible})
    \li division (i.e., \texttt{\_\_div\_\_})
  \end{enumerate}

  \solutionlink{sol:rsa-10}
  \qed
\end{ex} 
\begin{python0}
from solutions import *
add(label="ex:rsa-10",
    srcfilename='exercises/rsa-10/answer.tex') 
\end{python0}


The concept of \lq\lq closure" is very common in math and CS.

Back to the transitive closure, if $R$ is a relation between
$X, Y$ and $R'$ is a relation between $Y, Z$,
then we can define the \defone{composition} of $R \circ R'$:
\[
R \circ R' =
\{
(x, z) \in X \times Z \mid (x, y) \in R, (y, z) \in R' \text{ for some }
y \in Y
\}
\]
I'll write $R^2$ for $R \circ R$.
In general $R^1 = R$, $R^{n+1}$ will denote $R^n \circ R$.

\begin{prop}
  $R$ is transitive if $R^2 \subseteq R$.
\end{prop}

\begin{prop}
  $R$ is transitive if $\bigcup_{n = 1}^\infty R^n$.
\end{prop}
