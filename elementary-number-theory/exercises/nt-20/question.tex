\tinysidebar{\debug{exercises/{nt-20/question.tex}}}
  A positive integer is a \defterm{perfect} number if it is the
  sum of its positive divisors strictly less than itself.
  For instance $6$ is perfect since $6 = 1 + 2 + 3$.
  $28$ is also perfect since $28 = 1 + 2 + 4 + 7 + 14$.
  Euclid knew that if $p$ is prime and $2^p - 1$ is a prime,
  then $2^{p-1}(2^p - 1)$ is an even perfect number.
  If $p$ is a prime and $2^p - 1$ is a also a prime,
  then $2^p - 1$ is called a \defterm{Mersenne prime}.
  Almost 2000 years after Euclid, Euler proved the converse, that
  every even perfect number must be of the form
  $2^{p-1}(2^p - 1)$ where $2^p - 1$ is a Mersenne prime.
  There are two famous unsolved problems in number theory on perfect
  numbers:
  \begin{enumerate}[nosep]
  \item Are there odd perfect numbers?
  \item Are there infinitely many perfect numbers?
  \end{enumerate}
  (See \url{https://en.wikipedia.org/wiki/Perfect_number}.)
  There is an ongoing search for primes and Mersenne primes using
  computers.
  At this point (2023), the top 8 largest known primes are all
  Mersenne primes.
  (See \url{https://en.wikipedia.org/wiki/Great_Internet_Mersenne_Prime_Search}.)
  As of Feb 2023, the largest known prime is $2^{82,589,933} - 1$.
  (See \url{https://en.wikipedia.org/wiki/Largest_known_prime_number}.)
  Write a program that prints perfect numbers, printing the time between pairs of
  perfect numbers discovered.
