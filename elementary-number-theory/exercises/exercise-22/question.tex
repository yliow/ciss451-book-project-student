\tinysidebar{\debug{exercises/{exercise-22/question.tex}}}
(Dr.Liow's magic formula)
  Suppose I want to simplify $23532 \pmod{26}$.
  With a calculator or \cpp\ or python, we see quickly that
  \[
    23532 \equiv 2 \pmod{26}
  \]
  I claim that you can use this magic formula:
  Suppose the digits of a 5--digit number is $edcba$.
  For instance $edcba = 23532$ where $e = 2, d = 3, c = 5, b = 3, a = 2$.
  Then
  \[
    edcba \equiv ba + 2(-2c + 6d - 5e) \pmod{26}
  \]
  For instance when $edcba = 23532$, we have
  \[
    edcba \equiv 32 + 2(-2\cdot 5 + 6 \cdot 3 - 5 \cdot 2) \pmod{26}
  \]
  and
  \[
    32 + 2(-2\cdot 5 + 6 \cdot 3 - 5 \cdot 2) \equiv 6 + 2(-2) \equiv 2 \pmod{26}
  \]
  which is easier to work with than
  the much larger 23532.
  So ... is
  \[
    edcba \equiv ba + 2(-2c + 6d - 5e) \pmod{26}
  \]
  really true for any 5-digit number $edcba$? Write a program to test all cases.
  Next, write a proof (that does not involve checking all cases like a program).

