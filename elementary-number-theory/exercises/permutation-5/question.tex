\tinysidebar{\debug{exercises/{permutation-5/question.tex}}}
  Of course the process is tedious and error prone.
  So you know what to do:
  \begin{itemize}
    \li Write a program to encrypt and decrypt messages for the
    permutation cipher.
    For instance you can try this permutation
    \[
      \begin{pmatrix}
        1 & 2 & 3 & 4 & 5 \\
        2 & 3 & 5 & 4 & 1
      \end{pmatrix}
    \]
    by entering \verb!2,3,5,4,1!.
    Note that (mathematically speaking) permutations
    are written as bijection between sets $\{1, 2, 3, ..., m\}$.
    You might want to start with $0$ instead.
    In that case, you should tell your user.
    \li 
    From the above experience of computation-by-hand,
    we see that
    for breaking a permutation cipher, the program accepts
    a ciphertext and ask you for a permutation.
    Furthermore your program should allow you to enter a permutation
    partially.
    For instance to enter this permutation
    \[
      \begin{pmatrix}
        1 & 2 & 3 & 4 & 5 \\
        & 3 &   &   & 1
      \end{pmatrix}
    \]
    the user enters \verb!?,3,?,?,1! in your program.
    You also want to allow the user to modify a permutation.
    \li 
    You also want the program to list commonly occurring digrams
    that appears in the ciphertext, especially including those
    that appear within substrings of length $m$ (the length of
    the permutation).
    \li With all the above, you can then automate the process
    to imitate what I did in the previous example.
  \end{itemize}
