\tinysidebar{\debug{exercises/{exercise-21/question.tex}}}
Although the algebraic manipulations on mod $N$ seems to be very similar to
the algebraic manipulations in $\Z$ (or even $\R$),
certain \lq\lq bizarre'' things do happen.
  In $\Z$ (in fact in $\Q$ and $\R$ as well), you have the implication
  \[
    x y = 0 \implies x = 0 \text{ or } y = 0
  \]
  Is it true that
  \[
    x y \equiv 0 \pmod{26} \implies x \equiv 0 \pmod{26} \text{ or } y \equiv 0 \pmod{26}
  \]
  For each $N$, check when
  \[
    x y \equiv 0 \pmod{N} \implies x \equiv 0 \pmod{N} \text{ or } y \equiv 0 \pmod{N}
  \]
  holds.
  And when it does not, give one counterexample.
  Do you see a pattern?
