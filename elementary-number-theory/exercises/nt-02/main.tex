
\begin{ex} 
  \label{ex:nt-02}
  \tinysidebar{\debug{exercises/{nt-61/question.tex}}}
  In your \verb!Zmod.py!, complete the following methods:
  \begin{enumerate}[nosep]
    \li multiplicative inverse mod $N$ (i.e., \texttt{inv})
    \li invertibility mod $N$ (i.e., \texttt{is\_invertible})
    \li division (i.e., \texttt{\_\_div\_\_})
  \end{enumerate}

  \solutionlink{sol:nt-02}
  \qed
\end{ex} 
\begin{python0}
from solutions import *
add(label="ex:nt-02",
    srcfilename='exercises/nt-02/answer.tex') 
\end{python0}
