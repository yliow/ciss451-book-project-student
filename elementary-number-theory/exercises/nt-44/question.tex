\tinysidebar{\debug{exercises/{nt-44/question.tex}}}
  In your \verb!Zmod.py! complete the following:
  \begin{enumerate}[nosep]
    \li Exponentiation (i.e., \verb!__pow__!)
  \end{enumerate}
  Note that $x^{-1000} \pmod{N}$ is $(x^{-1})^{1000} \pmod{N}$.
  You want to first check if you can use Euler's Theorem.
  If it is, use the theorem to lower the exponent to say $\ell$.
  Then use the obvious loop to compute $x^\ell$ and apply $\pmod{N}$
  as frequently as possible.
  After you are done with the above,
  you might want to improve your \texttt{\_\_pow\_\_} by
  \textit{not} use Euler's theorem
  if the exponent is \lq\lq small".
  You can determine for yourself what if \lq\lq small".
  For instance you can choose to use Euler's Theorem
  only when the exponent is greater than $10$.
  (Later in the RSA chapter, we will talk about the squaring method.)
