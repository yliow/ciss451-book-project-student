\tinysidebar{\debug{exercises/{group-theory-powerset/answer.tex}}}

%% \text{Union}:
%% \begin{myenum}
%% \item If $A, B \in P(X)$, then $A \cup B \subseteq X$ and therefore
%% $A \cup B \in P(X)$.
%% \item If $A, B, C \in P(X)$, then $A \cup (B \cup C) = (A \cup B) \cup C$.
%% \end{myenum}
%% For the identity, we need $e$ to satisfy
%% \begin{myenum}
%% \item If $A \in P(X)$, $A \cup e = A = e \cup A$.
%% \end{myenum}
%% The only reasonable set for $e$ is $\emptyset$.
%% But with that, for the inverse axiom I need
%% \begin{myenum}
%% \item If $A \in P(X)$, there exists some $B \in P(X)$
%% such that $A \cup B = \emptyset = B \cup A$.
%% \end{myenum}
%% But this is not possible since $A \cup B$ contains $A$.
%% The only way for this to be possible is when $A = \emptyset$,
%% which means the only way for this axiom to hold is when $X = \emptyset$.
%% Hence $(P(X), \cup, \emptyset)$ is a group (in fact abelian group)
%% if $X = \emptyset$.
%% Otherwise $(P(X), \cup, \emptyset)$
%% satisfies all group axioms except for the inverse axiom.

$(P(X), \cup, \emptyset)$ satisfies all group axioms except for the inverse axiom.

%% \textsc{Intersection}:
%% \begin{myenum}
%% \item If $A, B \in P(X)$, then $A \cap B \subset X$ and therefore
%% $A cap B \in P(X)$.
%% \item If $A, B, C \in P(X)$, then $A \cap (B \cap C) = (A \cap B) \cap C$.
%% \end{myenum}
%% For the identity, we need $e$ to satisfy
%% \begin{myenum}
%% \item If $A \in P(X)$, $A \cap e = A = e \cap A$.
%% \end{myenum}
%% The only reasonable set for $e$ is $X$.
%% But with that, for the inverse axiom I need
%% \begin{myenum}
%% \item If $A \in P(X)$, there exists some $B \in P(X)$
%% such that $A \cap B = X = B \cap A$.
%% \end{myenum}
%% But this is not possible since $A \cap B$ is a subset of $A$.
%% The only way for this to be possible is $X = \emptyset$.
%% Hence $(P(X), \cap, X)$ is a group (in fact abelian group)
%% if $X = \emptyset$.
$(P(X), \cap, X)$ satisfies all group axioms except for the inverse axiom.



%% \textsc{Difference}:
%% Set difference is not associative.
%% For instance if $A = \{a, b\}, B = \{b\} = C$,
%% then
%% \[
%% (A - B) - C = \{a\}
%% \]
%% while
%% \[
%% A - (B - C) = \{a, b\}
%% \]
%% Therefore set difference cannot be a group law.
Set difference is not associative.

%% \textsc{Symmetric difference}:
%% If $A, B$ are subsets of X, then the symmetric difference of $A$ and $B$ is 
%% \[
%% A \triangle B = (A \cap \overline B) \cup (\overline A \cap B)
%% \]
%% $\triangle$ is closed (on $P(X)$) and abelian.
%% It is also associative
%% \begin{align*}
%% (A \triangle B) \triangle C
%% &= ((A \cap \overline B) \cup (\overline A \cap B)) \triangle C \\
%% &=
%% \left(((A \cap \overline B) \cup (B \cap \overline A)) \cap \overline{C}\right)
%% \cup
%% \left(\overline{((A \cap \overline B) \cup (B \cap \overline A))} \cap C\right)
%% \\
%% &=
%% \\
%% &= A \triangle (B \triangle C)
%% \end{align*}
%% Also
%% \[
%% A \triangle \emptyset = A = \emptyset \triangle A 
%% \]
%% and
%% \[
%% A \triangle \overline{A} = \emptyset = \overline{A} \triangle A 
%% \]
%% Therefore $(P(X), \triangle, \emptyset)$ is an abelian group.

$(P(X), \triangle, \emptyset)$ is an abelian group
where $\triangle$ is symmetric difference.

