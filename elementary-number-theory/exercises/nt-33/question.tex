\tinysidebar{\debug{exercises/{nt-33/question.tex}}}
  \mbox{}
  \begin{enumerate}[nosep]
    \item[(a)]
      Solve $\phi(n) = 2$, i.e., find all positive integers $n$ such that
      $\phi(n) = 2$.
      (Hint: Write down the prime factorization of
      $n = p_1^{e_1}\cdots p_g^{e_g}$ and use the
      equation $\phi(n) = 2$.)
    \item[(b)] Solve $\phi(n) = 3$.
    \item[(b)] Solve $\phi(n) = 6$.
  \end{enumerate}
\BEGINCOMMENT
\begin{Verbatim}
phi(3) = 3 - 1 = 2. 3 works.
phi(5) = 5 - 1 is too big.
phi(7) = 7 - 1 is too big.
phi(n)/n = (1 - 1/p)....
phi(n)/n = 2/n => 2/n = (1-1/p)...
If 2 | n:
  2 = n(1 - 1/2)((p-1)(q-1)(r-1)...)/(pqr...)
  4 = n((p-1)(q-1)(r-1)...)/(pqr...)
  if 2 || n:
     there can only be one other prime dividing n
     n = 2p^k
     p^k - p^{k-1} = 2
     p^{k-1}(p-1) = 2
     k = 1 and p = 3
     n = 2x3
  if 2^m || n, m >= 0
     2 = (2^m - 2^{m-1})p^(k-1)(p-1)...
     2 = 2^{m-1} p^(k-1)(p-1)...
     If m = 0: 2 = p^(k-1)(p-1)... => only one prime p, k = 1, p=3. n=3.
     If m = 1: 2 = p^(k-1)(p-1)... => only one prime p and k=1, p=3. n=6.
     If m = 2: 2 = 2xp^(k-1)(p-1)... => no other prime. n = 4.
     If m = 3: 2 = 4xp^(k-1)(p-1)... => not possible.
Therefore the only solutions are n = 3, 4, 6.

phi(n) = 3 
2^{a-1})p^(b-1)(p-1)... = 3. not possible if p > 2.

phi(n) = 6
2^{a-1}p^(b-1)(p-1)... = 6 => only primes 2, 3, a = 1, b = 2. n = 2x3x3
\end{Verbatim}
\ENDCOMMENT
