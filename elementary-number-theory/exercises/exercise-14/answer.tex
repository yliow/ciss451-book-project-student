\tinysidebar{\debug{exercises/{exercise-14/answer.tex}}}


    
Of course $(x,y,z) = (0,0,0)$ is a solution since, for instance, the polynomial is
homogeneous.
If any two of $x,y,z$ are 0s, then the third must also be 0.

Now consider the case where exactly one of $x,y,z$ is 0.
\begin{myenum}
\item
  If $x = 0$, then $4y^2 = 5z^2$.
  We can assume that all common factors between $y,z$ are removed so that
  $\gcd(y,z) = 1$.
  (Suppose $g = \gcd(y, z)$. Let $y' = y/g$ and $z' = z/g$,
  We again would arrive at $4y'^2 = 5z'^2$.)
  Since $5 \mid 5z^2$, we have $5 \mid 4y^2$, and hence $5 \mid y$.
  Therefore $y = 5y'$.
  Hence $4(5y')^2 = 5z^2$, i.e.,
  $4(5)y'^2 = z^2$, which implies that $5 \mid z$.
  This contradicts $\gcd(y, z) = 1$.

\item
  If $y = 0$, then $3 x^2 = 5 z^2$.
  We can assume that all common factors between $x,z$ are removed so that $\gcd(x, z) = 1$.
  Since $3 \mid 3 x^2$,
  we obtain $3 \mid 5z^2$, which implies that $3 \mid z$.
  Therefore $z = 3z'$ and hence $3x^2 = 5(3z')^2$.
  This implies that $x^2 = 5(3)z'^2$.
  Therefore $3 \mid x^2$, and hence $3 \mid x$.
  This contradicts $\gcd(x, z) = 1$.
  
\item
  If $z = 0$, then $3x^2 + 4y^2 = 0$.
  Since $3 \mid 3 x^2$, we get $3 \mid 4 y^2$,
  which implies that $3 \mid y$.
  Let $y = 3y'$.
  Then $3x^2 + 4(3y')^2 = 0$ and hence $x^2 = 4(3)y'^2$.
  Therefore $3 \mid x^2$ and hence $3 \mid x$.
  This contradicts $\gcd(x, y) = 1$.
\end{myenum}

Now suppose $(x, y, z) \neq (0,0,0)$.

Let $x,y,z$ be a solution with $x > 0, y > 0, z > 0$.
We may assume $\gcd(x,y,z) = 1$.

\textsc{Method 1.}
Taking mod 3,
\[
  y^2 \equiv 2 z^2 \pmod{3}
\]
Squares in mod 3 are $0$ or $1$ mod 3.
If $z^2 \equiv 1 \pmod{3}$, then $y^2 \equiv 2 \pmod{3}$
which is impossible.
Hence $z^2 \equiv 0 \pmod{3}$ and therefore $y^2 \equiv 0 \pmod{3}$.
This implies that $3 \mid z^2$ and $3 \mid y^2$.
Therefore
\[
  3 \mid y, \,\,\,\,\, 3 \mid z
\]
Let $y = 3y'$ and $z = 3z'$.
Then
\[
3x^2 + 4(3y')^2 = 5(3z')^2  
\]
i.e.,
\[
x^2 + 12y'^2 = 15z'^2  
\]
which implies that $3 \mid x$.
This is a contradiction since $\gcd(x, y, z) = 1$.


\textsc{Method 2.}
Taking mod 4, we get
\[
  3x^2 \equiv z^2 \pmod{4}
\]
Then $2 \mid x$ and $2 \mid z$.
Let $x = 2x'$ and $z = 2z'$.
Then
\[
  3(2x')^2 + 4y^2 = 5(2z')^2
\]
i.e.,
\[
  3x'^2 + y^2 = 5z'^2
\]
Taking mod 4,
\[
  3x'^2 + y^2 \equiv z'^2 \pmod{4}
\]
Then
\begin{tightlist}
  \li Assume $z'$ is even. Then either $x',y$ are even or $x'^2 \equiv 1 \equiv y^2 \pmod{4}$.
  If $x',y$ are even and $z'$ is also even, then $\gcd(x, y, z) \neq 1$ which is a contradiction.
  For the other case
  \[
    3(4m + 1)^2 + (4n + 1)^2 = (2z'')^2
  \]
  i.e.,
  \[
    3(16m^2 + 8m + 1) + (16n^2 + 8n + 1) = 4z''^2
  \]
  i.e.,
  \[
    12m^2 + 6m + 4n^2 + 2n  + 1 = z''^2
  \]
  All we can say is $z''$ is odd.
  There's no clear path forward.
  \li Assume $z'^2 \equiv 1 \pmod{4}$. Then $2 | x'$ and $y'^2 \equiv 1 \pmod{4}$.
\end{tightlist}
There are more cases to consider and this seems to be a bad direction to take.


\textsc{Method 3.}
Taking mod 5, we get
\[
  3x^2 + 4y^2 \equiv 0 \pmod{5}
\]
i.e.
\[
  3x^2 \equiv y^2 \pmod{5}
\]
Squares in mod 5 are $0,1,4 \pmod{5}$.
If $x^2 \equiv 1\pmod{5}$, then $y^2 \equiv 3x^2 \equiv 3 \pmod{5}$
which is impossible.
If $x^2 \equiv 4\pmod{5}$, then $y^2 \equiv 3x^2 \equiv 12 \equiv 2 \pmod{5}$
which is again impossible.
Hence $x^2 \equiv 0 \pmod{5}$ and hence $y^2 \equiv 0 \pmod{5}$.
Altogether we have
\[
  5 \mid x^2, \,\,\,\,\, 5 \mid y^2
\]
and therefore
\[
  5 \mid x, \,\,\,\,\, 5 \mid y
\]
Let $x = 5x'$ and $y = 5y'$. Then we have
\[
  3(5x')^2 + 4(5y')^2 = 5z^2
\]
i.e.,
\[
  3(5)x'^2 + 4(5)y'^2 = z^2
\]
which implies that $5 \mid z^2$. Therefore $5 \mid z$.
This contradicts $\gcd(x, y, z) = 1$.
\qed

\textsc{Notes.}
\begin{itemize}
  \item What about $9x^2 + 4y^2 = 5z^2$? How far can you go using the method above?
  \item What about $3x^2 + 4y^2 = 25z^2$? How far can you go using the method above?
  \item What about $9x^2 + 4y^2 = 25z^2$?
  \item What about $3x^2 + 2y^2 = 5z^2$?
  \item What about $3x^3 + 4y^3 = 5z^4$?
\end{itemize}

