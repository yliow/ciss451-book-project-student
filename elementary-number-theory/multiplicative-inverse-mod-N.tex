\sectionthree{Multiplicative inverse in $\Z/N$}
\begin{python0}
from solutions import *; clear()
\end{python0}

Quick review:
In algebra classes, lots of time is spent on solving equations.
You usually start with something like: \lq\lq Find the roots of
$x + b$''. This means finding a value for $x$ such that 
\[
x + a = 0
\]
Of course this is dead easy.
As long as you have the concept of $-a$ (additive inverse)
you just add $-a$ to both sides and 
voila:
\begin{align*}
(x + a) + (-a) &= 0 + (-a) \\
x + (a + (-a)) &= 0 + (-a) \\
x + 0 &= 0 + (-a) \\
x &= 0 + (-a) \\
x &= -a 
\end{align*}

Next up, you usually see this: \lq\lq Find the roots of $ax + b$''
which is the same as finding a value for $x$ such that 
\[
ax + b = 0
\]
Assuming you are working in $\R$, 
you would do something like this:
\begin{align*}
(ax + b) + (-b) &= 0 + (-b) \\
ax +  (b + (-b)) &= 0 + (-b) \\
ax + 0 &= 0 + (-b) \\
ax &= 0 + (-b) \\
ax &= -b 
\end{align*}
and at this point you would do this:
\begin{align*}
ax &= -b \\
a^{-1}(ax) &= a^{-1}(-b) \\
(a^{-1}a)x &= a^{-1}(-b) \\
1x &= a^{-1}(-b) \\
x &= a^{-1}(-b)
\end{align*}
and v\'oila (again) and you're done.

In the case of real numbers (or even just fractions), 
if you're given a number $a$ (which is not zero), you always have
another number called $a^{-1}$ (multiplicative inverse) such that 
\[
a \cdot a^{-1} = 1 = a^{-1} \cdot a
\]
The reason why we like this is exactly because it allows us to solve
the above equation.

Given a number $a$, the number $-a$ is called an
\defone{additive inverse}
if it behaves like this:
\[
a + (-a) = 0 = (-a) + a
\]
The number $a^{-1}$ is called a \defone{multiplicative inverse} of $a$
if it does this:
\[
a \cdot a^{-1} = 1 = a^{-1} a
\]
If $a$ has a multiplicative inverse, we also say that $a$ is invertible.

\begin{defn}
We also say that $a$ is a \defone{unit} if $a$ has a multiplicative inverse.
\end{defn}

\begin{defn}
The set of multiplicatively invertible elements of $\Z/N$
is denoted by $(\Z/N)^\times$ or $U(\Z/N)$.
Likewise the multiplicatively invertible elemnts of $\Z$
is denoted by $\Z^\times$ or $U(\Z)$.
Likewise we have $\Q^\times$, $\R^\times$, and $\C^\times$.
\end{defn}

Almost all values in $\R$ are invertible;
the only exception being $0$.
Therefore $\R^\times = \R - \{0\}$.
On the other hand, note that most numbers in $\Z$ do not have multiplicative
inverses.
In fact the only numbers in $\Z$ that have inverses are $1$ and $-1$.
So the only units in $\Z$ are $1, -1$:
\[
U(\Z) = \{-1, 1\}
\]

\begin{prop}
  The only units of $\Z$ are $1$, $-1$.
  In particular, $1^{-1} = 1$ and $(-1)^{-1} = -1$.
\end{prop}
\proof
This was already proven earlier.
The following is a different proof that uses the fundamental theorem
of arithmetic.
TODO
\qed

Recall from the section on congruences, we have a relation $\equiv \pmod{N}$
that relates values of $\Z$.
For instance when $N = 26$,
\[
3 \equiv 29 \pmod{26}
\]
and
\[
3 \equiv -26 \pmod{26}
\]
You can think of $\equiv \pmod{26}$ as a kind of equality on $\Z$
so that, in mod $26$, $3$ and $29$ are the same and $3$ and $-26$
are the same.
In this context, write $\Z/N$ for the set
\[
\Z/N = \{0, 1, 2, ..., N - 1\}
\]
We call the set $\Z/N$ \lq\lq $\Z$ mod $N$".
Note that this set $\Z/N$ also contains $N$, but $N$ is \lq\lq the same as" $0$, i.e.,
$N$ is just another way to write $0$ in the sense that
\[
N \equiv 0 \pmod{N}
\]
Note that $\Z/N$ also has $+$, $\cdot$ and has $0$ and $1$.

In fact $(\Z/N, +, \cdot, 0, 1)$ satisfies the algebraic properties of $+$, 
properties of $\cdot$, and distributivity.
If $N > 1$, $(\Z/N, +, \cdot, 0, 1)$ also satisfies the nontriviality
axiom, i.e., $0 \not\equiv 1 \pmod{N}$.
However $(\Z/N, +, \cdot, 0, 1)$ does not satisfy the integrality
axiom. Furthermore there's no concept of order $<$ on $\Z/N$
and hence there's no WOP or induction on $\Z/N$.

The above will be made precise in the section on congruence classes.

Later we will define the concept of commutative rings
which are exactly sets, each with its own $+$ and $\cdot$ operations,
satisfying the properties of $+$ and $\cdot$ and distributivity of $\Z$.
$\Z$ and $\Z/N$ are both commutative rings.

The ability for a number in some commutative ring
to have a multiplicative inverse (in that ring) is special.

While $\Z$ has only two units, $\Z/N$ ($N > 1$) might contain lots of units.
For instance look at $\Z/10$.
Is $3$ invertible mod 10?
In other words is there some $x$ such that
\[
3x \equiv 1 \pmod{10}
\]
$x \equiv 7 \pmod{10}$ satisfies the above.
In other words in mod $10$, $7$ is the multiplicative inverse of $3$.

(You can talk about $\Z/N$ for $N = 1$, but it's not very interesting
nor useful. $\Z/1$ has only one value, i.e., $0$ which behaves like the
additive neutral element and the multiplicative neutral element.)

Here's the obvious brute algorithm to find the multiplicative inverse of
an integer mod $N$:

\begin{Verbatim}[frame=single,fontsize=\footnotesize]
ALGORITHM: brute-force-inverse
INPUT: a, N
OUTPUT: x such that a * x % N is 1

for x = 1, 2, 3, ..., N - 1:
    if a * x % N == 1:
        return x
return None
\end{Verbatim}

Note that in the above algorithm we need not test $0$.


\begin{ex} 
  \label{ex:hill-10}
  \tinysidebar{\debug{exercises/{rsa-43/question.tex}}}
  What if you test for the condition $n = x^3 - y^3$?
  Is there are version of
  factorization using difference of cubes?
  %We have
  %\[
  %  x^3 - y^3 = (x - y)(x^2 + xy + y^2)
  %\]
  %Suppose $n = abc$.
  %\[
  %  \frac{}{}
  %\]

  \solutionlink{sol:hill-10}
  \qed
\end{ex} 
\begin{python0}
from solutions import *
add(label="ex:hill-10",
    srcfilename='exercises/hill-10/answer.tex') 
\end{python0}


It turns out that in $\Z/N$, you can easily find all the elements
with multiplicative inverses: $a \in \Z/N$ has a multiplicative
inverse if $\gcd(a, N) = 1$.

\begin{prop}
  Let $a, N$ be integers where $N > 0$. Then
  $a$ is (multiplicatively) invertible in $\Z/N$ iff $\gcd(a, N) = 1$.
\end{prop}

If $m,n$ are two integers such that $\gcd(m,n) = 1$, we say that
$m$ and $n$ are
\defone{coprime}.

\proof
TODO
\qed

The above immediately implies that
\begin{align*}
(\Z/N)^\times
&= \{ a \mid 0 \leq a \leq N - 1, \,\,\, \text{$a$ is invertible mod $N$} \}\\
&= \{ a \mid 0 \leq a \leq N - 1, \,\,\, \gcd(a, N) = 1 \}
\end{align*}
(Of course $\gcd(0, N) = N > 0$, so we can throw away $0$ in the above
if $N > 1$.)
So the key is the computation of $x,y$ such that
\[
1 = \gcd(a, N) = ax + Ny
\]
which is achieved by the Extended Euclidean Algorithm.
But wait a minute. I just need the $x$.
So I'll just modify the Extended Euclidean Algorithm slightly.

Here's the (earlier) EEA algorithm, slightly modified,
together with a function to compute the multiplicative inverse of $a \pmod{N}$:
\begin{Verbatim}[frame=single, fontsize=\footnotesize]
ALGORITHM: EEA2 (sort of EEA ... without the d, d0)
INPUTS: a, b
OUTPUTS:  r, c where r = gcd(a, b) = c*a + d*b for some d

a0, b0 = a, b
c0, c = 1, 0
q = a0 // b0
r = a0 - q * b0

while r > 0:   
    c, c0 = c0 - q * c, c

    a0, b0 = b0, r
    q = a0 // b0
    r = a0 - q * b0

r = b0
return r, c


ALGORITHM: mod-inverse
INPUTS: a, N
OUTPUT: x such that  (a * x) % N is 1

g, x = EEA2(a, N)
if g == 1:
    return x % N
else:
    return None
\end{Verbatim}


\begin{eg}
Does $135$ have an inverse mod $1673$?
If it does, find it using the Extended Euclidean Algorithm.
\end{eg}
\textsc{Solution}.
\begin{enumerate}[nosep]
  \item c0, c, q, r: 1, 0, 0, 135
  \item c0, c, q, r: 0, 1, 12, 53
  \item c0, c, q, r: 1, -12, 2, 29
  \item c0, c, q, r: -12, 25, 1, 24
  \item c0, c, q, r: 25, -37, 1, 5
  \item c0, c, q, r: -37, 62, 4 4
  \item c0, c, q, r: 62, -285, 1, 1
  \item c0, c, q, r: -285, 347, 4, 0
  \item r, c: 1, 347
\end{enumerate}
Therefore $135^{-1} \pmod{1673}$ is $347$.
And we check:
\[
135 \cdot 347 = 34845 = 1 + 28 \cdot 1673 \equiv 1 \pmod{1673}  
\]



\begin{ex} 
  \label{ex:hill-10}
  \tinysidebar{\debug{exercises/{rsa-43/question.tex}}}
  What if you test for the condition $n = x^3 - y^3$?
  Is there are version of
  factorization using difference of cubes?
  %We have
  %\[
  %  x^3 - y^3 = (x - y)(x^2 + xy + y^2)
  %\]
  %Suppose $n = abc$.
  %\[
  %  \frac{}{}
  %\]

  \solutionlink{sol:hill-10}
  \qed
\end{ex} 
\begin{python0}
from solutions import *
add(label="ex:hill-10",
    srcfilename='exercises/hill-10/answer.tex') 
\end{python0}


\begin{ex} 
  \label{ex:hill-10}
  \tinysidebar{\debug{exercises/{rsa-43/question.tex}}}
  What if you test for the condition $n = x^3 - y^3$?
  Is there are version of
  factorization using difference of cubes?
  %We have
  %\[
  %  x^3 - y^3 = (x - y)(x^2 + xy + y^2)
  %\]
  %Suppose $n = abc$.
  %\[
  %  \frac{}{}
  %\]

  \solutionlink{sol:hill-10}
  \qed
\end{ex} 
\begin{python0}
from solutions import *
add(label="ex:hill-10",
    srcfilename='exercises/hill-10/answer.tex') 
\end{python0}


\begin{ex} 
  \label{ex:hill-10}
  \tinysidebar{\debug{exercises/{rsa-43/question.tex}}}
  What if you test for the condition $n = x^3 - y^3$?
  Is there are version of
  factorization using difference of cubes?
  %We have
  %\[
  %  x^3 - y^3 = (x - y)(x^2 + xy + y^2)
  %\]
  %Suppose $n = abc$.
  %\[
  %  \frac{}{}
  %\]

  \solutionlink{sol:hill-10}
  \qed
\end{ex} 
\begin{python0}
from solutions import *
add(label="ex:hill-10",
    srcfilename='exercises/hill-10/answer.tex') 
\end{python0}

