\sectionthree{B\'ezout's identity and the Extended Euclidean Algorithm}

\begin{python0}
from solutions import *; clear()
\end{python0}

The following theorem is very important and plays a role in
Euclid's lemma on primes.
First we will define GCD.

\begin{defn}
Let $a,b \in \Z$ such that not both $a,b$ are $0$.
\begin{myenum}
\item Let $d in \Z$ where $d \neq 0$. $d$ is a \defterm{common divisor}
of $a$ and $b$ if $d \mid a$ and $d \mid b$.
\item Let $g \in \Z$. Then $g$ is the \defone{greatest common divisor}
of $a$ and $b$
if $g$ is a common divisor of $a,b$ and $g$ is the largest among all
common divisors of $a,b$.
The greatest common divisor is an integer if $a,b$ are not both zero
since in this case $\gcd(a, b)$ divides $\max(|a|, |b|)$
and hence $\gcd(a, b)$ is finite. 
\end{myenum}
\end{defn}

(Note that is $a = 0 = b$, then all integers are common divisors of
$a,b$. In this case the greatest common divisor of $a,b$ is not defined.)

\begin{thm}
  \textnormal{(\defterm{B\'ezout's identity}\index{B\'ezout's identity}\tinysidebar{B\'ezout's identity})}
If $a$ and $b$ be integers which are not both zero,
then there are integers $x,y$ such that
\[
\gcd(a,b) = ax + by
\]
\end{thm}

The $x,y$ in the above theorem are called
\defone{B\'ezout's coefficients}
of $a,b$.
They are not unique.


\begin{ex} 
  \label{ex:hill-10}
  \tinysidebar{\debug{exercises/{rsa-43/question.tex}}}
  What if you test for the condition $n = x^3 - y^3$?
  Is there are version of
  factorization using difference of cubes?
  %We have
  %\[
  %  x^3 - y^3 = (x - y)(x^2 + xy + y^2)
  %\]
  %Suppose $n = abc$.
  %\[
  %  \frac{}{}
  %\]

  \solutionlink{sol:hill-10}
  \qed
\end{ex} 
\begin{python0}
from solutions import *
add(label="ex:hill-10",
    srcfilename='exercises/hill-10/answer.tex') 
\end{python0}


B\'ezout's identity
\[
ax + by = \gcd(a, b)
\]
does not provide an algorithm to compute the
B\'ezout's coefficients $x, y$.
Then I'll do a computational example
that compute the gcd of $a,b$ as a linear combination of $a$ and $b$.
The example actually contains the idea behind the algorithm
to compute the B\'ezout's coefficients.
The algorithm is called
the Extended Euclidean Algorithm.

\proof
For convenience, let
me write $(a, b)$ as the set of linear combinations of $a$ and $b$, i.e.,
\[
(a, b) = \{ax + by \mid x,y \in \Z \}
\]
We will also write $(g)$ for the linear combination of $g$, i.e.,
\[
(g) = \{gx \mid x \in \Z \}
\]
(Such linear combinations are called ideals. They are extremely
important in of themselves. Historically, they were created
to study Fermat's last theorem. Since then they are crucial in the
the study of ring theory.)

TODO
\qed

The above does not give you an algorithm to compute the
$x$ and $y$.
First let me do an example to show you that
it's possible to compute $\gcd(a,b)$ as a linear combination of $a$ and $b$.
Then I'll give you the actual algorithm.

Recall that we have computed $\gcd(514, 24) = 2$.
B\'ezout's theorem says that it's possible to find $x$ and $y$
such that
\[
2 = \gcd(514, 24) = 514x + 24y
\]
How do we compute the $x$ and $y$?
Just like the gcd computation (the Euclidean Algorithm),
the $x,y$ are computed using the Euclidean property.
First we have
\begin{align*}
514 = 21 \cdot 24 + 10
\end{align*}
This implies that 
\[
514 \cdot 1 + 24 \cdot (-21) = 10
\]
Now it would be nice if the pesky $10$ goes away and is replaced by
$2$.
How would we do that?
Well look at $24$ and $10$ now.
We have
\[
24 = 2 \cdot 10 + 4
\]
again by Euclidean algorithm.
Multiplying the equation
\[
514 \cdot 1 + 24 \cdot (-21) = 10
\]
throughout by 2 gives us
\[
514 \cdot 2 + 24 \cdot (-42) = 2 \cdot 10
\]
The previous equation
\[
24 = 2 \cdot 10 + 4
\]
say that $2 \cdot 10$ can be replaced by $24 - 4$.
This means that
\[
514 \cdot 2 + 24 \cdot (-42) = 24 - 4
\]
Hmmm ... this says that we have now
\[
514 \cdot 2 + 24 \cdot (-43) = -4
\]
or
\[
514 \cdot (-2) + 24 \cdot 43 = 4
\]
What about 4?
Well, if we look at $10$ and $4$ just like what we did with $24$ and $10$
we would get
\[
10 = 2 \cdot 4 + 2
\]
and the remainder $2$ gives us the GCD!!!
Rearranging it a bit we have
\[
1 \cdot 10 + (-2) \cdot 4 = 2
\]
i.e. 2 is a linear combination of 10 and 4.
But earlier we say that 4 is a linear combination of 514 and 24 ...
\[
514 \cdot (-2) + 24 \cdot 43 = 4
\]
and even earlier we saw that 10 is also a linear combination of 514 and 24 ...
\[
514 \cdot 1 + 24 \cdot (-21) = 10
\]
Surely if we substitute all these values into the equation
\[
1 \cdot 10 + (-2) \cdot 4 = 2
\]
we would get 2 as a linear combination of 514 and 24.
Let's do it ...
\begin{align*}
2 
&= 1 \cdot 10 + (-2) \cdot 4 \\
&= 1 \cdot (514 \cdot 1 + 24 \cdot (-21)) + (-2) (514 \cdot (-2) + 24 \cdot 43) \\
&= 514 \cdot 1 + 24 \cdot (-21) + 514 \cdot 4 + 24 \cdot (-86) \\
&= 514 \cdot 5 + 24 \cdot (-107)
\end{align*}
V\'oila!


\begin{ex} 
  \label{ex:hill-10}
  \tinysidebar{\debug{exercises/{rsa-43/question.tex}}}
  What if you test for the condition $n = x^3 - y^3$?
  Is there are version of
  factorization using difference of cubes?
  %We have
  %\[
  %  x^3 - y^3 = (x - y)(x^2 + xy + y^2)
  %\]
  %Suppose $n = abc$.
  %\[
  %  \frac{}{}
  %\]

  \solutionlink{sol:hill-10}
  \qed
\end{ex} 
\begin{python0}
from solutions import *
add(label="ex:hill-10",
    srcfilename='exercises/hill-10/answer.tex') 
\end{python0}


\begin{ex} 
  \label{ex:hill-10}
  \tinysidebar{\debug{exercises/{rsa-43/question.tex}}}
  What if you test for the condition $n = x^3 - y^3$?
  Is there are version of
  factorization using difference of cubes?
  %We have
  %\[
  %  x^3 - y^3 = (x - y)(x^2 + xy + y^2)
  %\]
  %Suppose $n = abc$.
  %\[
  %  \frac{}{}
  %\]

  \solutionlink{sol:hill-10}
  \qed
\end{ex} 
\begin{python0}
from solutions import *
add(label="ex:hill-10",
    srcfilename='exercises/hill-10/answer.tex') 
\end{python0}


To help you analyze the above computation,
let me organize our computations a little.
If we can make the process systematic, then there is hope that we can 
make the idea work for all $a$ and $b$, i.e., then we would have an
algorithm and hence can program it and compute it's runtime performance.

We know for sure that we have to continually use
Euclidean property on pairs of numbers.
So here we go:
\begin{align*}
514 &= 21 \cdot 24 + 10 \\
24  &= 2 \cdot 10 + 4 \\
10  &= 2 \cdot 4 + 2 \\
4   &= 2 \cdot 2 + 0
\end{align*}
Note that this corresponds to the $\gcd$ computation
\begin{align*}
\gcd(514, 24) 
&= \gcd(24, 514 - 21 \cdot 24) = \gcd(24, 10) \\
&= \gcd(10, 24 - 2 \cdot 10) = \gcd(10, 4) \\
&= \gcd(4, 10 - 2 \cdot 4) = \gcd(4, 2)  \\
&= \gcd(2, 4 - 2\cdot 2) = \gcd(2, 0) \\
&= 2
\end{align*}
So in the computation
\begin{align*}
514 &= 21 \cdot 24 + 10 \\
24  &= 2 \cdot 10 + 4 \\
10  &= 2 \cdot 4 + 2 \\
4   &= 2 \cdot 2 + 0
\end{align*}
if the remainder is $0$ (see the last line), 
then the previous line's remainder must be the gcd.

Let's look at our computation of the gcd of $514$ and $24$:
\begin{align*}
514 &= 21 \cdot 24 + 10 \\
24  &= 2 \cdot 10 + 4 \\
10  &= 2 \cdot 4 + 2 \\
4   &= 2 \cdot 2 + 0
\end{align*}
Recall that the above computation means that the gcd is 2.
Note only that through backward substitution, we can rewrite
$2$ as a linear combination of $514$ and $24$.

Let's try to do this in a more organized way.
So here's our facts again:
\begin{align*}
514 &= 21 \cdot 24 + 10 \\
24  &= 2 \cdot 10 + 4 \\
10  &= 2 \cdot 4 + 2 
\end{align*}
Let me put the remainders on one side:
\begin{align*}
10 &= 514 - 21 \cdot 24 \tag{1} \\
4  &= 24  - 2 \cdot 10  \tag{2} \\
2  &= 10  - 2 \cdot 4   \tag{3} \\
\end{align*}
Note that (1) tells you that
$10$ is a linear combination of $514,24$.
(2) tells you that $4$ is a linear combination of $24,10$.
If we substitute (1) into (2), $4$ will become a linear
combination of $514,24$.
(3) says that $2$ is a linear combination of $10,4$.
But $10$ is a linear combination of $514,24$
and $4$ is a linear combination of $514,24$.
Hence $2$ is also a linear combination of $514,24$.
See it?

OK. Let's do it.
From
\begin{align*}
10 &= 514 - 21 \cdot 24 \tag{1} \\
4  &= 24  - 2 \cdot 10  \tag{2} \\
2  &= 10  - 2 \cdot 4   \tag{3} \\
\end{align*}
if we substitute (1) into (2) and (3) (i.e., rewrite $10$ as a linear
combination of $514,24$):
\begin{align*}
10 &= 514 - 21 \cdot 24 \tag{1} \\
4  &= 24  - 2 \cdot (514 - 21 \cdot 24)  \tag{2} \\
2  &= (514 - 21 \cdot 24)  - 2 \cdot 4   \tag{3} 
\end{align*}
Collecting the multiples of 514 and 24:
\begin{align*}
10 &= 514 - 21 \cdot 24 \tag{1} \\
4  &= (-2) \cdot 514 + (1 + (-2)(-21)) \cdot 24  \tag{2'} \\
2  &= (1) \cdot 514 + (-21) \cdot 24  - 2 \cdot 4   \tag{3'} 
\end{align*}
and simplifying:
\begin{align*}
10 &= 514 - 21 \cdot 24 \tag{1} \\
4  &= (-2) \cdot 514 + (43) \cdot 24  \tag{2$'$} \\
2  &= (1) \cdot 514 + (-21) \cdot 24  - 2 \cdot 4   \tag{3$'$} 
\end{align*}
Substituting (2$'$) into (3$'$):
\begin{align*}
10 &= 514 - 21 \cdot 24 \tag{1} \\
4  &= (-2) \cdot 514 + (43) \cdot 24  \tag{2$'$} \\
2  &= (1) \cdot 514 + (-21) \cdot 24  - 2 \cdot ((-2) \cdot 514 + (43) \cdot 24)  \tag{3$'$} 
\end{align*}
Tidying up:
\begin{align*}
10 &= 514 - 21 \cdot 24 \tag{1} \\
4  &= (-2) \cdot 514 + (43) \cdot 24  \tag{2'} \\
2  &= (1 - 2(-2)) \cdot 514 + (-21 - 2(43)) \cdot 24 \tag{3''} 
\end{align*}
Simplifying:
\begin{align*}
10 &= 514 - 21 \cdot 24 \tag{1} \\
4  &= (-2) \cdot 514 + (43) \cdot 24  \tag{2'} \\
2  &= (5) \cdot 514 + (-107) \cdot 24 \tag{3''}
\end{align*}

(It's a good idea to check after each substitution that
the equalities still hold. We all make mistakes, right?)

OK.
That's great.
It looks more organized now.
So much so that you can now easily write a program to compute
the above.


Now let's look at the general case.
Suppose instead of $514$ and $24$, we write $a$ and $b$.
The computation will look like this:
\begin{align*}
a   &= q_1 \cdot b   + r_1 \\
b   &= q_2 \cdot r_1 + r_2 \\
r_1 &= q_3 \cdot r_2 + r_3 \\
r_2 &= q_4 \cdot r_3 + 0
\end{align*}
To make things even more regular and uniform, let me rewrite it this way:
\begin{align*}
r_0 &= q_1 \cdot r_1 + r_2 \\
r_1 &= q_2 \cdot r_2 + r_3 \\
r_2 &= q_3 \cdot r_3 + r_4 \\
r_3 &= q_4 \cdot r_4 + 0
\end{align*}
In other words let $r_0 = a$ and $r_1 = b$ and do the obvious.
A lot nicer, right?
Let me write it this way with the remainder term on the lefts:
\begin{align*}
r_2 &= (1) \cdot r_0 + (-q_1) \cdot r_1 \\
r_3 &= (1) \cdot r_1 + (-q_2) \cdot r_2 \\
r_4 &= (1) \cdot r_2 + (-q_3) \cdot r_3 
\end{align*}
(Remember that $r_4$ is the gcd of $r_0 = 514$ and $r_1 = 24$, right?)
Organized this way, we have the gcd on one side of the equation.
Now if we substitute the first equation into the second we get
\begin{align*}
r_2 &= (1) \cdot r_0 + (-q_1) \cdot r_1 \text{ ... USED }\\
r_3 &= (1) \cdot r_1 + (-q_2) \cdot ((1) \cdot r_0 + (-q_1) \cdot r_1) \\
r_4 &= (1) \cdot r_2 + (-q_3) \cdot r_3 
\end{align*}
i.e.,
\begin{align*}
r_2 &= (1) \cdot r_0 + (-q_1) \cdot r_1 \text{ ... USED }\\
r_3 &= (-q_2) \cdot r_0 + (1 + q_1q_2) \cdot r_1 \\
r_4 &= (1) \cdot r_2 + (-q_3) \cdot r_3  
\end{align*}
Note that we cannot throw away the first equation yet.
We need to keep $r_2$ around since it appears in the third equation!
So when can we throw the first equation away?
Look at the general case.
Suppose we have
\begin{align*}
r_2 &= (1) \cdot r_0 + (-q_1) \cdot r_1  \\
r_3 &= (1) \cdot r_1 + (-q_2) \cdot r_2  \\
r_4 &= (1) \cdot r_2 + (-q_3) \cdot r_3  \\
r_5 &= (1) \cdot r_3 + (-q_4) \cdot r_4  \\
r_6 &= (1) \cdot r_4 + (-q_5) \cdot r_5  \\
...
\end{align*}
Aha! $r_2$ is used only in the next \textit{two} equations.

Suppose we are at equation 3:
\begin{align*}
r_3 &= c_1 \cdot r_0 + d_1 \cdot r_1  \\
r_4 &= c_2 \cdot r_0 + d_2 \cdot r_1 
\end{align*}
We have to compute the next equation:
This requires $r_3, r_4$.
Then we have
\[
r_5 = (1) \cdot r_3 + (-q_4) \cdot r_4
\]
where
\[
q_4 = \floor{r_3/r_4}, \,\,\,\,\, r_5 = r_3 - q_4r_4
\]
Altogether we have
\begin{align*}
r_3 &= c_1 \cdot r_0 + d_1 \cdot r_1     \\
r_4 &= c_2 \cdot r_0 + d_2 \cdot r_1     \\
r_5 &= (1) \cdot r_3 + (-q_4) \cdot r_4 
\end{align*}
The last equation becomes
\begin{align*}
r_5 &= c_1 \cdot r_0 + d_1 \cdot r_1  
+ 
(-q_4) \cdot (c_2 \cdot r_0 + d_2 \cdot r_1)
\end{align*}
i.e.
\begin{align*}
r_5 &= (c_1 - q_4 c_2) \cdot r_0 + (d_1 - q_4 d_2) \cdot r_1 
\end{align*}

Let me repeat that in a slightly more general context.
If we have
\begin{align*}
r_3 &= c_1 \cdot r_0 + d_1 \cdot r_1  \\
r_4 &= c_2 \cdot r_0 + d_2 \cdot r_1 
\end{align*}
then we get (throwing away the first equation):
\begin{align*}
r_4 &=c_2 \cdot r_0 + d_2 \cdot r_1 \\
r_5 &= (c_1 - q_4 c_2) \cdot r_0 + (d_1 - q_4 d_2) \cdot r_1 
\end{align*}
To put it in terms of numbers instead of equations this is what we get:
If we have
\[
c_1, d_1, c_2, d_2, r_3, r_4
\]
then we get
\[
c_2, d_2, 
c_1 - \floor{r_3/r_4} c_2, 
d_1 - \floor{r_3/r_4}d_2, 
r_4, r_3 - \floor{r_3/r_4}r_4
\]
In general, if we have
\[
c, d, c', d', r, r'
\]
then we get
\[
c', d', 
c - \floor{r/r'} c', d - \floor{r/r'}d', 
r', r - \floor{r/r'}r'
\]

Of course since we start off with $r_0, r_1$ (i.e. what we call $a$ and $b$
above), we have
\begin{align*}
r_0 &= 1 \cdot r_0 + 0 \cdot r_1 \\
r_1 &= 0 \cdot r_0 + 1 \cdot r_1 \\
\end{align*}
i.e., you would start off with
\[
c=1, d=0, c'=0, d'=1, r=r_0, r'=r_1
\]
Let's check this algorithm with our $r_0 = 514, r_1 = 24$.

STEP 1: The initial numbers are
\[
c=1, d=0, c'=0, d'=1, r=514, r'=24
\]
Again this corresponds to
\begin{align*}
r_3 &= 1 \cdot 514 + 0 \cdot 24  \\
r_4 &= 0 \cdot 514 + 1 \cdot 24 \\
\end{align*}

STEP 2: 
The new numbers (6 of them) are
\begin{align*}
 c' &=0 \\
 d' &=1 \\
 c - \floor{r/r'} c' &= 1 - \floor{514/24} 0 = 1\\
 d - \floor{r/r'}d' &= 0 - \floor{514/24}1 = 0 - 21 = -21 \\
 r' &= 24 \\ 
 r - \floor{r/r'}r' &= 514 - \floor{514/24}24 = 514 - 504 = 10
\end{align*}
So the new numbers (we reset the variables in the algorithm):
\[
c=0, d=1, c'=1, d'=-21, r=24, r'= 10
\]
These corresponds to the data on the second and third line of the following:
\begin{align*}
514 &= 1 \cdot 514 + 0 \cdot 24      \\
24  &= 0 \cdot 514 + 1 \cdot 24      \\
10  &= 1 \cdot 514 + (-21) \cdot 24 
\end{align*}

STEP 3: From the 6 numbers from STEP 2 we get
\begin{align*}
 c' &= 1 \\
 d' &= -21 \\
 c - \floor{r/r'} c' &= 0 - \floor{24/10} 1 = -2 \\
 d - \floor{r/r'}d'  &= 1 - \floor{24/10} (-21) = 1 + 42 = 43 \\
 r' &= 10 \\ 
 r - \floor{r/r'}r' &= 24 - \floor{24/10}10 = 24 - 20 = 4
\end{align*}
So the new numbers (we reset the variables in the algorithm):
\[
c=1, d=-21, c'=-2, d'= 43, r=10, r'= 4
\]
These corresponds to the data on the third and fourth line of the following:
\begin{align*}
514 &= 1 \cdot 514 + 0 \cdot 24     \\
24  &= 0 \cdot 514 + 1 \cdot 24     \\
10  &= 1 \cdot 514 + (-21) \cdot 24 \\
4   &= (-2) \cdot 514 + 43 \cdot 24 
\end{align*}

STEP 4: From the 6 numbers from STEP 3 we get
\begin{align*}
 c' &= -2 \\
 d' &= 43 \\
 c - \floor{r/r'} c' &= 1 - \floor{10/4} (-2) = 1 + 4 = 5 \\
 d - \floor{r/r'}d'  &= -21 - \floor{10/4} (43) = -21 - 86 = -107 \\
 r' &= 4 \\ 
 r - \floor{r/r'}r' &= 10 - \floor{10/4} 4 = 10 - 8 = 2
\end{align*}
So the new numbers (we reset the variables in the algorithm):
\[
c = -2, d = 43, c' = 5, d'= -107, r=4, r'= 2
\]
These corresponds to the data on the fourth and fifth line of the following:
\begin{align*}
514 &= 1 \cdot 514 + 0 \cdot 24      \\
24  &= 0 \cdot 514 + 1 \cdot 24      \\
10  &= 1 \cdot 514 + (-21) \cdot 24  \\
4   &= (-2) \cdot 514 + 43 \cdot 24  \\
2   &= 5 \cdot 514 + (-107) \cdot 24 
\end{align*}

STEP 5: From the 6 numbers from STEP 4 we get
\begin{align*}
 c' &= 5 \\
 d' &= -107 \\
 c - \floor{r/r'} c' &= -2 - \floor{4/2} 5 = -2 - 10 = -12 \\
 d - \floor{r/r'} d'  &= 43 - \floor{4/2} (-107) = 43 + 214 = 257 \\
 r' &= 2 \\ 
 r - \floor{r/r'}r' &= 4 - \floor{4/2} 2 = 4 - 4 = 0
\end{align*}
So the new numbers (we reset the variables in the algorithm):
\[
c = 5, d = -107, c' = -12, d'= 257, r=2, r'= 0
\]
These corresponds to the data on the fifth and sixth line of the following:
\begin{align*}
514 &= 1 \cdot 514 + 0 \cdot 24       \\
24  &= 0 \cdot 514 + 1 \cdot 24       \\
10  &= 1 \cdot 514 + (-21) \cdot 24   \\
4   &= (-2) \cdot 514 + 43 \cdot 24   \\
2   &= 5 \cdot 514 + (-107) \cdot 24  \\
0   &= (-12) \cdot 514 + 257 \cdot 24 
\end{align*}

Of course (as before) at this point, you see that the $r'=0$.
Therefore 
\[
\gcd(514, 24) = 2
\]
and furthermore from $c=5, d=-107$, we get
\[
5 \cdot 514 + (-107) \cdot 24 = \gcd(514, 24) 
\]

Here's a Python implementation with some test code:
\begin{Verbatim}[frame=single, fontsize=\footnotesize]
ALGORITHM: EEA (Extended Euclidean Algorithm)
INPUTS:    a, b
OUTPUTS:   r, c, d where r = gcd(a, b) = c * a + d * b

a0, b0 = a, b
d0, d = 0, 1
c0, c = 1, 0
q = a0 // b0
r = a0 - q * b0

while r > 0:
    d, d0 = d0 - q * d, d    
    c, c0 = c0 - q * c, c

    a0, b0 = b0, r
    q = a0 // b0
    r = a0 - q * b0

r = b0
return r, c, d
\end{Verbatim}
You can pound real hard at the Extended Euclidean Algorithm with this:
\begin{Verbatim}[frame=single,fontsize=\footnotesize]
for a in range(1, 1000):
    for b in range(1, 1000):
        gcd, x, y = EEA(a, b)
        if gcd != a * x + b * y:
            print("ERROR! ...", a, b)
\end{Verbatim}

Note that even if you have to go through 100 lines of 
Euclidean computations, the number of variables stay the same:
besides the inputs, there are 7 variables (i.e., \verb!q, c, d, c', d', r, r'!)

By the way, this is somewhat similar to what we call
\textit{tail recursion} (CISS445)
an extremely important technique in functional programming.
All LISP hackers and people working in high performance computing
and compilers swear by it.
You don't see recursion in the above code, but you can 
replace the while-loop with recursion and if
you have a compiler/interpreter that can perform 
true tail recursion, then it would run exactly like the above algorithm.


Why is this algorithm correct?
And why does it terminate?
Here's the earlier list of computations:
\begin{align*}
r_2 &= (1) \cdot r_0 + (-q_1) \cdot r_1  \\
r_3 &= (1) \cdot r_1 + (-q_2) \cdot r_2  \\
r_4 &= (1) \cdot r_2 + (-q_3) \cdot r_3  \\
r_5 &= (1) \cdot r_3 + (-q_4) \cdot r_4  \\
r_6 &= (1) \cdot r_4 + (-q_5) \cdot r_5  \\
...
\end{align*}
where $r_0 = a$ and $r_1 = b$.
First of all the $r_0, r_1, r_2$ comes from the Euclidean property:
\[
r_0 = r_1q_1 + r_2, \,\,\, 0 \leq r_2 < r_1
\]
Likewise
\[
r_1 = r_2q_2 + r_3, \,\,\, 0 \leq r_3 < r_2
\]
Etc.
Hence
\[
0 \leq \ldots < r_3 < r_2 < r_1
\]
Clearly at some point some $r_k$ must be $0$ and furthermore
this is the smallest $k$ such taht $r_k = 0$.

Furthermore
\begin{align*}
\gcd(r_0, r_1) = \gcd(r_1, r_2) = \gcd(r_2, r_3) = ... = \gcd(r_{k-1}, r_k)
\end{align*}
Assuming $r_k = 0$, then
\begin{align*}
\gcd(r_0, r_1) = ... = \gcd(r_{k-1}, r_k) = \gcd(r_{k-1}, 0) = r_{k - 1}
\end{align*}

Also, from the earlier computation, from
\begin{align*}
r_3 &= c_1 \cdot r_0 + d_1 \cdot r_1  \\
r_4 &= c_2 \cdot r_0 + d_2 \cdot r_1 
\end{align*}
we get
\begin{align*}
r_4 &=c_2 \cdot r_0 + d_2 \cdot r_1 \\
r_5 &= (c_1 - q_4 c_2) \cdot r_0 + (d_1 - q_4 d_2) \cdot r_1 
\end{align*}
More generally, each $r_i$ is expressed as a linear combination of
$r_0 = a$ and $r_1 = b$.



\begin{ex} 
  \label{ex:hill-10}
  \tinysidebar{\debug{exercises/{rsa-43/question.tex}}}
  What if you test for the condition $n = x^3 - y^3$?
  Is there are version of
  factorization using difference of cubes?
  %We have
  %\[
  %  x^3 - y^3 = (x - y)(x^2 + xy + y^2)
  %\]
  %Suppose $n = abc$.
  %\[
  %  \frac{}{}
  %\]

  \solutionlink{sol:hill-10}
  \qed
\end{ex} 
\begin{python0}
from solutions import *
add(label="ex:hill-10",
    srcfilename='exercises/hill-10/answer.tex') 
\end{python0}


You'll see that there are times when you're only interested in 
the value of $x$ and not $y$ (or $y$ and not $x$ -- the above is symmetric
about $x$ and $y$).
Do you notice $x$ comes from $c$?
If you analyze the above algorithm, you see immediately that $c$
is computed from $c'$ and $c'$ is computed from $c,c',q$, 
$q$ is computed from $r, r'$, $r$ is computed from $r'$,
and finally (phew!) $r'$ is computed from $r, q, r'$.
Therefore if you're interested in $c$, you don't need to compute $d$ 
or $d'$.
So you can change the EEA to this:

\begin{Verbatim}[frame=single, fontsize=\small]
ALGORTHM: EEA2 (sort of EEA ... without the d, d0)
INPUTS: a >=0, b >=0
OUTPUTS: r, c where r = gcd(a, b) = c*a + d*b for some d

    a0, b0 = a, b
    c0, c = 1, 0
    q = a0 // b0
    r = a0 - q * b0

    while r > 0:   
        c, c0 = c0 - q * c, c
    
        a0, b0 = b0, r
        q = a0 // b0
        r = a0 - q * b0

    r = b0
    return r, c
\end{Verbatim}

Let's test this with 514 and 24 again.
\begin{myenum}
\item[] STEP 1: $c=1, c'=0, r=514, r'=24$
\item[] STEP 2: $q=21, c=c'=0, c'=c-qc'=1-21(0)=1, r=24,r'=r-qr'=514-21(24)=10$.
\item[] STEP 3: $q=2, c=c'=1, c'=c-qc'=0-2(1)=-2, r=10, r'=r-qr'=24-2(10)=4$
\item[] STEP 4: $q=2, c=c'=-2, c'=c-qc'=1-2(-2)=5, r=4, r'=r-qr'=10-2(4)=2$
\item[] STEP 5: $q=2, c=c'=5, c'=c-qc'=-2-2(5)=-12, r=2, r'=r-qr'=4-2(2)=0$
\end{myenum}
At this point $r'=0$. Therefore the gcd is again 2 and $x=5$.
Therefore 
\[
5 \cdot 514 + y \cdot 24 = 2
\]
for some $y$.

Later you'll see why we compute only $x$.


\begin{ex} 
  \label{ex:hill-10}
  \tinysidebar{\debug{exercises/{rsa-43/question.tex}}}
  What if you test for the condition $n = x^3 - y^3$?
  Is there are version of
  factorization using difference of cubes?
  %We have
  %\[
  %  x^3 - y^3 = (x - y)(x^2 + xy + y^2)
  %\]
  %Suppose $n = abc$.
  %\[
  %  \frac{}{}
  %\]

  \solutionlink{sol:hill-10}
  \qed
\end{ex} 
\begin{python0}
from solutions import *
add(label="ex:hill-10",
    srcfilename='exercises/hill-10/answer.tex') 
\end{python0}


\begin{ex} 
  \label{ex:hill-10}
  \tinysidebar{\debug{exercises/{rsa-43/question.tex}}}
  What if you test for the condition $n = x^3 - y^3$?
  Is there are version of
  factorization using difference of cubes?
  %We have
  %\[
  %  x^3 - y^3 = (x - y)(x^2 + xy + y^2)
  %\]
  %Suppose $n = abc$.
  %\[
  %  \frac{}{}
  %\]

  \solutionlink{sol:hill-10}
  \qed
\end{ex} 
\begin{python0}
from solutions import *
add(label="ex:hill-10",
    srcfilename='exercises/hill-10/answer.tex') 
\end{python0}


\begin{ex} 
  \label{ex:hill-10}
  \tinysidebar{\debug{exercises/{rsa-43/question.tex}}}
  What if you test for the condition $n = x^3 - y^3$?
  Is there are version of
  factorization using difference of cubes?
  %We have
  %\[
  %  x^3 - y^3 = (x - y)(x^2 + xy + y^2)
  %\]
  %Suppose $n = abc$.
  %\[
  %  \frac{}{}
  %\]

  \solutionlink{sol:hill-10}
  \qed
\end{ex} 
\begin{python0}
from solutions import *
add(label="ex:hill-10",
    srcfilename='exercises/hill-10/answer.tex') 
\end{python0}


\begin{ex} 
  \label{ex:hill-10}
  \tinysidebar{\debug{exercises/{rsa-43/question.tex}}}
  What if you test for the condition $n = x^3 - y^3$?
  Is there are version of
  factorization using difference of cubes?
  %We have
  %\[
  %  x^3 - y^3 = (x - y)(x^2 + xy + y^2)
  %\]
  %Suppose $n = abc$.
  %\[
  %  \frac{}{}
  %\]

  \solutionlink{sol:hill-10}
  \qed
\end{ex} 
\begin{python0}
from solutions import *
add(label="ex:hill-10",
    srcfilename='exercises/hill-10/answer.tex') 
\end{python0}

