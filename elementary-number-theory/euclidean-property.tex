\sectionthree{Euclidean property}
\begin{python0}
from solutions import *; clear()
\end{python0}

$\Z$ satisfies this very important property:

\begin{thm}\label{euclideanproperty}
  \textnormal{(\defterm{Euclidean property}\index{Euclidean property}\tinysidebar{Euclidean property})}
  If $a, b \in \Z$ with $b \neq 0$, then there are
  integers $q$ and $r$ 
  satisfying
  \[
  a = bq + r, \,\,\,\,\, 0 \leq |r| < |b|
  \]
\end{thm}

The above theorem is the version that can be generalized to general
rings.
Below is the version for $\Z$.
The only difference is the $|r|$ is replaced by $r$:
\begin{thm}\label{euclideanproperty2}
  \textnormal{(\defterm{Euclidean property 2}\index{Euclidean property 2}\tinysidebar{Euclidean property 2})}
  If $a, b \in \Z$ with $b \neq 0$, then there are
  integers $q$ and $r$ 
  satisfying
  \[
  a = bq + r, \,\,\,\,\, 0 \leq r < |b|
  \]
\end{thm}

In many cases, one is interested in the case when $a \geq 0$.
So this version is the one found in most textbooks:

\begin{thm}\label{euclideanproperty3}
  \textnormal{(\defterm{Euclidean property 3}\index{Euclidean property 3}\tinysidebar{Euclidean property 3})}
  If $a, b \in \Z$ with $a \geq 0, b > 0$, then there are
  integers $q \geq 0$ and $r \geq 0$ 
  satisfying
  \[
  a = bq + r, \,\,\,\,\, 0 \leq r < b
  \]
\end{thm}
Note that $a, b, q, r$ are in $\N = \{0, 1, 2, ...\}$.
Hence the above can also be called Euclidean property of $\N$.

$q$ is called the
\defterm{quotient}\index{quotient}\tinysidebar{quotient\\remainder} 
when $a$ is divided by $b$; $r$ is the
\defterm{remainder}\index{remainder}.
$q$ and $r$ are unique (see proposition below).
For instance if $a = 25$ and $b = 3$, then
\[
25 = 3 \cdot 8 + 1, \,\,\,\,\, 0 \leq 1 < 3
\]
The computation
\[
a,b \rightarrow q, r
\]
is called a
\defterm{division algorithm}\index{division algorithm}\tinysidebar{division algorithm}.

In Python, you can do this:
\begin{python}
s = r'''
a = 25
b = 8
q, r = divmod(25, 8)
print("%s = %s * %s + %s" % (a, b, q, r))'''.strip()
f = open('divmod.py', 'w'); f.write(s); f.close()
from latextool_basic import *
print(r'{\footnotesize %s}' % console(s))
print('The output is')
print(shell('python divmod.py'))
\end{python}
Algorithmically, when $a$ and $b$ have a huge number of digits
and they are represented using arrays of digits, the division
algorithm to compute $q,r$ is basically long division you learnt in middle
school.
At the hardware level, the same division algorithm occurs but
the computation is in terms of bits and not digits.

If we peek ahead and pretend for the time being that fractions such as 
$\frac{a}{b}$ exists,
then for $a > 0$ and $b > 0$, we have
\[
q = \floor{\frac{a}{b}}, \,\,\,\,\, r = a - bq
\] 
where $\floor{x}$ means the floor of $x$.
If we write $\verb!(a/b)!$ for the
\textit{integer} quotient of $a$ by $b$
(i.e. this is the $/$ in C\texttt{++} for integers) and $\verb!(a%b)!$
for the corresponding remainder, then of course we have
\[
   \texttt{a = b * (a/b) + (a\%b)}
\]

Although the above Euclidean property is for $\Z$,
we will first prove it for $a \geq 0$ and $b > 0$.
The $q,r$ will satisfy $q \geq 0$, $r \geq 0$.
(Furthermore in this setup $q,r$ are unique.)
Once we have proven the Euclidean property for integer $a \geq 0$,
it will not be difficult to extend the result to the whole of $\Z$.

To prove the Euclidean property of $\Z$, we will use WOP.
(One can also prove the Euclidean property of $\Z$ using induction.)

\textsc{Well-ordering principle (WOP) for $\N$}\index{Well-ordering principle for $\N$}\tinysidebar{Well-ordering principle for $\N$}:
Let $X$ be a nonempty subset of $\N$. Then $X$ has a least element.
In other words there is some $m \in X$ such that
$m \leq x$ for all $x \in X$.

You can prove the following version of
well-ordering principle on $\Z$:

\index{WOP for $\Z$}\textsc{Well-ordering principle for $\Z$}\tinysidebar{Well-ordering principle for $\Z$}:
Let $X$ be a nonempty subset of $\Z$ that is \textit{bounded below}.
Then $X$ has a least element.
In other words there is some $m \in X$ such that
$m \leq x$ for all $x \in X$.

$\R$ does not satisfy the second version well-ordering principle with $\Z$
replaced by $\R$.
For instance the open interval $X = (0, 1)$ is bounded below (for instance by $-42$).
However there is no $m$ in $X$ such that $m \leq x$ for all $x$ in $X$.
For instance $m = 0.01 \in X$ is not a minimum element of $X$
since $0.0001 \in X$
is smaller than $m$.
Also, $m = 0.0000001 \in X$ is also not a minimum of $X$ since
$0.0000000001 \in X$ is less than $m$.
In fact for any $m \in X$, $(1/2)m$ is in $X$ and is less than $m$.
In other words no value in $X$ can be a minimum value of $X$.

Now we will prove Theorem \ref{euclideanproperty3}.
  

\textit{Proof.}
TODO
\qed


\begin{ex} 
  \label{ex:hill-10}
  \tinysidebar{\debug{exercises/{rsa-43/question.tex}}}
  What if you test for the condition $n = x^3 - y^3$?
  Is there are version of
  factorization using difference of cubes?
  %We have
  %\[
  %  x^3 - y^3 = (x - y)(x^2 + xy + y^2)
  %\]
  %Suppose $n = abc$.
  %\[
  %  \frac{}{}
  %\]

  \solutionlink{sol:hill-10}
  \qed
\end{ex} 
\begin{python0}
from solutions import *
add(label="ex:hill-10",
    srcfilename='exercises/hill-10/answer.tex') 
\end{python0}



\begin{prop}
  Given $a,b$, the $q,r$ in
  Theorem \ref{euclideanproperty3}
  are unique.
  In other words,
  if
  \begin{align*}
    a &= bq+r, \,\,\, 0 \leq r < |b| \\
    a &= bq'+r', \,\,\, 0 \leq r' < |b|
  \end{align*}
  then
  \[
  q=q', \,\,\,\,\, r=r'
  \]
\end{prop}

\proof
TODO
\qed

Now I'm going to prove Theorem \ref{euclideanproperty}
which allows $a$ to be any integer.

\textit{Proof of Theorem \ref{euclideanproperty}}.
TODO
\qed


\begin{ex} 
  \label{ex:hill-10}
  \tinysidebar{\debug{exercises/{rsa-43/question.tex}}}
  What if you test for the condition $n = x^3 - y^3$?
  Is there are version of
  factorization using difference of cubes?
  %We have
  %\[
  %  x^3 - y^3 = (x - y)(x^2 + xy + y^2)
  %\]
  %Suppose $n = abc$.
  %\[
  %  \frac{}{}
  %\]

  \solutionlink{sol:hill-10}
  \qed
\end{ex} 
\begin{python0}
from solutions import *
add(label="ex:hill-10",
    srcfilename='exercises/hill-10/answer.tex') 
\end{python0}


\begin{ex} 
  \label{ex:hill-10}
  \tinysidebar{\debug{exercises/{rsa-43/question.tex}}}
  What if you test for the condition $n = x^3 - y^3$?
  Is there are version of
  factorization using difference of cubes?
  %We have
  %\[
  %  x^3 - y^3 = (x - y)(x^2 + xy + y^2)
  %\]
  %Suppose $n = abc$.
  %\[
  %  \frac{}{}
  %\]

  \solutionlink{sol:hill-10}
  \qed
\end{ex} 
\begin{python0}
from solutions import *
add(label="ex:hill-10",
    srcfilename='exercises/hill-10/answer.tex') 
\end{python0}


\begin{ex} 
  \label{ex:hill-10}
  \tinysidebar{\debug{exercises/{rsa-43/question.tex}}}
  What if you test for the condition $n = x^3 - y^3$?
  Is there are version of
  factorization using difference of cubes?
  %We have
  %\[
  %  x^3 - y^3 = (x - y)(x^2 + xy + y^2)
  %\]
  %Suppose $n = abc$.
  %\[
  %  \frac{}{}
  %\]

  \solutionlink{sol:hill-10}
  \qed
\end{ex} 
\begin{python0}
from solutions import *
add(label="ex:hill-10",
    srcfilename='exercises/hill-10/answer.tex') 
\end{python0}


\begin{ex} 
  \label{ex:hill-10}
  \tinysidebar{\debug{exercises/{rsa-43/question.tex}}}
  What if you test for the condition $n = x^3 - y^3$?
  Is there are version of
  factorization using difference of cubes?
  %We have
  %\[
  %  x^3 - y^3 = (x - y)(x^2 + xy + y^2)
  %\]
  %Suppose $n = abc$.
  %\[
  %  \frac{}{}
  %\]

  \solutionlink{sol:hill-10}
  \qed
\end{ex} 
\begin{python0}
from solutions import *
add(label="ex:hill-10",
    srcfilename='exercises/hill-10/answer.tex') 
\end{python0}


\begin{ex} 
  \label{ex:hill-10}
  \tinysidebar{\debug{exercises/{rsa-43/question.tex}}}
  What if you test for the condition $n = x^3 - y^3$?
  Is there are version of
  factorization using difference of cubes?
  %We have
  %\[
  %  x^3 - y^3 = (x - y)(x^2 + xy + y^2)
  %\]
  %Suppose $n = abc$.
  %\[
  %  \frac{}{}
  %\]

  \solutionlink{sol:hill-10}
  \qed
\end{ex} 
\begin{python0}
from solutions import *
add(label="ex:hill-10",
    srcfilename='exercises/hill-10/answer.tex') 
\end{python0}


\begin{ex} 
  \label{ex:hill-10}
  \tinysidebar{\debug{exercises/{rsa-43/question.tex}}}
  What if you test for the condition $n = x^3 - y^3$?
  Is there are version of
  factorization using difference of cubes?
  %We have
  %\[
  %  x^3 - y^3 = (x - y)(x^2 + xy + y^2)
  %\]
  %Suppose $n = abc$.
  %\[
  %  \frac{}{}
  %\]

  \solutionlink{sol:hill-10}
  \qed
\end{ex} 
\begin{python0}
from solutions import *
add(label="ex:hill-10",
    srcfilename='exercises/hill-10/answer.tex') 
\end{python0}



