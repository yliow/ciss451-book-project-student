\sectionthree{Primes}
\begin{python0}
from solutions import *; clear()
\end{python0}

\begin{defn}
A prime $p$ is a positive integer greater than 1 that is
divisible by only 1 and itself.
In other words $p \in \N$ is a
\defterm{prime}\index{prime}\tinysidebar{prime}
if $p > 1$ and 
if $d \in \Z, d > 0, d \mid p$, then $d = 1$ or $d = p$.
\end{defn}

Examples of primes are $2, 3, 5, 7, 11, 13, 17, 19, ...$.

Integers at least zero can be divided into the following types:
\begin{enumerate}[nosep]
  \li $0$ -- the zero element
  \li $1$ -- the unit element (i.e. the only invertible element $\geq 0$)
  \li primes -- $2, 3, 5, 7, 11, ...$
  \li \defone{composites} -- integers $> 1$ which are not primes 
\end{enumerate}

(Instead of primes of $\N \cup \{0\}$,
it's also possible to define primes of $\Z$.
A prime of $\Z$ is an integer not $-1, 0, 1$
such that if $d \mid p$, then $d = \pm 1$ or $d = \pm p$.
In that case primes of $\Z$ are
$\pm 2, \pm 3, \pm 5, \pm 7, \pm 11, ...$.)


For the next few theorems,
you may assume various properties of integers.
There is no circular arguments since since these properties
does not depend on the next few theorems.

\begin{prop} \textnormal{(Euclid)}
  There are infinitely many primes.
\end{prop}

\proof
TODO
\qed



\begin{ex} 
  \label{ex:hill-10}
  \tinysidebar{\debug{exercises/{rsa-43/question.tex}}}
  What if you test for the condition $n = x^3 - y^3$?
  Is there are version of
  factorization using difference of cubes?
  %We have
  %\[
  %  x^3 - y^3 = (x - y)(x^2 + xy + y^2)
  %\]
  %Suppose $n = abc$.
  %\[
  %  \frac{}{}
  %\]

  \solutionlink{sol:hill-10}
  \qed
\end{ex} 
\begin{python0}
from solutions import *
add(label="ex:hill-10",
    srcfilename='exercises/hill-10/answer.tex') 
\end{python0}


\begin{ex} 
  \label{ex:hill-10}
  \tinysidebar{\debug{exercises/{rsa-43/question.tex}}}
  What if you test for the condition $n = x^3 - y^3$?
  Is there are version of
  factorization using difference of cubes?
  %We have
  %\[
  %  x^3 - y^3 = (x - y)(x^2 + xy + y^2)
  %\]
  %Suppose $n = abc$.
  %\[
  %  \frac{}{}
  %\]

  \solutionlink{sol:hill-10}
  \qed
\end{ex} 
\begin{python0}
from solutions import *
add(label="ex:hill-10",
    srcfilename='exercises/hill-10/answer.tex') 
\end{python0}


\begin{ex} 
  \label{ex:hill-10}
  \tinysidebar{\debug{exercises/{rsa-43/question.tex}}}
  What if you test for the condition $n = x^3 - y^3$?
  Is there are version of
  factorization using difference of cubes?
  %We have
  %\[
  %  x^3 - y^3 = (x - y)(x^2 + xy + y^2)
  %\]
  %Suppose $n = abc$.
  %\[
  %  \frac{}{}
  %\]

  \solutionlink{sol:hill-10}
  \qed
\end{ex} 
\begin{python0}
from solutions import *
add(label="ex:hill-10",
    srcfilename='exercises/hill-10/answer.tex') 
\end{python0}


The above two exercises are special cases of the following theorem:

\begin{thm} \textnormal{(Dirichlet 1837)}
If $\gcd(a, b) = 1$, then there are infinitely many primes
of the form $an + b$.
\end{thm}

The proof of Dirichlet's theorem above uses group theory and
complex analysis and is beyond the scope of this book.

\begin{prop} \textnormal{(Euclid)}
  There are arbitrarily long consecutives integers of composites.
\end{prop}

\proof
\BEGINTODO
$n!+2$,
$n!+3$,
$n!+4$,
$\ldots$,
$n!+n$
are all composities for $n \geq 2$.
This list is therefore a list of consecutive composites of length
$n - 1$.
\ENDTODO
\qed

The follow lemma is extremely important and is used
for instance in the fundamental theorem of arithmetic to prove
the uniqueness of prime factorization in $\N$ (or $\Z$).
To break tradition, we will call this a theorem (instead of a lemma):

\begin{thm}
  \textnormal{(\defone{Euclid's lemma})}
  If $p$ is a prime and $p \mid ab$, then either $p \mid a$ or $p \mid b$.
\end{thm}

\proof
TODO
\qed



The above generalizes easily (by induction) to the following:


\begin{cor}
If $p$ is a prime and $p \mid a_1 a_2 \cdots a_n$, then 
$p$ divides at least one of the $a_1, \ldots, a_n$.
\end{cor}
\proof
TODO
\qed


\begin{thm} \textnormal{(Euclid's \defone{Fundamental Theorem of Arithmetic})}
Every positive integer $> 1$ can be written as a unique product of primes
up to permutation of the prime factors.
This means
\begin{enumerate}[nosep,label=\textnormal{(\alph*)}]
\item[\textnormal{(a)}] If $n > 1$ is an integer, then $n$ can be written as a product
of primes.
\item[\textnormal{(b)}] If $n$ is written as two products of primes:
\[
n = p_1 p_2 \cdots p_k = q_1 q_2 \cdots q_\ell
\]
where $p_i$ and $q_j$ are primes arranged in ascending order, i.e.,
\begin{align*}
p_1 \leq p_2 \leq \cdots \leq p_k \\
q_1 \leq q_2 \leq \cdots \leq q_\ell
\end{align*}
then $k = \ell$ and 
\[
p_1 = q_1, \,\,\,\,\,
p_2 = q_2, \,\,\,\,\,
\cdots, \,\,\,\,\,
p_k = q_k, \,\,\,\,\,
\]
\end{enumerate}
\end{thm}

\proof
TODO
\qed


The statement of the Fundamental Theorem of Arithmetic can include
the case of $n = 1$ if we accept that the product of an empty
tuple of integers is $1$:
\[
\prod_{p \in ()} p = 1
\]
Furthermore the statement can be slightly modified to include
negative integers:
Every nonzero integers can be written uniquely in the form
\[
u \prod_{i=1}^g p_i^{e_i}
\]
where $u = \pm 1$ (a unit), $p_i$ are distinct primes, $e_i > 0$ are integers.

\begin{prop} Let $a = \prod_{p \in P} p^{a_p}$, $b = \prod_{p
    \in P} p^{b_p}$ and $c = \prod_{p \in P} p^{c_p}$
  where $P$ is a finite set of primes. Then
\begin{enumerate}[nosep,label=\textnormal{(\alph*)}]
 \item $c = ab$ $\implies$ $c_p = a_p + b_p$.
 \item $a \mid b$ $\implies$ $a_p \leq b_p$ for all $p \in P$.
 \item $c = \gcd(a,b)$ $\implies$ $c_p = \min(a_p, b_p)$.
 \item $c = \lcm(a,b)$ $\implies$ $c_p = \max(a_p, b_p)$.
 \item $\gcd(a,b) \cdot \lcm(a,b) = ab$
\end{enumerate}
\end{prop}


Here, $\lcm(a b)$ is the \defone{lowest common multiple} of $a,b$, i.e.,
it is the smallest positive integer $m$ such that $a \mid m, b\mid m$.

The above assumes the easily proven facts that
\[
\prod_{p \in P} p^{a_p}
\prod_{p \in P} p^{b_p}
=
\prod_{p \in P} p^{a_p + b_p}
\]
(by commutativity of $\cdot$) and
\[
\prod_{p \in P} p^{a_p}
=
\prod_{p \in P} p^{b_p}
\implies
a_p = b_p \text{ for all $p \in P$}
\]
by uniqueness of prime factorization from the Fundamental Theorem of Arithmetic.
Here, $P$ is a set of distinct primes.


\begin{prop}
  If $n > 1$ is not a prime, then there is a prime factor $p$ such that
  $p \leq \sqrt{n}$.
\end{prop}
\proof
TODO
\qed

Therefore a very simple primality test algorithm for $n$ is the following:
  
\begin{Verbatim}[fontsize=\footnotesize,frame=single]
ALGORITHM: BRUTE-FORCE-PRIMALITY-TEST
INPUT: n
OUTPUT: true if n is prime. If n < 2, false is returned.

if n < 2: return false

d = 2
while d <= sqrt(n):
    if n % d == 0:
        return false
    d = d + 1
return true
\end{Verbatim}

In terms of $n$, the runtime is $O(\sqrt{n})$.
However in terms of the bits of $b$ of $n$, by Proposition 1.5.2,
\[
b = \floor{\log_2 ( n + 1 )} = \log_2(n + 1) - \alpha
\]
where $0 \leq \alpha < 1$.
Hence
\[
n = 2^b2^\alpha - 1
\]
Hence in terms of the number of bits of $n$,
the runtime is
\[
O(\sqrt{n}) = O(\sqrt{2^b2^\alpha - 1}) = O(2^{b/2})
\]
It is common to denote the number of bits of the input by $n$.
Hence the runtime in number of bits $n$ is
$O(2^{n/2})$, i.e., it has \defone{exponential runtime with linear exponent}.

$O(\sqrt{n})$ is said to be the \defone{pseudo-polynomial runtime} of the
algorithm
to indicate that the $n$ is the numeric input and not a
correct measure of the complexity of the input,
which should be in number of bits of the input.


\begin{ex} 
  \label{ex:hill-10}
  \tinysidebar{\debug{exercises/{rsa-43/question.tex}}}
  What if you test for the condition $n = x^3 - y^3$?
  Is there are version of
  factorization using difference of cubes?
  %We have
  %\[
  %  x^3 - y^3 = (x - y)(x^2 + xy + y^2)
  %\]
  %Suppose $n = abc$.
  %\[
  %  \frac{}{}
  %\]

  \solutionlink{sol:hill-10}
  \qed
\end{ex} 
\begin{python0}
from solutions import *
add(label="ex:hill-10",
    srcfilename='exercises/hill-10/answer.tex') 
\end{python0}


\begin{ex} 
  \label{ex:hill-10}
  \tinysidebar{\debug{exercises/{rsa-43/question.tex}}}
  What if you test for the condition $n = x^3 - y^3$?
  Is there are version of
  factorization using difference of cubes?
  %We have
  %\[
  %  x^3 - y^3 = (x - y)(x^2 + xy + y^2)
  %\]
  %Suppose $n = abc$.
  %\[
  %  \frac{}{}
  %\]

  \solutionlink{sol:hill-10}
  \qed
\end{ex} 
\begin{python0}
from solutions import *
add(label="ex:hill-10",
    srcfilename='exercises/hill-10/answer.tex') 
\end{python0}


\begin{ex} 
  \label{ex:hill-10}
  \tinysidebar{\debug{exercises/{rsa-43/question.tex}}}
  What if you test for the condition $n = x^3 - y^3$?
  Is there are version of
  factorization using difference of cubes?
  %We have
  %\[
  %  x^3 - y^3 = (x - y)(x^2 + xy + y^2)
  %\]
  %Suppose $n = abc$.
  %\[
  %  \frac{}{}
  %\]

  \solutionlink{sol:hill-10}
  \qed
\end{ex} 
\begin{python0}
from solutions import *
add(label="ex:hill-10",
    srcfilename='exercises/hill-10/answer.tex') 
\end{python0}


The following are some DIY exercises for self-study on famous unsolved problems in
number theory.
You might want to write programs to check on the conjectures.


\begin{ex} 
  \label{ex:hill-10}
  \tinysidebar{\debug{exercises/{rsa-43/question.tex}}}
  What if you test for the condition $n = x^3 - y^3$?
  Is there are version of
  factorization using difference of cubes?
  %We have
  %\[
  %  x^3 - y^3 = (x - y)(x^2 + xy + y^2)
  %\]
  %Suppose $n = abc$.
  %\[
  %  \frac{}{}
  %\]

  \solutionlink{sol:hill-10}
  \qed
\end{ex} 
\begin{python0}
from solutions import *
add(label="ex:hill-10",
    srcfilename='exercises/hill-10/answer.tex') 
\end{python0}


\begin{ex} 
  \label{ex:hill-10}
  \tinysidebar{\debug{exercises/{rsa-43/question.tex}}}
  What if you test for the condition $n = x^3 - y^3$?
  Is there are version of
  factorization using difference of cubes?
  %We have
  %\[
  %  x^3 - y^3 = (x - y)(x^2 + xy + y^2)
  %\]
  %Suppose $n = abc$.
  %\[
  %  \frac{}{}
  %\]

  \solutionlink{sol:hill-10}
  \qed
\end{ex} 
\begin{python0}
from solutions import *
add(label="ex:hill-10",
    srcfilename='exercises/hill-10/answer.tex') 
\end{python0}


\begin{ex} 
  \label{ex:hill-10}
  \tinysidebar{\debug{exercises/{rsa-43/question.tex}}}
  What if you test for the condition $n = x^3 - y^3$?
  Is there are version of
  factorization using difference of cubes?
  %We have
  %\[
  %  x^3 - y^3 = (x - y)(x^2 + xy + y^2)
  %\]
  %Suppose $n = abc$.
  %\[
  %  \frac{}{}
  %\]

  \solutionlink{sol:hill-10}
  \qed
\end{ex} 
\begin{python0}
from solutions import *
add(label="ex:hill-10",
    srcfilename='exercises/hill-10/answer.tex') 
\end{python0}


\begin{ex} 
  \label{ex:hill-10}
  \tinysidebar{\debug{exercises/{rsa-43/question.tex}}}
  What if you test for the condition $n = x^3 - y^3$?
  Is there are version of
  factorization using difference of cubes?
  %We have
  %\[
  %  x^3 - y^3 = (x - y)(x^2 + xy + y^2)
  %\]
  %Suppose $n = abc$.
  %\[
  %  \frac{}{}
  %\]

  \solutionlink{sol:hill-10}
  \qed
\end{ex} 
\begin{python0}
from solutions import *
add(label="ex:hill-10",
    srcfilename='exercises/hill-10/answer.tex') 
\end{python0}


\begin{ex} 
  \label{ex:hill-10}
  \tinysidebar{\debug{exercises/{rsa-43/question.tex}}}
  What if you test for the condition $n = x^3 - y^3$?
  Is there are version of
  factorization using difference of cubes?
  %We have
  %\[
  %  x^3 - y^3 = (x - y)(x^2 + xy + y^2)
  %\]
  %Suppose $n = abc$.
  %\[
  %  \frac{}{}
  %\]

  \solutionlink{sol:hill-10}
  \qed
\end{ex} 
\begin{python0}
from solutions import *
add(label="ex:hill-10",
    srcfilename='exercises/hill-10/answer.tex') 
\end{python0}


\begin{ex} 
  \label{ex:hill-10}
  \tinysidebar{\debug{exercises/{rsa-43/question.tex}}}
  What if you test for the condition $n = x^3 - y^3$?
  Is there are version of
  factorization using difference of cubes?
  %We have
  %\[
  %  x^3 - y^3 = (x - y)(x^2 + xy + y^2)
  %\]
  %Suppose $n = abc$.
  %\[
  %  \frac{}{}
  %\]

  \solutionlink{sol:hill-10}
  \qed
\end{ex} 
\begin{python0}
from solutions import *
add(label="ex:hill-10",
    srcfilename='exercises/hill-10/answer.tex') 
\end{python0}


\begin{ex} 
  \label{ex:hill-10}
  \tinysidebar{\debug{exercises/{rsa-43/question.tex}}}
  What if you test for the condition $n = x^3 - y^3$?
  Is there are version of
  factorization using difference of cubes?
  %We have
  %\[
  %  x^3 - y^3 = (x - y)(x^2 + xy + y^2)
  %\]
  %Suppose $n = abc$.
  %\[
  %  \frac{}{}
  %\]

  \solutionlink{sol:hill-10}
  \qed
\end{ex} 
\begin{python0}
from solutions import *
add(label="ex:hill-10",
    srcfilename='exercises/hill-10/answer.tex') 
\end{python0}


\begin{ex} 
  \label{ex:hill-10}
  \tinysidebar{\debug{exercises/{rsa-43/question.tex}}}
  What if you test for the condition $n = x^3 - y^3$?
  Is there are version of
  factorization using difference of cubes?
  %We have
  %\[
  %  x^3 - y^3 = (x - y)(x^2 + xy + y^2)
  %\]
  %Suppose $n = abc$.
  %\[
  %  \frac{}{}
  %\]

  \solutionlink{sol:hill-10}
  \qed
\end{ex} 
\begin{python0}
from solutions import *
add(label="ex:hill-10",
    srcfilename='exercises/hill-10/answer.tex') 
\end{python0}


\begin{ex} 
  \label{ex:hill-10}
  \tinysidebar{\debug{exercises/{rsa-43/question.tex}}}
  What if you test for the condition $n = x^3 - y^3$?
  Is there are version of
  factorization using difference of cubes?
  %We have
  %\[
  %  x^3 - y^3 = (x - y)(x^2 + xy + y^2)
  %\]
  %Suppose $n = abc$.
  %\[
  %  \frac{}{}
  %\]

  \solutionlink{sol:hill-10}
  \qed
\end{ex} 
\begin{python0}
from solutions import *
add(label="ex:hill-10",
    srcfilename='exercises/hill-10/answer.tex') 
\end{python0}


\begin{ex} 
  \label{ex:hill-10}
  \tinysidebar{\debug{exercises/{rsa-43/question.tex}}}
  What if you test for the condition $n = x^3 - y^3$?
  Is there are version of
  factorization using difference of cubes?
  %We have
  %\[
  %  x^3 - y^3 = (x - y)(x^2 + xy + y^2)
  %\]
  %Suppose $n = abc$.
  %\[
  %  \frac{}{}
  %\]

  \solutionlink{sol:hill-10}
  \qed
\end{ex} 
\begin{python0}
from solutions import *
add(label="ex:hill-10",
    srcfilename='exercises/hill-10/answer.tex') 
\end{python0}


\begin{ex} 
  \label{ex:hill-10}
  \tinysidebar{\debug{exercises/{rsa-43/question.tex}}}
  What if you test for the condition $n = x^3 - y^3$?
  Is there are version of
  factorization using difference of cubes?
  %We have
  %\[
  %  x^3 - y^3 = (x - y)(x^2 + xy + y^2)
  %\]
  %Suppose $n = abc$.
  %\[
  %  \frac{}{}
  %\]

  \solutionlink{sol:hill-10}
  \qed
\end{ex} 
\begin{python0}
from solutions import *
add(label="ex:hill-10",
    srcfilename='exercises/hill-10/answer.tex') 
\end{python0}


\begin{ex} 
  \label{ex:hill-10}
  \tinysidebar{\debug{exercises/{rsa-43/question.tex}}}
  What if you test for the condition $n = x^3 - y^3$?
  Is there are version of
  factorization using difference of cubes?
  %We have
  %\[
  %  x^3 - y^3 = (x - y)(x^2 + xy + y^2)
  %\]
  %Suppose $n = abc$.
  %\[
  %  \frac{}{}
  %\]

  \solutionlink{sol:hill-10}
  \qed
\end{ex} 
\begin{python0}
from solutions import *
add(label="ex:hill-10",
    srcfilename='exercises/hill-10/answer.tex') 
\end{python0}


\begin{ex}
Verify by hand or by using a program that if
\[
P(x) = x^2 - x + 41
\]
then $P(0), P(1), ..., P(40)$ are all primes.
This \lq\lq prime generating" polynomial was first discovered by
Euler.
Find positive integer $n$ such that
$x^2 - x + n$ are primes for $0 \leq x < n$.
Such an $n$ is called an Euler lucky number.
41 is an example of an Euler lucky number.
\end{ex}

\begin{ex}
Prove that if $P(x)$ is a non-constant polynomial, then
it is impossible for all $P(0)$, $P(1)$, $P(2)$, ... to be prime.
In fact there are infinitely many $n$ such that $P(n)$ are all
composites.
\end{ex}
