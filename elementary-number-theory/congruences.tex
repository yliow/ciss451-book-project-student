%-*-latex-*-
\sectionthree{Congruences}
\begin{python0}
from solutions import *; clear()
\end{python0}

\begin{defn}
  Let $a, b \in \Z$ and $N \in \Z$ with $N > 0$.
  Then $a$ is congruent to $b$ mod $N$ and we write
  \[
  a \equiv b \pmod{N}
  \]
  if $N \mid a - b$.
  In the above expression
  \[
  a \equiv b \pmod{N}
  \]
  we say that $N$ is the modulus of the above relation between $a$ and $b$.
\end{defn}

\begin{prop}
  Let $a,b,c,a',b' \in \Z$ and $N,N' \geq$ be in $\Z$.
  \begin{myenum}       
  \item \textnormal{Reflexivity:}
    $a \equiv a \pmod{N}$
  \item \textnormal{Symmetry:}
    If $a \equiv b \pmod{N}$, then $b \equiv a \pmod{N}$
  \item \textnormal{Transitivity:}
    If $a \equiv b, b \equiv c \pmod{N}$, then
    $a \equiv c \pmod{N}$
  \item \textnormal{Additivity:}
    If
    $a \equiv b$,
    $a' \equiv b' \pmod{N}$,
    then
    $a + a' \equiv b + b' \pmod{N}$.
  \item \textnormal{Multiplicativity:}
    If
    $a \equiv b$,
    $a' \equiv b' \pmod{N}$,
    then
    $a a' \equiv b b' \pmod{N}$.
  \item
    If
    $a \equiv b \pmod{NN'}$, then
    $a \equiv b \pmod{N}$
  \end{myenum}
\end{prop}
\proof
TODO
\qed

The following connects the Euclidean property and the congruence relation:

\begin{prop}
  Let $a, N \in \Z$ with $N > 0$.
  Let $q, r \in \Z$ such that
  \[
  a = Nq + r, \,\,\, 0 \leq r < N
  \]
  Then $a \equiv r \pmod{N}$.
\end{prop}
\proof
TODO

\begin{defn}
  Let $a, N \in \Z$ with $N > 0$.
  By the Euclidean property of $\Z$, there exist unique $q,r$ such that
  \[
  a = Nq + r, \,\,\, 0 \leq r < N
  \]
  $r$ is called the \lq\lq \defterm{residue} of $a$ mod $N$"
  (\lq\lq residue" = \lq\lq what is left" after $a$ is divided by $N$, i.e.,
  the remainder after $a$ is divided by $N$).
  It is common to write $r$ as $a$ mod $N$.
\end{defn}

Sometimes $a$ mod $N$ is written as $r_N(a)$.
For instance to find the residue of 15 mod 4, there is some $q$ such that
\[
15 = 4q + 3, \,\,\, 0 \leq 1 < 4
\]
i.e.
\[
15 \equiv 3 \pmod{4}
\]
where $0 \leq 1 < 4$.
Therefore the residue of 15 mod 4 is 1, or we simple write
\[
15 \text{ mod } 4 = 3
\]
i.e., $r_4(15) = 3$.

\textsc{Warning}: \lq\lq mod" now has two meanings.
\lq\lq mod $N$", where $N > 0$ is an integer,
can be used to denote a relation between integers
\[
a \equiv b \pmod{N}
\]
and \lq\lq mod $N$" can also be used to denote a function
\[
a \text{ mod } N = r
\]


\begin{ex} 
  \label{ex:hill-10}
  \tinysidebar{\debug{exercises/{rsa-43/question.tex}}}
  What if you test for the condition $n = x^3 - y^3$?
  Is there are version of
  factorization using difference of cubes?
  %We have
  %\[
  %  x^3 - y^3 = (x - y)(x^2 + xy + y^2)
  %\]
  %Suppose $n = abc$.
  %\[
  %  \frac{}{}
  %\]

  \solutionlink{sol:hill-10}
  \qed
\end{ex} 
\begin{python0}
from solutions import *
add(label="ex:hill-10",
    srcfilename='exercises/hill-10/answer.tex') 
\end{python0}

