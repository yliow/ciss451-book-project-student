\sectionthree{Congruence classes}
\begin{python0}
from solutions import *; clear()
\end{python0}

I want to view modulo arithmetic in a different way.
This section requires knowledge of equivalence relations and
equivalence classes.

Recall that for $N > 1$, we think of $\Z/N$ as the set
\[
\Z/N = \{0, 1, 2, ..., N - 1\}
\]
where we can add, multiply, etc.
There are other symbols in this system.
For instance I can write $N$.
But in this system,
\[
N \equiv 0 \pmod{N}
\]
In other words, informally, in this system $N$ is just another notation
for $0$.
That's the point of the congrence notation $\equiv \pmod{N}$.

Recall the concept of relations on a set $X$.
If a relation $R$ is an equivalence relation on $X$,
then $X$ can be partitioned into subsets.

The $\equiv \pmod N$ relation on $\Z$ is an equivalence relation.
For instance $\equiv \pmod 6$ is an equivalence relation on $\Z$.
Therefore this partitions $\Z$ into disjoint subsets.
In fact the disjoint subsets are
\begin{align*}
&\{-12, -6, 0, 6, 12, ...\} \\
&\{-11, -5, 1, 7, 13, ...\} \\
&\{-10, -4, 2, 8, 14, ...\} \\
&\{-9, -3, 3, 9, 15, ...\} \\
&\{-8, -2, 4, 10, 16, ...\} \\
&\{-7, -1, 5, 11, 17 ...\} 
\end{align*}
I'll give these subsets of $\Z$ names:
\begin{align*}
[0]_6 &= \{-12, -6, 0, 6, 12, ...\} \\
[1]_6 &= \{-11, -5, 1, 7, 13, ...\} \\
[2]_6 &= \{-10, -4, 2, 8, 14, ...\} \\
[3]_6 &= \{-9, -3, 3, 9, 15, ...\} \\
[4]_6 &= \{-8, -2, 4, 10, 16, ...\} \\
[5]_6 &= \{-7, -1, 5, 11, 17 ...\} 
\end{align*}
These sets are called \defone{congruences classes} of $\equiv \pmod{6}$.
Note that when I write
\[
[0]_6
\]
I mean the congruence class that contains $0$.
I could have chosen $6$ and write
\[
[6]_6
\]
However
\[
[0]_6 = [6]_6
\]
Right?

You can think of $\Z/6$ as the set of these congruence classes:
\[
\Z/6 = \{[0]_6, [1]_6, [2]_6, [3]_6, [4]_6, [5]_6\}
\]
When write
\[
2 + 4 \equiv 0 \pmod{6}
\]
we can rephrase that as 
\[
[2]_6 + [4]_6 = [0]_6
\]
In other words we define $+_6$ and $\cdot_6$ on these congruence classes
as
\begin{align*}
[a]_6 +_6 [b]_6 &= [a + b]_6 \\
[a]_6 \cdot_6 [b]_6 &= [a \cdot b]_6 \\
\end{align*}
The $+_6$ and $\cdot_6$ on the left are operations on the congruence classes
of $\Z/6$.
The $+$ and $\cdot$ on the right are the usual operations in $\Z$.
I will usually write $+$ for $+_6$
and $\cdot$ for $\cdot_6$.

With these two operations, the set $Z\/6$ becomes
\[
(\Z/6, +_6, \cdot_6, [0]_6, [1]_6)
\]
which satisfies the same set of axioms of $(\Z, +, \cdot, 0, 1)$:
CAIN for addition,
CAN for multiplication, and distribution.
But not the integrality, topology axioms.
In other words
$(\Z, +, \cdot, 0, 1)$ and 
$(\Z/6, +_6, \cdot_6, [0]_6, [1]_6)$
are both ring.
In particular, since multiplication for both rings are commutative,
both are called commutative rings.
Every formula of the form
\[
x \equiv y \pmod{6}
\]
can be re-expressed as
\[
[x]_6 = [y]_6
\]

The above is true when $6$ is replaced by any $N > 0$.

Note that whereas $\Z$ is an infinite ring, $\Z/N$ is a finite
ring of size $N$.
Furthermore, there is a map
\[
\Z \rightarrow \Z/N
\]
given by
\[
x \mapsto [x]_N
\]

A ring is a \defone{field}
if every nonzero element has a multiplicative inverse.
Therefore if $p > 1$, then $\Z/p$ is a field iff $p$ is prime.

